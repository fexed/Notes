\documentclass[10pt]{book}
\usepackage[utf8]{inputenc}
\usepackage[italian]{babel}
\usepackage{multicol}
\usepackage[bookmarks]{hyperref}
\usepackage[a4paper, total={18cm, 25cm}]{geometry}
\usepackage{listings}
\usepackage{graphicx}
\usepackage{makecell}
\graphicspath{ {./img/} }
\usepackage{color}

\begin{document}
\renewcommand*\contentsname{Indice}
\title{Gestione di Reti}
\author{Federico Matteoni}
\date{ }
\maketitle
\tableofcontents
\pagebreak
\section*{Introduzione}
...
\chapter{lezione 2}
Livello 2 consente di identificare un device sulla rete. In tutte le reti c'è la necessità di identificare la porta di rete. Ogni dispositivo ha almeno un'interfaccia di rete: loopback, che consente di far comunicare processi di rete sulla stessa macchina. \texttt{127.0.0.1} consente di parlare su stessa macchina senza trasmettere sul filo, fondamentalmente un cortocircuito.\\
\texttt{ifconfig} consente di vedere le interfacce di rete disponibili su unix.\\
Se si vuole gestire una rete è fondamentale la standardizzazione.\\\\
Output \texttt{ifconfig}. Parte degli indirizzi, no indirizzo hw su loopback perché il traffico non esce mai (loopback sulla pila OSI è nel livello 3 Network, il MAC address è sul livello 2 Data Link, che non viene toccato da loopback). Indirizzo MAC 6 byte divisi in blocchi dai due punti. I primi 3 identificano il costruttore della scheda di rete. I successivi tre identificano la scheda di rete per il costruttore, che lo setta univocamente. Ciò garantisce univocità. Per primo blocco di tre ho 16M di dispositivi possibili. I MAC address quindi \textbf{non sono univoci}, lo sono \textit{probabilmente}. L'univocità è fondamentale sulla stessa rete. Quindi indirizzo hw identifica univocamente device sulla rete locale. divisi in due blocchi, il primo identifica costruttore della scheda di rete.\\\\
Qualsiasi dispositivo ha indirizzo hw diverso per ciascuna porta.
\section{Ethernet}
Ethernet è un cavo seriale, trasmissione e ricezione. Mezzo seriale. Un filo.\\
Quando si mandano dati non posso tutti insieme ma man mano. Non c'è collisione perché ricezione e trasmissione sono su due fili separati.\\
Pacchetti inviati nel tempo sul filo. Vengono distinti tra loro dal \texttt{preamble}. Pacchetti inviati in una direzione: preambolo, destinazione, sorgente, tipo dei dati, dati effettivi, padding (per rendere pacchetto di 64 se pacchetto è troppo corto), CRC.\\
Quindi per spedire pacchetto necessito di indirizzi (chi voglio e chi sono) e cosa mandare. chi sono lo so, è scritto nella scheda. Voglio conoscere indirizzo di chi voglio.\\
Alla connessione del cavo, se DHCP manda fuori pacchetto per richiesta quindi switch lo impara, se IP statico manda pacchetto ARP quindi switch lo impara.\\
MAC address randomizzato per privacy, spesso e volentieri sui dispositivi mobili.
\end{document}