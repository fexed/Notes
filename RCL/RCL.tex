\documentclass[10pt]{article}
\usepackage[utf8]{inputenc}
\usepackage[italian]{babel}
\usepackage{multicol}
\usepackage[a4paper, total={18cm, 25cm}]{geometry}
\begin{document}
\title{Reti di Calcolatori e Laboratorio}
\author{Federico Matteoni}
\date{ }
\renewcommand*\contentsname{Indice}

\maketitle
\tableofcontents
\pagebreak
\section{Introduzione}
Appunti del corso di \textbf{Reti di Calcolatori} presi a lezione da \textbf{Federico Matteoni}.\\\\
Prof.: \textbf{Federica Paganelli}, federica.paganelli@unipi.it\\
\begin{list}{-}{Riferimenti web:}
\item \emph{elearning.di.unipi.it/enrol/index.php?id=109}\\Password: \textbf{RETI2019}
\end{list}
Esame: scritto (o compitini), discussione orale facoltativa + progetto con discussione (progetto + teoria di laboratorio, progetto da consegnare 7gg prima della discussione)\\
\begin{list}{-}{Libri e materiale didattico:}
\item Slide su eLearning
\item IETF RFC\\tools.ietf.org/rfc\\www.ietf.org/rfc.html
\item "Computer Networks: A Top-Down Approach" B. A. Forouzan, F. Mosharraf, McGraw Hill
\end{list}
Ricevimento: stanza 355 DO, II piano

\section{Obiettivi e programma}
Obiettivi: concetti chiave delle reti con TCP/IP come riferimento

\section{Modelli Stratificati}
\paragraph{Perché un modello a strati} Per mandare dati da un host all'altro comunicando su rete si devono eseguire una serie di operazioni: trovare il percorso di rete da attraversare, decidere in che modo spedire e codificare i dati, risolvere eventuali problemi di comunicazione e altro ancora. Programmare ogni volta tutto il procedimento è un lavoro estremamente complesso e ripetitivo. Il modello a strati \textbf{astrae su più livelli il problema della trasmissione dati} in modo da fornire di volta in volta strumenti al programmatore per poter evitare di "\textit{reinventare la ruota}".
\paragraph{Definizioni generali} Nell'architetture di comunicazione a strati sono importanti una serie di definizioni:
\begin{list}{-}{}
\item Stratificazione
\item Information hiding
\item Separation of concern
\item Modello ISO/OSI
\item Stack TCP/IP
\end{list}
Tali definizioni verranno viste durante il corso.

\subsubsection{Stack TCP/IP}
\paragraph{Livello Applicativo} Il livello più alto, con il quale interagisce l'utente\\
Identificativi risorse: URL, URI, URN\\
Il web: user agents, http: request, response, connessioni persistenti, GET, POST, PUT, DELETE, status code, proxy server, caching\\
FTP: connessioni dati e di controllo, rappresentazione\\
TELNET\\
Posta elettronica: SMTP, POP3, IMAP\\
DNS e risoluzioni nomi: gerarchia nomi, risoluzione iterativa e ricorsiva, formato pessaggi, nslookup...\\
\paragraph{Livello Trasporto} Livello al quale si definisce la codifica e il protocollo di trasporto\\
Servizi: mux demux, co ntrollo errore, connectionless\\
TCP: formato segmenti, gestione connessione, controllo flusso e congestione\\
UDP: formato segmenti\\
\paragraph{Livello Rete} Dove si gestisce l'indirizzamento dei vari host\\
strato di rete e funzioni\\
indirizzamentoi ip: classful IPv4, NAT, sottoreti e maschere, classless, CIDR\\
risoluzione IP e MAC, ARP\\
IPv4: formato datagramma ip, frammentazione\\
routing IP e istradamento\\
introduzione IPv6\\
\paragraph{Link} Il livello più basso, dove avviene la vera e propria comunicazione a livello elettrico\\
Cenni livello link\\
Ethernet\\

\section{Rete}
\paragraph{Rete, definizione} Interconnessione di dispositivi in grado di scambiarsi informazioni, come terminali (\textbf{end system}), router, switch e modem.\\
I \textbf{sistemi terminali} possono essere host o server:
\begin{list}{-}{}
\item \textbf{Host}: macchina in genere di proprietà degli utenti, dedicata ad eseguire applicazioni (desktop, laptop, smartphone...)
\item \textbf{Server}: macchina con elevate prestazioni destinato a fornire servizi a diverse applicazioni (es. e-mail, web...)
\end{list}
Con il termine host si può anche indicare un server, per questo \textbf{è usato come termine generico per indicare un dispositivo connesso}.

\subsection{Tipi di Rete}
\paragraph{Local Area Network} Una \textbf{LAN è una rete di area geografica limitata}: un ufficio, una casa ecc.. I dispositivi comunicano attraverso una determinata tecnologica: switch, BUS, HUB ecc..\\
In una rete locale tipicamente una serie di host comunicano tra loro, ad esempio, attraverso uno switch centrale.
\paragraph{Wide Area Network}

\subsection{Internetwork}
Reti collegate fra loro: es 4 WAN collegate punto a punto e tre LAN collegate alle WAN

\subsection{Switching}
Una rete internet è formata dall'interconnesione di reti composte da link e dispositivi capaci di scambiarsi informazioni.
In particolare, i sistemi terminali comunicano tra di loro per mezzo di dispositivi come switch, router ecc. che si trovano nel percorso tra i sistemi sorgente e destinazione.
\paragraph{Switched Network} Reti a commutazione di circuito, tipico delle vecchie reti telefoniche\\
Le risorse sono riservate end-to-end per una connessione. Le risorse di rete (es. bandwidth) vengono suddivise in pezzi, e ciascun pezzo è allocato ai vari collegamenti. Le risorse rimangono inattive se non vengono utilizzate, cioè \textbf{non c'è condivisione}. L'allocazione della rete rende necessario un setup della comunicazione.\\A tutti gli effetti vi è un circuito dedicato per tutta la durata della connessione. Ciò è rende poco flessibile l'utilizzo delle risorse (\textbf{overprovisioning}).
\paragraph{Packet-Switched Network} Reti a commutazione di pacchetto, più moderno\\
Flusso di dati punto-punto suddiviso in pacchetti. I pacchetti degli utenti condividono le risorse di rete. Ciascun pacchetto utilizza completamente il canale.\\\textbf{Store and Forward}: il commutatore deve ricevere l'intero pacchetto prima di ritrasmetterlo in uscita.\\Le risorse vengono usate \textbf{a seconda delle necessità}. Vi è \textbf{contesa per le risorse}: la richiesta di risorse può eccedere la disponibilità e si può verificare \textbf{congestione} quando i pacchetti vengono accodati in attesa di utilizzare il collegamento. Si possono anche verificare perdite.



\end{document}