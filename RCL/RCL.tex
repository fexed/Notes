\documentclass[10pt]{article}
\usepackage[utf8]{inputenc}
\usepackage[italian]{babel}
\usepackage{multicol}
\usepackage[a4paper, total={18cm, 25cm}]{geometry}
\usepackage{listings}
\begin{document}
\title{Reti di Calcolatori e Laboratorio}
\author{Federico Matteoni}
\date{ }
\renewcommand*\contentsname{Indice}

\maketitle
\tableofcontents
\pagebreak
\section{Introduzione}
Appunti del corso di \textbf{Reti di Calcolatori} presi a lezione da \textbf{Federico Matteoni}.\\\\
Prof.: \textbf{Federica Paganelli}, federica.paganelli@unipi.it\\
\begin{list}{-}{Riferimenti web:}
\item \emph{elearning.di.unipi.it/enrol/index.php?id=109}\\Password: \textbf{RETI2019}
\end{list}
Esame: scritto (o compitini), discussione orale facoltativa + progetto con discussione (progetto + teoria di laboratorio, progetto da consegnare 7gg prima della discussione)\\
\begin{list}{-}{Libri e materiale didattico:}
\item Slide su eLearning
\item IETF RFC\\tools.ietf.org/rfc\\www.ietf.org/rfc.html
\item "Computer Networks: A Top-Down Approach" B. A. Forouzan, F. Mosharraf, McGraw Hill
\end{list}
Ricevimento: stanza 355 DO, II piano

\section{Obiettivi e programma}
Obiettivi: concetti chiave delle reti con TCP/IP come riferimento

\section{Modelli Stratificati}
\paragraph{Perché un modello a strati} Per mandare dati da un host all'altro comunicando su rete si devono eseguire una serie di operazioni: trovare il percorso di rete da attraversare, decidere in che modo spedire e codificare i dati, risolvere eventuali problemi di comunicazione e altro ancora. Programmare ogni volta tutto il procedimento è un lavoro estremamente complesso e ripetitivo. Il modello a strati \textbf{astrae su più livelli il problema della trasmissione dati} in modo da fornire di volta in volta strumenti al programmatore per poter evitare di "\textit{reinventare la ruota}".
\paragraph{Definizioni generali} Nell'architetture di comunicazione a strati sono importanti una serie di definizioni:
\begin{list}{-}{}
\item Stratificazione
\item Information hiding
\item Separation of concern
\item Modello ISO/OSI
\item Stack TCP/IP
\end{list}
Tali definizioni verranno viste durante il corso.

\subsubsection{Stack TCP/IP}
\paragraph{Livello Applicativo} Il livello più alto, con il quale interagisce l'utente\\
Identificativi risorse: URL, URI, URN\\
Il web: user agents, http: request, response, connessioni persistenti, GET, POST, PUT, DELETE, status code, proxy server, caching\\
FTP: connessioni dati e di controllo, rappresentazione\\
TELNET\\
Posta elettronica: SMTP, POP3, IMAP\\
DNS e risoluzioni nomi: gerarchia nomi, risoluzione iterativa e ricorsiva, formato pessaggi, nslookup...\\
\paragraph{Livello Trasporto} Livello al quale si definisce la codifica e il protocollo di trasporto\\
Servizi: mux demux, co ntrollo errore, connectionless\\
TCP: formato segmenti, gestione connessione, controllo flusso e congestione\\
UDP: formato segmenti\\
\paragraph{Livello Rete} Dove si gestisce l'indirizzamento dei vari host\\
strato di rete e funzioni\\
indirizzamentoi ip: classful IPv4, NAT, sottoreti e maschere, classless, CIDR\\
risoluzione IP e MAC, ARP\\
IPv4: formato datagramma ip, frammentazione\\
routing IP e istradamento\\
introduzione IPv6\\
\paragraph{Link} Il livello più basso, dove avviene la vera e propria comunicazione a livello elettrico\\
Cenni livello link\\
Ethernet\\

\section{Rete}
\paragraph{Rete, definizione} Interconnessione di dispositivi in grado di scambiarsi informazioni, come terminali (\textbf{end system}), router, switch e modem.\\
I \textbf{sistemi terminali} possono essere host o server:
\begin{list}{-}{}
\item \textbf{Host}: macchina in genere di proprietà degli utenti, dedicata ad eseguire applicazioni (desktop, laptop, smartphone...)
\item \textbf{Server}: macchina con elevate prestazioni destinato a fornire servizi a diverse applicazioni (es. e-mail, web...)
\end{list}
Con il termine host si può anche indicare un server, per questo \textbf{è usato come termine generico per indicare un dispositivo connesso}.

\subsection{Tipi di Rete}
\paragraph{Local Area Network} Una \textbf{LAN è una rete di area geografica limitata}: un ufficio, una casa ecc.. I dispositivi comunicano attraverso una determinata tecnologica: switch, BUS, HUB ecc..\\
In una rete locale tipicamente una serie di host comunicano tra loro, ad esempio, attraverso uno switch centrale.
\paragraph{Wide Area Network}

\subsection{Internetwork}
Reti collegate fra loro: es 4 WAN collegate punto a punto e tre LAN collegate alle WAN

\subsection{Switching}
Una rete internet è formata dall'interconnesione di reti composte da link e dispositivi capaci di scambiarsi informazioni.
In particolare, i sistemi terminali comunicano tra di loro per mezzo di dispositivi come switch, router ecc. che si trovano nel percorso tra i sistemi sorgente e destinazione.
\paragraph{Switched Network} Reti a commutazione di circuito, tipico delle vecchie reti telefoniche\\
Le risorse sono riservate end-to-end per una connessione. Le risorse di rete (es. bandwidth) vengono suddivise in pezzi, e ciascun pezzo è allocato ai vari collegamenti. Le risorse rimangono inattive se non vengono utilizzate, cioè \textbf{non c'è condivisione}. L'allocazione della rete rende necessario un setup della comunicazione.\\A tutti gli effetti vi è un circuito dedicato per tutta la durata della connessione. Ciò è rende poco flessibile l'utilizzo delle risorse (\textbf{overprovisioning}).
\paragraph{Packet-Switched Network} Reti a commutazione di pacchetto, più moderno\\
Flusso di dati punto-punto suddiviso in pacchetti. I pacchetti degli utenti condividono le risorse di rete. Ciascun pacchetto utilizza completamente il canale.\\\textbf{Store and Forward}: il commutatore deve ricevere l'intero pacchetto prima di ritrasmetterlo in uscita.\\Le risorse vengono usate \textbf{a seconda delle necessità}. Vi è \textbf{contesa per le risorse}: la richiesta di risorse può eccedere la disponibilità e si può verificare \textbf{congestione} quando i pacchetti vengono accodati in attesa di utilizzare il collegamento. Si possono anche verificare perdite.

16/09/19
Chiave elearning: RETI2019
Introduzione
Una rete è un’interconnessione di dispositivi in grado di scambiarsi informazioni, quali sistemi terminali (end system), router, switch e modem.
I sistemi terminali possono essere di due tipi: host o server.
    • Un host è una macchina in genere di proprietà degli utenti e dedicata ad eseguire applicazioni, quale un computer desktop, un portatile, un cellulare o un tablet.
    • Un server è tipicamente un computer con elevate prestazioni destinato a eseguire programmi che forniscono servizio a diverse applicazioni utente come, per esempio, la posta elettronica o il web.
Il termine host può essere usato anche per indicare un server.

Una internet è data dall’interconnessione di reti, composte da link e dispositivi capaci di scambiarsi informazioni.
In particolare, i dispositivi si distinguono in sistemi terminali che comunicano tra di loro per mezzo di dispositivi come switch e router che si trovano nel percorso o rotta tra i sistemi sorgente e destinazione. Prendiamo in analisi due tipi di circuiti:
    • Circuit-switched network o reti a commutazione di circuito.
      Usano risorse riservate end to end per una connessione, quindi le risorse di rete (larghezza di banda, bandwidth) sono suddivise a “pezzi”. A ciascun “pezzo” viene allocato ai vari collegamenti; le risorse rimangono inattive se non utilizzate. La performance è garantita dal tipo di circuito. È necessario il setup della comunicazione
    • Packet-switching network o reti a commutazione di pacchetto.
      Il flusso dei dati punto-punto viene suddiviso in pacchetti. I pacchetti degli utenti A e B condividono le risorse di rete, e ciascun pacchetto utilizza completamente il canale.
      Proprietà di store and forward: il commutatore (come ad esempio un router) deve ricevere l’intero pacchetto prima di poter iniziare a trasmettere sul collegamento in uscita.
      In tale contesto le risorse vengono usate a seconda delle necessità, non c’è uno spreco di risorse se gli utenti sono inattivi. Di contro abbiamo un contesa per le risorse: la richiesta di risorse può eccedere il quantitativo disponibile, quindi è possibile che ci sia un problema di congestione cioè un accodamento di pacchetti che rimangono in attesa per l’utilizzo del collegamento.
      

17/09/19
Internet
Una internet è costituita da due o più reti interconnesse.
L’internet più famosa è chiamata Internet ed è composta da migliaia di reti interconnesse.
Ogni rete connessa a Internet deve usare IP e rispettare certe convenzioni su nomi e indirizzi. Nuove reti si aggiungono facilmente.
$I dispositivi connessi possono essere host, end_system come pc, workstation, server, pda, telefoni ect.$
I link di comunicazione possono essere fibre ottiche, doppini telefonici, cavi coassiali, onde radio.
I router instradano i pacchetti (sequenze) di dati attraverso la rete.
Le entità software possono essere:
    • applicazioni e processi
    • Protocolli: regolamentano la trasmissione e la ricezione di messaggi (tcp, ip, http, ftp, ppp)
    • Interfacce: definite in seguito, sono le membrane che separano gli strati.
    • Standard Internet e del web: RFC (Request for comments), W3C.

Che cosa è internet? Una visione dei servizi.
L’infrastruttura di comunicazione permette le applicazioni distribuite per scambio di informazioni (www, email giochi, e-commerce, database, controllo remoto, etc).
Fornisice servizi di comunicazione per le applicazioni: connectionless (senza garaznia di consegna) o connection-oriented (garantiti in integrità, completezza ed ordine).

IETF l’organismo che studia e sviluppa i protocollo in uso su internet. Si basa su gruppi di lavoro a cui chiunque può accede
RFC/STD I documenti ufficiali
…
(slide con vista gerarchica di internet)
Internet è una internetwork che consente a qualsiasi utente di farne parte.
L’utente tuttavia deve essere fisicamente collegatoa un ISP.
Il collegamento che connette l’utente al primo router di Internet è detto rete di accesso.
Rete di accesso:
    • accesso via rete telefonica
    • ADSL
    • accesso tramite reti wireless o reti mobili
    • Collegamento diretto: collegamento WAN dedicati ad alta velocità (aziende, università).

Metriche di riferimento
Le prestazioni della rete si misurano in:
    • larghezza di banda
    • throughtput
    • latenza

Larghezza di banda (bandwidth): larghezza dell’intervallo di frequenze utilizzato dal sistema trasmissivo misurato in Hz.
Bit o trasmission rate: quantità di dati (bit) che possono essere trasmessi (“inseriti nella linea”) o ricevuti nell’unità di tempo (bit/secondo o bps).
Bitrate dipende dalla bandwidth e dalla tecnica trasmissiva usata.

Il throughput è la quantità di traffico che arriva realmente a destinazione nell’unità di tempo. Non + corrispondente uno a uno alla larghezza di banda poiché ci potrebbe essere un collo di bottiglia nela connessione (vedi slides per chiarezza).

La latenza è il tempo richiesto affinché un messaggio arrivi a destinazione dal moneto in cui il primo bit parte dalla sorgente.
Latenza = ritardo di propagazione + ritardo di trasmissione + ritardo di accodamento + ritardo di elaborazione.

I pacchetti si accodano nei buffer dei router. Il tasso di arrivo dei pacchetti sul collegamento eccede la capacità del collegamento di evaderli. In questo modo i pacchetti si accodano in attesa del proprio turno. (vedi slides).
Ritardo di elaborazione del nodo è dato da un controllo sui bit e dalla determinazione del canale di uscita (errore trascurabile).
Ritardo di accodamento: attesa di trasmissione…
Ritardo di trasmissione tempo impiegato per trasmette un pacchetto sul link. R = rete di trasmissione del collegamento in bps. L = lunghezza del pacchetto in bit. Ritardo di trasmissione = L/R
Ritardo di propagazione d/s tempo impiegato da 1 bit per essere propagato da un nodo all’altro.
$D = lunghezza del collegamento, s = velocita di propagazione del collegamento (si usa la velocita della luce circa 3-2x10^8 m/sec). Ritardo di propagazione = d/s$
(slides di esempio)
(slides analogia del casello autostradale)
(slides di ricapitolazione)

\section{Modelli Stratificati}
\subsection{Protocollo}
\paragraph{Cos'è un protocollo} insieme di regole che dice come comunicare ed esporre verso l'esterno.
\paragraph{Es. modello stratificato: sistema postale} vedi slide, da livello alto a livello basso per spedizione, viceversa per ricevere (da basso ad alto). Il problema grosso di mandare lettera in ita a jap in una serie di passi, in cui viene eseguito un particolare compito su un messaggio, che viene trasferito ad un altro livello. Segretaria prepara lettera affinché postino la possa prendere. Però il messaggio scritto da un livello è pensato per essere interpretato dallo stesso livello del sistema di arrivo (segretaria scrive per segretaria, direttore scrive per direttore).
\subsection{Incapsulamento} aggiungo strati, involucri al messaggio originale che vengono man mano tolti alla destinazione.
\subsection{Perché stratificare} prendo sistema che per una singola coppia di aziende è costoso, lo trasformo a strati così che il costo della singola lettera sia irrisorio.
Definisco funzioni di base per effettuare trasferimento e agenti che le svolgono.
Principi di base: separation of concern e information hiding
\subsubsection{Separation of Concern}
Fare ciò che compete delegando agli altri ciò che è delegabile.
\subsubsection{Information Hiding}
Nascondere info non indispensabili affinché il committente possa svolgere l'operazione.

se traduco modlelo postale in a strati ho\\
utente\\segretaria\\postino\\smistamento\\stazione\\
smistamento intermedio: arrivo fino ad un livello intermedio per evitare che si possano esporre info sensibili.\\\\
vantaggi a strati: svilupp e impl singolo strato più semplice rispetto a fare tutto il complesso\\
servizi di strati inferiori usati da più entità di strati superiori\\\\
\subsection{OSI RM (Open Systems Interconnction Reference Model)}
Prime reti chiuse, tecnologie e protocolli proprietari. Es. ARPANET, SNA (IBM), Dna (Digital), non intercomunicavano. Le reti erano per servizi specifici (TELCO)\\
Quindi obiettivo: modello riferimento per sistemi aprti (quasliasi terminale deve poter comunicarer mediante qualsiasi rete). $\rightarrow$ accordarsi sulle regole\\\\
OSI era un set di protocolli aperti: dettagli pubblici e cambiamenti gestiti da organizzazione con partecipazione aperta al pubblico. Un sistema che implementa protocolli aperti è un sistema aperto. L'ISO.\\
Elementi fondamentali modello stratificiato:\\
flusso dati\\servizi\\protocolli\\interfacce\\
uno strato fornisce servizi a strato sopra e richiede servizi a strato sotto. Strato n comunica con strato n di altra unità tramite protocollo assegnato. comunicaz strati attraverso interfaccia.\\

\subsection{Protocolli}
\paragraph{cos'è un protocollo (x2)} protocolli definiscono formatop e ordine messaggi inviati e ricevute e azioni per trasmette e ricevere messaggi.
\subparagraph{efficace} sistema raggiunge socpo con maggior freq possibile
\subparagraph{efficiente} sistema raggiunge scopo con minor sforzo possibile

\subsubsection{Pila di protocolli}
OSI prevede sette strati\\
7-5 più modi di raggrupparli, livello software e applicativo\\
7 applicativo: elaborazione dati\\
6 presentazione: unificazione dati, preparazione del pacchetto\\
5 sessione: controllo del dialogo tra due host sorgente e dest. Decidere regole per la comunicazione tra due host.\\
4 trasporto: offre trasferimento dati tra host terminali. Astrazione logica per cui si consegna a questo livello un messaggio e dove mandarlo. \textbf{dialogo end-to-end}\\
3 rete: instradamento del traffico (router, offre il servizio di consegna attraverso sistema distribuito di nodi intermedi)\\
2 datalink: frame fra interfacce, scheda di rete deve interpretare il flusso di bit\\
1 fisico: mezzo fisico, modulare segnale elettrico per trasmettere il flusso di bit\\
4-1 livelli fisico+sw\\\\
servizio: funzione che uno strato offre a strato superiore, attraverso un'interfaccia\\
interfaccia: regole che governano formato e significato frame, pacchetti o messaggi che vengono scmabiati tra strati successivi della stessa entità\\
servizi cosa, interfaccia come\\\\
connection-oriented: livello trasferimento crea connessione logica tra due sistemi (instaurazione connessione, trasferimento dati, chiusura connesione)\\
connectionless: dati trasmessi senza stabilire connessione\\

\section{Flusso dell'Informazione}
Qualcuno (livello applicativo) genera dati da mandare in remoto. L'info scende i livelli fino al canale fisico, e ogni livello aggiungeall'informazione ricevuta dal livello superiore una propria (o più) sezione informativa sotto forma di header con info esclusive di quel livello.\\
HEADER|DATA|TRAILER\\
header: qualificazione del pacchetto per questo livello\\
data: payload dal liv superiore\\
trailer: coda\\
slide esempio incapsulamento dei dati\\

\section{Stack protocollare TCP/IP}
famiglia protocollo attualmente usati in interneti. gerarchia di protocollli costituita da moduli interagenti ciascuno ocn funz specifiche.\\
gerarchia: ciascun protocollo liv superiore è supportato da servizi di livelli inferiore\\\\
applicaizone: supporta applicazione rete: smtp, ftp, http\\
trasporto: trasferimento dati host-host: tcp, udp\\
rete: instradamento datagrammi dalla sorgente alla destinazione e protocolli di management rete: ip, icmp\\
link: trasferimento dati tra elementi di rete vicini (da un hop all'altro): ppp, ethernet,... qualunque cosa\\
fisico: aggiunto successivamente: bits on the wire\\\\
come protocollo la specifica fa comuniazione a livello orizzontale (da livello n origine a livello n destinazione).\\
\lstset{language=Java}
\begin{lstlisting}
Ciao ciao = new Ciao();
\end{lstlisting}
\end{document} 