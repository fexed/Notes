\documentclass[10pt]{article}
\usepackage[utf8]{inputenc}
\usepackage[italian]{babel}
\usepackage{multicol}
\usepackage[a4paper, total={18cm, 25cm}]{geometry}
\usepackage{listings}
\usepackage{graphicx}
\graphicspath{ {./img/} }
\usepackage{color}

\begin{document}
\title{Quick Reference per Reti di Calcolatori e Laboratorio}
\author{Federico Matteoni}
\date{V 1.0 -- Primo Compitino}
\renewcommand*\contentsname{Indice}
\definecolor{pblue}{rgb}{0.13,0.13,1}
\definecolor{pgreen}{rgb}{0,0.5,0}
\definecolor{pred}{rgb}{0.9,0,0}
\definecolor{pgrey}{rgb}{0.46,0.45,0.48}
\lstset{
  language=Java,
  showspaces=false,
  showtabs=false,
  breaklines=true,
  showstringspaces=false,
  breakatwhitespace=true,
  commentstyle=\color{pgreen},
  keywordstyle=\color{pblue},
  stringstyle=\color{pred},
  basicstyle=\small\ttfamily
}

\maketitle

\section{TCP}
\subsubsection{Controllo congestione}
\paragraph{cWnd} parte a \textbf{1 MSS}
\paragraph{RENO} \begin{list}{-}{}
\item \texttt{cWnd $<$ soglia} -- crescita \textbf{esponenziale} (\textit{slow start}): + 1 MSS ad ogni ACK
\item \texttt{cWnd $>$ soglia} -- crescita \textbf{lineare} (\textit{AI})
\item \textbf{Perdita: 3 ACK duplicati} soglia = cWnd/2, cWnd = soglia + 3 MSS (\textit{fast recovery})
\item \textbf{Timeout}: soglia = cWnd/2, cWnd = 1 MSS (\textit{slow start})
\end{list}
\end{document}