\documentclass[10pt]{article}
\usepackage[utf8]{inputenc}
\usepackage[italian]{babel}
\usepackage{multicol}
\usepackage[a4paper, total={18cm, 25cm}]{geometry}
\usepackage{listings}
\usepackage{graphicx}
\usepackage{makecell}
\graphicspath{ {./img/} }
\usepackage{color}

\begin{document}
\title{Laboratorio di Reti}
\author{Federico Matteoni}
\date{ }
\renewcommand*\contentsname{Indice}

\maketitle
\section{Thread}
\paragraph{Processo} Istanza di un programma in esecuzione
\paragraph{Thread} \textbf{Flusso di esecuzione} all'interno di un processo $\Rightarrow$ Ogni processo ha almeno un thread.\\I \textbf{thread condividono le risorse di un processo}.\\
Possono essere eseguiti sia su single-core (es. interleaving, time-sharing\ldots) che su multicore (più flussi di esecuzione in parallelo)
\paragraph{Multitasking} \begin{list}{}{Si può riferire a}
	\item Processi, controllato esclusivamente dal S.O.
	\item Thread, controllato in parte dal programmatore
\end{list}
\paragraph{Contesto di un processo} Insieme delle informazioni necessarie per ristabilire esattamente lo stato in cui si trova il S.O. nel momento in cui si interrompe l'esecuzione di un processo per passare ad un altro: registri del processore, memoria del processo\ldots
\paragraph{Perché?} Per gestire più funzionalità contemporaneamente, come gestire input, visualizzare a schermo, monitorare la rete ed eseguire calcoli.\\
Esempi noti: browser web, videogame multiplayer. Si creano \textbf{più componenti interagenti} in modo da:\begin{list}{}{}
	\item Usare meglio le risorse
	\item Migliorare le performance per applicazioni che richiedono grossi calcoli: si dividono i task per eseguirli in parallelo.
	\item Anche problemi: difficile debugging e manutenzione, sincronizzazione, deadlocks\ldots
\end{list}
\paragraph{In Java} Il main thread, invocato dalla JVM all'esecuzione del programma, può attivare altri thread. La JVM attiva automaticamente altri thread come il garbage collector.\\
Un thread è un oggetto. Per creare un thread si definisce un task che implementi l'interfaccia \texttt{Runnable} e si crea un thread passandogli l'istanza del task creato. Altrimenti si può estendere la classe \texttt{java.lang.Thread}.
\paragraph{Runnable} Appartiene a \texttt{java.language}, contiene solo la firma del metodo \texttt{void run()}. Un oggetto che la implementa è un frammento di codice che può essere eseguito in un thread.
\paragraph{Stati}
\begin{list}{}{}
	\item \textbf{Created}/\textbf{New}: subito dopo l'istruzione \texttt{new}, variabili allocate ed inizializzate. Thread in attesa di passare in esecuzione
	\item \textbf{Runnable}/\textbf{Running}: thread in esecuzione o in attesa per ottenere la CPU (Java non separa i due stati).
	\item \textbf{Not Runnable} (\textbf{Blocked}/\textbf{Waiting}): thread non può essere messo in esecuzione, può accadere quando attende un'operazione I/O o ha invocato metodi come \texttt{sleep()} oppure \texttt{wait()}.
	\item \textbf{Dead}: termine naturale o dopo l'invocazione di \texttt{stop()} da parte di altri thread (deprecato).
\end{list}
\subsection{Threadpool}
\paragraph{Perché?} In caso di task leggeri molto frequenti risulta impraticabile attivare ulteriori thread. Diventa quindi utile definire un limite massimo di thread che possono essere attivati contemporaneamente, così da sfruttare meglio i processori, evitare troppi thread in competizione e diminuire i costi di attivazione/terminazione dei thread.
\paragraph{Threadpool} Struttura dati la cui dimensione massima può essere prefissata, contenente riferimenti ad un insieme di thread. I thread possono essere riutilizzati: la sottomissione di un task al threadpool è \textbf{disaccoppiata} dall'esecuzione del thread. L'esecuzione può essere ritardata se non vi sono risorse disponibili.\\
La \textbf{politica di gestione dei thread} stabilisce quando i thread vengono attivati (al momento della creazione del pool, on demand, all'arrivo di un nuovo task\ldots) e quando è opportuno terminare l'esecuzione di un thread.\\
Il threadpool, quindi, al momento della sottomissione di un task può:
\begin{list}{}{}
	\item Usare un thread attivato in precedenza e al momento inattivo
	\item Creare un nuovo thread
	\item Memorizzare il task in una coda, in attesa
	\item Respingere la richiesta
\end{list}
\paragraph{Callable} Classe per definire un task che può restituire un risultato e sollevare eccezioni
\paragraph{Future} Rappresenta il risultato di una computazione asincrona. Definisce metodi per controllare se la computazione è terminata, attendere la terminazione oppure cancellarla. Viene implementata nella classe \texttt{FutureTask}.
\end{document}