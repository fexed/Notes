\documentclass[10pt]{report}
\usepackage[utf8]{inputenc}
\usepackage[italian]{babel}
\usepackage{multicol}
\usepackage[bookmarks]{hyperref}
\usepackage[a4paper, total={18cm, 25cm}]{geometry}
\usepackage{graphicx}
\usepackage{xcolor}
\usepackage{textcomp}
\graphicspath{ {./img/} }
\usepackage{listings}
\usepackage{makecell}
\usepackage{qtree}
\usepackage{pgfplots}
\usepackage{tikz}
\usepgflibrary{shapes}
\usepgfplotslibrary{fillbetween}
\definecolor{backcolour}{RGB}{255,255,255}
\definecolor{codegreen}{RGB}{27,168,11}
\definecolor{codeblue}{RGB}{35,35,205}
\definecolor{codegray}{RGB}{128,128,128}
\definecolor{codepurple}{RGB}{205,35,56}
\lstdefinestyle{myPython}{
	backgroundcolor=\color{backcolour},   
	commentstyle=\color{codegreen},
	keywordstyle=\color{codeblue},
	numberstyle=\tiny\color{codegray},
	stringstyle=\color{codepurple},
	basicstyle=\small\ttfamily,
	breakatwhitespace=false,         
	breaklines=true,                 
	captionpos=b,                    
	keepspaces=true,                 
	numbers=left,                    
	numbersep=2pt,                  
	showspaces=false,                
	showstringspaces=false,
	showtabs=false,                  
	tabsize=2,
	language=python
}
\newcommand*\triangled[1]{\tikz[baseline=(char.base)]{
            \node[regular polygon, regular polygon sides=3,draw,inner sep=1pt] (char) {#1};}}
            
\usepackage{fancyhdr}
\pagestyle{fancy}
\renewcommand{\headrulewidth}{0pt}
\fancyhead{}
\fancyfoot[L]{Telegram: \texttt{@fexed}}
\fancyfoot[R]{Github: \texttt{fexed}}
\begin{document}
\title{Intelligent Systems for Pattern Recognition}
\author{Federico Matteoni}
\date{A.A. 2021/22}
\renewcommand*\contentsname{Index}

\maketitle
\tableofcontents
\pagebreak
\section{Introduction}
Prof.s: Davide Bacciu and Antonio Carta
\paragraph{Objectives} Train ML specialists capable of: designing novel learning models, developing pattern recognition applications using ML, developing intelligent agents using \textbf{Reinforcement Learning}.\\
We're referring to images and signals, but not limited to that: practical applications.\\
Focusing on challenging and complex data: \textbf{machine vision} (noisy, hard to interpret, semantically rich\ldots) and \textbf{structured data} (relational information: sequences, trees, graphs\ldots)\\
Natural Language Processing will be used as an example, but will not be the focus of this course.
\paragraph{Methodology-Oriented Outcomes} Gain in-depth knowledge of advanced machine learning models, understanding the underlying theory. This gives the ability to read and understand and discuss research works in the field.
\paragraph{Application-Oriented Outcomes} Learn to address modern pattern recognition applications, gain knowledge of ML, PR and RL libraries and be able to develop an application using ML and RL models.
\paragraph{Prerequisites} Knowledge of ML fundamentals, mathematical tools for ML and Python.
\section{Pattern Recognition}
Automated recognition of meaningful patterns in noisy data.
\paragraph{Origins} 
\paragraph{Viola-Jones Algorithm} Framework for face recognition. Sum pixel in white area and subtract those in the black portion. The VJ algorithm positions the masks on the image and combines the responses (training set of $\simeq$5k images with hand-aligned filters)
\paragraph{An historical view} \begin{enumerate}
	\item Identification of distinguishing features of the object/entity (\textbf{feature detection})
	\item Extraction of features for the defining attributes (\textbf{feature extraction})
	\item Comparison with known patterns (\textbf{matching})
\end{enumerate}
Basically, lots of time spent hand-engineering the best data features.
\paragraph{A modern view}
Data is thrown into a neural network. A single stage process with a data crushing-and-munching neural network spitting out prediction, which encapsulates the three historical steps. But the time is now spent in fine-tuning the neural network.
\paragraph{The deep learning Lego} Creating applications by putting together various combinations of CNN and LSTM modules.
\subsection{Signals}
Signals are time series: a sequence of measurements in time. Examples of sources are: medicine, finance, geology, IoT, biometrics\ldots
\paragraph{Formalization} A time series x is a sequence of measurements in time $t$
$$\hbox{x} = x_0,\ldots,x_N$$
where $x_t$ or $x(t)$ is the measurement at time $t$.
\begin{list}{}{}
	\item Observation can be at \textbf{irregular} time intervals.
	\item We assume \textbf{weakly stationary} (or second-order stationary) data\begin{list}{}{}
		\item $\forall\:t\:\:E[x_t] = \mu$
		\item $\forall\:t$ Cov$(x_{t+\tau},x_t) = \gamma_\tau$ with $\gamma$ depending only on the lag $\tau$
	\end{list}
\end{list}
\paragraph{Goals}\begin{list}{}{}
	\item \textbf{Description}
	\item \textbf{Analysis}: identify and describe dependencies in data
	\item \textbf{Prediction}: forecast next values given information up to $t$
	\item \textbf{Control}: adjust parameters of the generative process to make the time series fit a target
\end{list}
\paragraph{Key Methods}\begin{list}{}{}
	\item \textbf{Time domain analysis}: assesses how a signal changes over time (correlation, convolution, auto-regressive models)
	\item \textbf{Spectral domain analysis}: assesses the distribution of the signal over a range of frequencies (Fourier analysis, wavelets)
\end{list}
\subsection{Time Domain Analysis}
\paragraph{Mean} $$\hat\mu=\frac{1}{N}\sum_{t=1}^N x_t$$
Can be used to subtract mean from values and "standardize" the two series.
\paragraph{Autocovariance} For lag $-N\leq \tau \leq N$
$$\hat\gamma_x(\tau) = \frac{1}{N}\sum_{t=1}^{N-|\tau|} (x_{t+|\tau|}-\hat\mu)(x_t - \hat\mu)$$
\paragraph{Autocorrelation} The correlation of a signal with itself. $$\hat\rho_x(\tau)=\frac{\hat\gamma_x(\tau)}{\hat\gamma_x(0)}$$
We can compute this with every possible $\tau$, finding the max/min which gives the $\tau$ where the autocorrelation is max/min, which means the lag where the signal starts repeating itself. The lags near zero typically dominates, so we want the maximum lag reasonably far from 0.
\subparagraph{Autocorrelation plot} %TODO
\paragraph{Cross-Correlation} A measure of similarity of x$_1$ and x$_2$ as a function of a time lag $\tau$ $$\phi_{\hbox{x}_1\:\hbox{x}_2}(\tau)=\sum_{t = \max\{0,\tau\}}^{\min\{(T_1 - 1 + \tau), (T_2 - 1)\}} \hbox{x}_1(t-\tau)\cdot \hbox{x}_2(t)$$
\subparagraph{Normalized cross-correlation} Returns an amplitude independent value %TODO
$$\overline{\phi}_{\hbox{x}_1\:\hbox{x}:2}(\tau) = \frac{\phi_{\hbox{x}_1\:\hbox{x}:2}}{\hbox{\texttt{todo}}} \in [-1,+1]$$
With $\overline{\phi}_{\hbox{x}_1\:\hbox{x}:2}(\tau) = +1$ mean that the two time series have the exact same shape if aligned at time $\tau$. Nearing $-1$ we get the maximum anticorrelation, same shape but opposite sign. Near 0 we get that the two signals are completely \textbf{linearly} uncorrelated.\\
Note that we measure \textbf{linear correlation}.\\\\
Cross correlation looks like the convolution $$(f * g)[n]=\sum_{t=-M}^M f(n-t)g(t)$$ but we have a flipped sign ($n-t$ instead of $t-\tau$).\\
Cross-correlation is not symmetric, whereas convolution is ($f * g = g * f$).
\paragraph{Autoregressive Process} A timeseries autoregressive process (AR) of order $K$ is the linear system $$x_t = \sum_{k=1}^K \alpha_k x_{t-k} + \epsilon_t$$\begin{list}{}{}
	\item Autoregressive means $x_t$ regresses on itself
	\item $\alpha_k \Rightarrow$ linear coefficients$\:|\:|\alpha|<1$
	\item $\epsilon_t\Rightarrow$ sequence of independent and identically distributed values with mean 0 and fixed variance.
	\item We look backward $K$ steps, so limited memory.
\end{list}
\paragraph{ARMA} Autoregressive with Moving Average process $$x_t = \sum_{k=1}^K \alpha_k x_{t-k} + \sum_{q=1}^Q \beta_q\epsilon_{t-1}+\epsilon_t$$
\begin{list}{}{}
	\item With $\epsilon_t$ Random white noise (again)
	\item The current time series values is the result of a regression on its past values plus a term that depends on a combination of stochastically uncorrelated information
\end{list}
\paragraph{Estimating Autoregressive Models} Need to estimate: the values of the linear coefficients $\alpha_t$ and $\beta_t$ and the order of the autoregressor $K$ and $Q$\\
Estimation of the $\alpha$, $\beta$ is performed with the Levinson-Durbin Recursion (\texttt{levinson(x, K)} in matlab, and included in several Python modules).\\
The order is often estimated with a Bayesian model selection criterion, choosing the largest $K$ and $Q$ possible. E.g.: BIC, AIC\ldots\\
The set of autoregressive parameters $\alpha_{i,1},\ldots,\alpha_{i,K}$ fitted to a specific time series x$_i$ is used to confront it with other time series. Same thing for $\beta$ so we can use $\alpha$ for both sets.
\paragraph{Comparing time series by AR}\begin{list}{}{}
	\item timeseries clustering %TODO
	\item novelty/anomaly detection %TODO
\end{list}
\subsection{Spectral Domain Analysis}
Analyze the time series in the frequency domain. Key idea: decomposing the time series into a linear combination of sines and cosines with random and uncorrelated coefficients. So a \textbf{regression on sinusoids} with Fourier analysis.
\paragraph{Fourier Transform} Discrete Fourier Transform (DFT): transform a time series from the time domain to the frequency domain. Can be easily inverted back to the time domain.\\
Useful to handle periodicity in the time series: seasonal trends, cyclic processes\ldots
\paragraph{Representing functions} We know that, given an orthonormal system for $E$ we can use linear combinations of the basis $\{e_1,\ldots\}$ to represent any function $f\in E$ $$\sum_{k=1}^\infty \langle f,e_k\rangle e_k$$
Given the orthonormal system $$\{\frac{1}{\sqrt{2}}, \sin(x),\cos(x),\sin(2x),\cos(2x),\ldots\}$$
The linear combination above becomes the Fourier series \texttt{todo} %TODO
\paragraph{Representing function in Complex space} Using $\cos(kx)-i\:\sin(kx) = e^{-ikx}$ with $i=\sqrt{-1}$ we can rewrite the Fourier series as $$\sum_{k=-\infty}^\infty c_k e^{ikx}$$
on the orthonormal system $$\{1,e^{ix},e^{-ix},e^{2ix},e^{-2ix},\ldots\}$$
\paragraph{Representing Discrete Time Series} Consider x of length $N$ and $x_n\in R$. Using the exponential formulation, the orthonormal system is finite, from $e_0$ to $e_{N-1}$ each $\in C^N$\\
The $n$-th component of the $k$-th vector is $$[e_k]n=e^{\frac{-2\pi ink}{2}}$$
\paragraph{Discrete Fourier Transform} Given a time series x = $x_0,\ldots,x_{N-1}$ its DFT is the sequence $$\hbox{Spectral domain}\:\:\:X_k = \sum_{n=1}^{N-1}x_ne^{\frac{-2\pi ink}{N}}\:\:\:\hbox{Time domain}$$
And can be inverted $$x_k = \frac{1}{N}\sum_{k=1}^{N-1}X_ke^{\frac{2\pi ink}{N}}$$
\paragraph{Basic Spectral Quantities in SFT}
\begin{list}{}{}
	\item Amplitude $A_k = |X_k| = \sqrt{Re^2(X_k)+Im^2(X_k)}$
	\item Power $P_k = \frac{|X_k|^2}{N}$, more used in reality and under some conditions this is a reasonable estimate of %TODO
\end{list}
\paragraph{DFT in Action} We use the DFT elements $X_1,\ldots,X_K$ as representation of the signal to train the predictor/classifier.\\
This representation can reveal patterns that are not clear in the time domain.
\end{document}