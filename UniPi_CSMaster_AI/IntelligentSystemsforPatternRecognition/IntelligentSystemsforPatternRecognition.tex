\documentclass[10pt]{report}
\usepackage[utf8]{inputenc}
\usepackage[italian]{babel}
\usepackage{multicol}
\usepackage[bookmarks]{hyperref}
\usepackage[a4paper, total={18cm, 25cm}]{geometry}
\usepackage{graphicx}
\usepackage{xcolor}
\usepackage{textcomp}
\graphicspath{ {./img/} }
\usepackage{listings}
\usepackage{makecell}
\usepackage{qtree}
\usepackage{pgfplots}
\usepackage{tikz}
\usepgflibrary{shapes}
\usepgfplotslibrary{fillbetween}
\definecolor{backcolour}{RGB}{255,255,255}
\definecolor{codegreen}{RGB}{27,168,11}
\definecolor{codeblue}{RGB}{35,35,205}
\definecolor{codegray}{RGB}{128,128,128}
\definecolor{codepurple}{RGB}{205,35,56}
\lstdefinestyle{myPython}{
	backgroundcolor=\color{backcolour},   
	commentstyle=\color{codegreen},
	keywordstyle=\color{codeblue},
	numberstyle=\tiny\color{codegray},
	stringstyle=\color{codepurple},
	basicstyle=\small\ttfamily,
	breakatwhitespace=false,         
	breaklines=true,                 
	captionpos=b,                    
	keepspaces=true,                 
	numbers=left,                    
	numbersep=2pt,                  
	showspaces=false,                
	showstringspaces=false,
	showtabs=false,                  
	tabsize=2,
	language=python
}
\newcommand*\triangled[1]{\tikz[baseline=(char.base)]{
            \node[regular polygon, regular polygon sides=3,draw,inner sep=1pt] (char) {#1};}}
            
\usepackage{fancyhdr}
\pagestyle{fancy}
\renewcommand{\headrulewidth}{0pt}
\fancyhead{}
\fancyfoot[L]{Telegram: \texttt{@fexed}}
\fancyfoot[R]{Github: \texttt{fexed}}
\begin{document}
\title{Intelligent Systems for Pattern Recognition}
\author{Federico Matteoni}
\date{A.A. 2021/22}
\renewcommand*\contentsname{Index}

\maketitle
\tableofcontents
\pagebreak
\section{Introduction}
Prof.s: Davide Bacciu and Antonio Carta
\paragraph{Objectives} Train ML specialists capable of: designing novel learning models, developing pattern recognition applications using ML, developing intelligent agents using \textbf{Reinforcement Learning}.\\
We're referring to images and signals, but not limited to that: practical applications.\\
Focusing on challenging and complex data: \textbf{machine vision} (noisy, hard to interpret, semantically rich\ldots) and \textbf{structured data} (relational information: sequences, trees, graphs\ldots)\\
Natural Language Processing will be used as an example, but will not be the focus of this course.
\paragraph{Methodology-Oriented Outcomes} Gain in-depth knowledge of advanced machine learning models, understanding the underlying theory. This gives the ability to read and understand and discuss research works in the field.
\paragraph{Application-Oriented Outcomes} Learn to address modern pattern recognition applications, gain knowledge of ML, PR and RL libraries and be able to develop an application using ML and RL models.
\paragraph{Prerequisites} Knowledge of ML fundamentals, mathematical tools for ML and Python.
\section{Pattern Recognition}
Automated recognition of meaningful patterns in noisy data.
\paragraph{Origins} 
\paragraph{Viola-Jones Algorithm} Framework for face recognition. Sum pixel in white area and subtract those in the black portion. The VJ algorithm positions the masks on the image and combines the responses (training set of $\simeq$5k images with hand-aligned filters)
\paragraph{An historical view} \begin{enumerate}
	\item Identification of distinguishing features of the object/entity (\textbf{feature detection})
	\item Extraction of features for the defining attributes (\textbf{feature extraction})
	\item Comparison with known patterns (\textbf{matching})
\end{enumerate}
Basically, lots of time spent hand-engineering the best data features.
\paragraph{A modern view}
Data is thrown into a neural network. A single stage process with a data crushing-and-munching neural network spitting out prediction, which encapsulates the three historical steps. But the time is now spent in fine-tuning the neural network.
\paragraph{The deep learning Lego} Creating applications by putting together various combinations of CNN and LSTM modules.
\end{document}