\documentclass[10pt]{report}
\usepackage[utf8]{inputenc}
\usepackage[italian]{babel}
\usepackage{multicol}
\usepackage[bookmarks]{hyperref}
\usepackage[a4paper, total={18cm, 25cm}]{geometry}
\usepackage{graphicx}
\usepackage{xcolor}
\usepackage{textcomp}
\graphicspath{ {./img/} }
\usepackage{listings}
\usepackage{makecell}
\usepackage{qtree}
\usepackage{pgfplots}
\usepackage{tikz}
\usepgflibrary{shapes}
\usepgfplotslibrary{fillbetween}
\definecolor{backcolour}{RGB}{255,255,255}
\definecolor{codegreen}{RGB}{27,168,11}
\definecolor{codeblue}{RGB}{35,35,205}
\definecolor{codegray}{RGB}{128,128,128}
\definecolor{codepurple}{RGB}{205,35,56}
\lstdefinestyle{myPython}{
	backgroundcolor=\color{backcolour},   
	commentstyle=\color{codegreen},
	keywordstyle=\color{codeblue},
	numberstyle=\tiny\color{codegray},
	stringstyle=\color{codepurple},
	basicstyle=\small\ttfamily,
	breakatwhitespace=false,         
	breaklines=true,                 
	captionpos=b,                    
	keepspaces=true,                 
	numbers=left,                    
	numbersep=2pt,                  
	showspaces=false,                
	showstringspaces=false,
	showtabs=false,                  
	tabsize=2,
	language=python
}
\newcommand*\triangled[1]{\tikz[baseline=(char.base)]{
            \node[regular polygon, regular polygon sides=3,draw,inner sep=1pt] (char) {#1};}}
            
\usepackage{fancyhdr}
\pagestyle{fancy}
\renewcommand{\headrulewidth}{0pt}
\fancyhead{}
\fancyfoot[L]{Telegram: \texttt{@fexed}}
\fancyfoot[R]{Github: \texttt{fexed}}
\begin{document}
\title{Introduction to Quantum Computing}
\author{Federico Matteoni}
\date{A.A. 2021/22}
\renewcommand*\contentsname{Index}

\maketitle
\tableofcontents
\pagebreak
\section{Introduction}
Prof.: Anna Bernasconi, Gianna del Corso
\paragraph{What is Quantum Computing?} Quantum computing concerns the \textbf{relationship between physics and computer science}. Physical phenomenon apply to information and computation: a \textbf{computational process is seen as a physical process}, performed on a machine whose operation obey certain physical laws.\\
The classical theory of computation is based on the Universal Turing Machine, a mathematical abstraction and \textbf{not a physical device}, that works according to a set of rules and principles enunciated in 1936 and elaborated in the 1940s.
\subparagraph{Church-Turing Thesis} \textit{Every function which would naturally be regarded as computable can be computed by the Universal Turing Machine}.\\
A stronger version: every function that we can compute efficiently on any machine efficiently on a Universal Turing Machine. So we can solve a problem if and only if we can solve it on a Turing machine.\\\\
The assumption underlying these principles is that a Turing machine idealizes a mechanical computing device that obeys the laws of classical physics, but nature is better described by the laws of quantum physics. Feynman stated that "\textit{nature isn't classical}", and that our model of computations (i.e. classical computers) cannot efficiently model quantum systems in a scalable manner. They seem to be extraordinarily slow and inefficient at doing quantum simulations.
\paragraph{David Deutsch} "\textit{Computers are physical objects, and computations are physical processes. What computers can or cannot compute is determined by the laws of physics alone, and not pure mathematics.}"\\
Is there a single universal computing device which can efficiently simulate any other physical system? To answer this, Deutsch proposed a new type of computing system: quantum computers.\\
Quantum computers can do everything that conventional computers can do, but are also capable of efficiently simulate quantum-mechanical processes. And so they are \textbf{more natural computing models than conventional computers}.
\paragraph{What is quantum?} Quantum physics is a mathematical model first used to describe physical phenomena that occur at the microscopic level, such as inside an atom, which exposed gaps in the preceding theory of classical physics. Quantum theory explains this behavior and gives us a more complete picture of the universe. The description of the universe given by quantum physics differs in fundamental ways from the classical description, and is often at odds with our intuition which has evolved according to observation of macroscopic phenomena (which are, to an extremely good approximation, classical physics).
\subparagraph{An experiment} Let us consider an experiment that could not be explained in a natural way using classical physics. This experiment involves photons:\begin{list}{}{}
	\item elementary particles (\textbf{quantum}) of light
	\item massless
	\item move at the speed of light in vacuum ($\simeq 3\cdot10^8$ m/s)
	\item exhibit wave-particle duality: behavior featuring properties of both waves and particles
\end{list}
We need a photon source, a beam splitter (implemented using a half silvered mirror) and a pair of photon detectors. We will trace the behavior of the photons.
\begin{center}
	\includegraphics[scale=0.5]{1.png}
\end{center}
We send a series of individual photons along a path from the source towards the splitter. We expect two behaviors: the beam splitter transmits or reflects the photon. We observe the photon arriving at the detector on the right of the splitter half of the time, and arriving at the detector above half of the time.\\
So, we can model the splitter as flipping a fair coin, and choosing whether to transmit or reflect the photon based on the result of the coin-flip.\\\\
A beam splitter behaves like a fair coin: head (state 0) $\rightarrow$ transmitted, tail (state 1) $\rightarrow$ reflected. So both detectors will expect a photon with probability $\frac{1}{2}$
\subparagraph{Second experiment} We extend the experiment with two mirrors and another beam splitter.
\begin{center}
	\includegraphics[scale=0.5]{2.png}
\end{center}
We have three detectors, and we observe a photon in A with probability $\frac{1}{2}$, and in B1 or B2 with probability both $\frac{1}{4}$.\\
Both experiments confirms our prediction.
\subparagraph{Third experiment} Let's remove the detector A.\begin{center}
	\includegraphics[scale=0.5]{3.png}
\end{center}
So we flip our photon, the "quantum coin", \textbf{without looking at the result of the first splitter}. What are the probabilities of observing the photon in B1 or B2? With the classical intuition, we expect $\frac{1}{2}$ probability in both and that's what would happen with a real "macroscopic" coin. So we predict to observe the photon in B1 and B2 evenly.\\
Let's see it in a mathematical way:
\begin{list}{}{}
	\item State 0: transmitted
	\item State 1: reflected
\end{list}
With a vector representation:
$$|0\rangle = \left(\begin{array}{c}
	1\\0
	\end{array}\right)\:\:\:\:\:\:\:\:|1\rangle = \left(\begin{array}{c}
	0\\1
	\end{array}\right)$$
With $|\:\:\rangle$ called \textbf{Dirac notation}, standard notation for states in quantum mechanics.\\
Uncertain states will be represented by linear combinations of $|0\rangle$ and $|1\rangle$
$$\alpha_0|0\rangle + \alpha_1|1\rangle = \alpha_0\left(\begin{array}{c}
1\\0
\end{array}\right) + \alpha_1\left(\begin{array}{c}
0\\1
\end{array}\right) = \left(\begin{array}{c}
\alpha_0\\\alpha_1
\end{array}\right)$$
With $\alpha_0,\alpha_1$ being the probabilities of finding the photon in state $|0\rangle$ or $|1\rangle$.\\
Since we should find the photon in exactly one path, we must have $\alpha_0+\alpha_1=1$\\\\
We model the splitter as randomly selecting whether the photon will be transmitted (state $|0\rangle$) or reflected (state $|1\rangle$)\\
After the initial step, we are in $|0\rangle$. We flip a coin (first splitter): the new probabilistic state is expected to be in both states with probability $\frac{1}{2}$
$$\frac{1}{2}|0\rangle + \frac{1}{2}|1\rangle = \left(\begin{array}{c}
\frac{1}{2}\\\frac{1}{2}
\end{array}\right)$$
The transition of a far coin can be represented by the matrix $$\left(\begin{array}{c c}
\frac{1}{2}&\frac{1}{2}\\\frac{1}{2}&\frac{1}{2}
\end{array}\right)$$
When the photon passes through the splitter, we multiply its state vector by this matrix, to derive the new state where the photon is expected to be in both states $|0\rangle$ and $|1\rangle$ with probability $\frac{1}{2}$
$$\left(\begin{array}{c c}
\frac{1}{2}&\frac{1}{2}\\\frac{1}{2}&\frac{1}{2}
\end{array}\right)\left(\begin{array}{c}
1\\0
\end{array}\right) = \left(\begin{array}{c}
\frac{1}{2}\\\frac{1}{2}
\end{array}\right) = \frac{1}{2}\left(\begin{array}{c}
1\\0
\end{array}\right) + \frac{1}{2}\left(\begin{array}{c}
0\\1
\end{array}\right)$$
Then we flip the coin again, and multiply the new state vector by the same matrix. The new probabilistic state will be the same:
$$\left(\begin{array}{c c}
\frac{1}{2}&\frac{1}{2}\\\frac{1}{2}&\frac{1}{2}
\end{array}\right)\left(\begin{array}{c}
\frac{1}{2}\\\frac{1}{2}
\end{array}\right) = \left(\begin{array}{c}
\frac{1}{2}\\\frac{1}{2}
\end{array}\right) = \frac{1}{2}\left(\begin{array}{c}
1\\0
\end{array}\right) + \frac{1}{2}\left(\begin{array}{c}
0\\1
\end{array}\right)$$
So our mathematical model confirms our expectations.\\\\
\textbf{The experimental results do not agree with our classical intuition!} We observe the photons \textbf{only in B1} and we \textbf{never observe any photon in B2}. This is the same problem which led to the development of quantum physics.
\paragraph{Quantum Physics} So let's use quantum physics to explain our experiments. "\textit{Quantum theory is probability theory with negative numbers}", but we can't have negative probabilities so we will use a new quantity called \textbf{amplitude}. To get around the fact that we cannot have negative probabilities and that all our probabilities must add up to 1, we use a mathematical trick: we square our amplitudes to calculate the probabilities.\\
According to the quantum mechanical description, the beam splitter causes the photon to go into a \textbf{superposition} of both states. Mathematically, we describe such superposition by taking a linear combination of the state vectors with $\alpha_0$ and $\alpha_1$ now being \textbf{complex numbers} $\in C$. If we measure the photon to see its state, we find it in state $|0\rangle$ with probability $|\alpha_0|^2$ and in state $|1\rangle$ with probability $|\alpha_1|^2$, and since a photon should be find in exactly one path, we need $|\alpha_0|^2+|\alpha_1|^2 = 1$\\\\
We start in state $|0\rangle=\left(\begin{array}{c}1\\0\end{array}\right)$. When it passes through the first splitter, we multiply its state vector with the matrix $$\left(\begin{array}{c c}
\frac{1}{\sqrt{2}}&\frac{1}{\sqrt{2}}\\\frac{1}{\sqrt{2}}&-\frac{1}{\sqrt{2}}
\end{array}\right)$$
After passing through the first splitter:
$$\left(\begin{array}{c c}
\frac{1}{\sqrt{2}}&\frac{1}{\sqrt{2}}\\\frac{1}{\sqrt{2}}&-\frac{1}{\sqrt{2}}
\end{array}\right)\left(\begin{array}{c}
1\\0
\end{array}\right) = \left(\begin{array}{c}
\frac{1}{\sqrt{2}}\\\frac{1}{\sqrt{2}}
\end{array}\right) = \frac{1}{\sqrt{2}}\left(\begin{array}{c}
1\\1
\end{array}\right)$$
Same as before, with $\frac{1}{\sqrt{2}}$ instead of $\frac{1}{2}$. The result correspond with the observed behavior: we measure the photon in state $|0\rangle$ with probability $\left(\frac{1}{\sqrt{2}}\right)^2=\frac{1}{2}$ and in state $|1\rangle$ with probability $\left(\frac{1}{\sqrt{2}}\right)^2=\frac{1}{2}$. The photon is in a \textbf{superposition} of states $|0\rangle$ and state $|1\rangle$, being in both states with amplitudes $\frac{1}{\sqrt{2}}$ and $\frac{1}{\sqrt{2}}$ respectively.\\
If we do not measure the state of the photon after passing through the first beam splitter, then its state remains $\frac{1}{\sqrt{2}}\left(\begin{array}{c}
1\\1
\end{array}\right)$. If the photon is allowed to pass through the second splitter before any measurement, the new state vector becomes
$$\left(\begin{array}{c c}
\frac{1}{\sqrt{2}}&\frac{1}{\sqrt{2}}\\\frac{1}{\sqrt{2}}&-\frac{1}{\sqrt{2}}
\end{array}\right)\cdot\frac{1}{\sqrt{2}}\left(\begin{array}{c}
1\\1
\end{array}\right) = \left(\begin{array}{c}
\frac{1}{2} + \frac{1}{2}\\\frac{1}{2}-\frac{1}{2}
\end{array}\right) = \left(\begin{array}{c}
1\\0
\end{array}\right)$$
The amplitude of the state $|0\rangle$ becomes 1, but the amplitude of the state $|1\rangle$ becomes 0 because \textbf{the amplitudes of finding the photon in state $|1\rangle$ cancel each other out}. We call this effect \textbf{interference}.\\
After being in both states at the same time with certain amplitudes, by passing through a second splitter the outcomes are interfered with each other: the interference can be destructive ($\frac{1}{2} - \frac{1}{2}$) or constructive ($\frac{1}{2} + \frac{1}{2}$).
\paragraph{What is Quantum Computing?} If we measure the photon, we find it coming out of state $|0\rangle$ with probability 1. Thus, after the second splitter the photon is entirely in state $|0\rangle$, which is what we observed.\\
In quantum "language": the second splitter has caused the two states (in superposition) to interfere, resulting in the cancellation of state $|1\rangle$.\\
The interference effects can be used to our advantage. We can combine operations such as the quantum coin toss to build more efficient algorithms. These algorithms can use interference effects to make the wrong answers cancel out quickly and give us high probabilities of measuring the right answer. This is the idea behind quantum computing.
\paragraph{Observations}\begin{list}{}{}
	\item This model works when the initial state is $|1\rangle$
	\item This model works also with complex numbers\\
	For instance we could use:
	\begin{list}{}{}
		\item Transition matrix: $\frac{1}{\sqrt{2}}\left(\begin{array}{c c}
		1&i\\i&1
		\end{array}\right)$
		\item Superposition of state $|0\rangle$ and $|1\rangle$: $\frac{1}{\sqrt{2}}\left(\begin{array}{c}
		i\\1
		\end{array}\right)$
		\item The model is consistent with the first and second experiment
		\begin{center}
			\includegraphics[scale=0.5]{4.png}
		\end{center}
	\end{list}
\end{list}
\paragraph{Phenomena of quantum mechanics that may intervene in the processing of information}
\begin{list}{}{}
	\item \textbf{Superposition} Property of a quantum system to be in different states at the same time.\\
	A quantum system can be in more than one state at the same time with non-zero amplitudes: we say that it's in a superposition of these states. When evolving from a superposition, the resulting transitions may affect each other constructively and destructively. This happens because of having opposite sign transition amplitudes
	\item \textbf{Decoherence} The attempt to observe or measure the state of the system causes its collapse towards a single state.\\
	The probability of a system to be observed in a specific state is the square value of its amplitude of a state. After the measurement, the system is no longer in a superposition: the information kept in the superposition is lost.\\
	The experimental manipulation of quantum systems is extremely difficult because every minimal disturbance can determine the decoherence.\\
	Qubits interact with their environments to some degree, even thought the physical substrate used to store them has been designed to keep the isolated.
	\item \textbf{No-Cloning} It's impossible to create an independent and identical copy of an arbitrary unknown quantum state
	\item \textbf{Entanglement} Possibility that two or more elements are in quantum states completely correlated with each other so that, even if transported at a great distance from each other, they maintain the correlation.
\end{list}
\subsection{Quantum Computer}
\paragraph{Bit and qubit} Conventional computers are made up of bits, while quantum computers are made up of quantum bits, or \textbf{qubits}.\\
A bit is the fundamental concept of classical computation: we can think of it in abstract terms as having a state which is either 0 or 1.\\
A qubit is the simplest of all quantum systems:
\begin{list}{}{}
	\item like a bit, it has a state
	\item two special states for qubits are the state $|0\rangle$ and $|1\rangle$, which correspond to states 0 and 1 of classical bits
	
$$|0\rangle = \left(\begin{array}{c}
	1\\0
	\end{array}\right)\:\:\:\:\:\:\:\:|1\rangle = \left(\begin{array}{c}
	0\\1
	\end{array}\right)$$
	These are called \textbf{computational basis states} and form an orthonormal basis for $C^2$
	\item The difference between bits and qubits is that a qubit can be in a state other than $|0\rangle$ and $|1\rangle$: it can be in a superposition of the two states simultaneously
	$$|\psi\rangle = \alpha|0\rangle+\beta|1\rangle$$
	\item The representation of information is binary, but each qubits contains double information with respect to a bit.
	\item We can examine a bit to determine if it's in state 0 or 1, but we cannot examine a qubit to determine it's quantum state (the values of $\alpha$ and $\beta$). We can only acquire much more restricted information about the quantum state.
	\item Measuring a qubit can only give classical results: either 0 with probability $|\alpha|^2$ or 1 with probability $|\beta|^2$\\
	Note that by measuring we lose information.
	\item A qubit $|\psi\rangle$ can be represented in a three-dimensional space as a point on the surface of a sphere of unitary radius known as \textbf{Bloch's sphere}.
\end{list}
\begin{center}
	\includegraphics[scale=0.5]{5.png}
\end{center}
\end{document}