\documentclass[10pt]{report}
\usepackage[utf8]{inputenc}
\usepackage[italian]{babel}
\usepackage{multicol}
\usepackage[bookmarks]{hyperref}
\usepackage[a4paper, total={18cm, 25cm}]{geometry}
\usepackage{graphicx}
\usepackage{xcolor}
\usepackage{textcomp}
\graphicspath{ {./img/} }
\usepackage{listings}
\usepackage{makecell}
\usepackage{qtree}
\usepackage{pgfplots}
\usepackage{tikz}
\usepgflibrary{shapes}
\usepgfplotslibrary{fillbetween}
\definecolor{backcolour}{RGB}{255,255,255}
\definecolor{codegreen}{RGB}{27,168,11}
\definecolor{codeblue}{RGB}{35,35,205}
\definecolor{codegray}{RGB}{128,128,128}
\definecolor{codepurple}{RGB}{205,35,56}
\lstdefinestyle{myPython}{
	backgroundcolor=\color{backcolour},   
	commentstyle=\color{codegreen},
	keywordstyle=\color{codeblue},
	numberstyle=\tiny\color{codegray},
	stringstyle=\color{codepurple},
	basicstyle=\small\ttfamily,
	breakatwhitespace=false,         
	breaklines=true,                 
	captionpos=b,                    
	keepspaces=true,                 
	numbers=left,                    
	numbersep=2pt,                  
	showspaces=false,                
	showstringspaces=false,
	showtabs=false,                  
	tabsize=2,
	language=python
}
\newcommand*\triangled[1]{\tikz[baseline=(char.base)]{
            \node[regular polygon, regular polygon sides=3,draw,inner sep=1pt] (char) {#1};}}
            
\usepackage{fancyhdr}
\pagestyle{fancy}
\renewcommand{\headrulewidth}{0pt}
\fancyhead{}
\fancyfoot[L]{Telegram: \texttt{@fexed}}
\fancyfoot[R]{Github: \texttt{fexed}}
\begin{document}
\title{Human Language Technologies}
\author{Federico Matteoni}
\date{A.A. 2021/22}
\renewcommand*\contentsname{Index}

\maketitle
\tableofcontents
\pagebreak
\section{Introduction}
Prof. Giuseppe Attardi\\
Prerequisites are: proficiency in Python, basic probability and statistics, calculus and linear algebra and notions of machine learning.
\paragraph{What will we learn} Understanding of and ability to use effective modern methods for \textbf{Natural Language Processing}. From traditional methods to current advanced ones like RNN, Attentions\ldots\\
Understanding the difficulties in dealing with NL and the capabilities of current technologies, with experience with \textbf{modern tools} and aiming towards the ability to build systems for some major NLP tasks: word similarities, parsing, machine translation, entity recognition, question answering, sentiment analysis, dialogue system\ldots
\paragraph{Books} Speech and Language Processing (Jurafsky, Martin), Deep Learning (Goodfellow, Bengio, Courville), Natural Langue Processing in Python (Bird, Klein, Loper)
\paragraph{Exam} Project (alone or team of 2-3 people) with the aim to experiment with techniques in a realistic setting using data from competitions (Kaggle, CoNLL, SemEval, Evalita\ldots). The topic will be proposed by the team or chosen from a list of suggestions.
\paragraph{Experimental Approach}\begin{enumerate}
	\item Formulate hypothesis
	\item Implement technique
	\item Train and test
	\item Apply evaluation metric
	\item If not improved:\begin{list}{}{}
		\item Perform error analysis
		\item Revise hypothesis
	\end{list}
	\item Repeat!
\end{enumerate}
\paragraph{Motivations} Language is the most distinctive feature of human intelligence, \textbf{it shapes thought}. Emulating language capabilities is a scientific challenge, a \textbf{keystone for intelligent systems} (see: Turing test)
\paragraph{Structured vs unstructured data} The largest amount of information shared with each other is unstructured, primarily text. Information is mostly communicated by e-mails, reports, articles, conversations, media\ldots and attempts to turn text to structured (HTML) or microformat only scratched the surface.\\
Problems: requires universal agreed \textbf{ontologies} and additional effort. Entity linking attempts to provide a bridge.
\end{document}