\documentclass[10pt]{book}
\usepackage[utf8]{inputenc}
\usepackage[italian]{babel}
\usepackage{multicol}
\usepackage[bookmarks]{hyperref}
\usepackage[a4paper, total={18cm, 25cm}]{geometry}
\usepackage{tikz}
\usepackage{color}
\usepackage{relsize}
\usepackage{amsmath}
\usepackage{amsfonts}
\definecolor{mygray}{rgb}{0.5,0.5,0.5}
\usepackage{listings}
\usepackage{mathrsfs}
\lstset{
	language=Matlab,
	commentstyle=\color{mygray}
}
\usepackage{graphicx}
\usepackage{makecell}
\graphicspath{ {./img/} }
\usepackage{color}

\begin{document}
\renewcommand*\contentsname{Indice}
\title{Teoremi di Calcolo Numerico}
\author{Federico Matteoni}
\date{A.A. 2019/20}
\maketitle
\tableofcontents
\pagebreak
\section*{Introduzione}
Questo PDF raccoglie i vari teoremi con dimostrazioni che possono essere richiesti durante la prova orale, come specificato dal prof. Luca Gemignani per l'A.A. 2019/20.
\chapter{Teoremi}
\section{Teorema 2.2.1: Aritmetica di Macchina} Sia $x \in \mathbb{R}$ con $\omega \leq |x| \leq \Omega$. Si ha $$|\epsilon_x| = \:\vline\:\frac{trn(x) - x}{x}\:\vline \leq u = B^{1-t}$$
\subsection{Dimostrazione} Sia $x = (-1)^sB^p\alpha$. L'errore assoluto $|trn(x) - x|$ risulta maggiorato dalla distanza tra due numeri di macchina consecutivi, quindi $$|trn(x) - x|\leq B^{p-t}$$ Inoltre $|x| \geq B^{p-1}$, pertanto vale $$|\epsilon_x| = \:\vline\:\frac{trn(x) - x}{x}\:\vline \leq \frac{B^{p-t}}{B^{p-1}} = B^{1-t} = u$$
\section{Teorema 3.1.1: Errori nel Calcolo di una Funzione Razionale} In un analisi al primo ordine, vale $$\epsilon_{tot} = \epsilon_{in} + \epsilon_{alg}$$
\subsection{Dimostrazione} $$\epsilon_{tot} = \frac{g(\tilde{x}) - f(x)}{f(x)} = \frac{g(\tilde{x}) - f(\tilde{x})}{f(\tilde{x})}\cdot\frac{f(\tilde{x})}{f(x)} + \frac{f(\tilde{x}) - f(x)}{f(x)} = \epsilon_{alg}(1 + \epsilon_{in}) + \epsilon_{in} \doteq \epsilon_{alg} + \epsilon_{in}$$
Viene espresso il fatto che nel calcolo di una funzione razione, in un'analisi al primo ordine, le due fonti di generazione d'errore forniscono contributi separati che possono essere analizzati indipendentemente.\\
L'obiettivo dell'analisi numerica è \textbf{trovare algoritmi numericamente stabili per problemi ben condizionati}.
\section{Teorema 4.4.1: Localizzazione degli Autovalori}
Sia $A \in \mathbb{C}^{n \times n}$ e siano $K_i$ gli \textbf{insiemi} definiti da $$K_i = \left\{z \in \mathbb{C} \:\:\vline\:\: |z-a_{ii}| \leq \sum_{j = 1, j \neq i}^n |a_{ij}|\right\}\:\:\:\:\textit{per}\:\:i = 1 \ldots n$$ Allora se $\lambda \in \mathbb{C}$ è autovalore di A $$\Rightarrow \lambda \in \bigcup_{i = 1}^n K_i$$\\
Gli autovalori di $A$ quindi appartengono ad almeno uno dei $K_i$, i cosiddetti \textbf{cerchi di Gerschgorin}, dove $a_{ii}$ è l'elemento della diagonale principale, $|z - a_{ii}|$ è il \textbf{centro} e $\sum_{j = 1, j \neq i}^n |a_{ij}|$ è il \textbf{raggio}, formato dalla somma degli elementi della stessa riga escluso l'elemento della diagonale principale.
\subsection{Dimostrazione} $\lambda$ autovalore di $A$ se $\exists\:x \neq 0 \:\:\vline\:\: Ax = \lambda x$. Riscrivendo per componenti, otteniamo
$$Ax = \lambda x \Leftrightarrow \sum_{j = 1}^n a_{ij}x_j = \lambda x_i\:\:\:\:\textsl{per}\:i = 1\ldots n$$
Porto al primo membro
$$\Leftrightarrow  \sum_{j = 1, j \neq i}^n a_{ij}x_j = (\lambda - a_{ii})x_i\:\:\:\:\textsl{per}\:i = 1\ldots n$$
$x \neq 0$, quindi esiste almeno una componente $\neq 0$. Prendiamo quella di modulo massimo $x_p$, quindi $|x_p| = ||x||_\infty$
$$Ax = \lambda x \Rightarrow \sum_{j = 1, j \neq p}^n a_{pj}x_j = (\lambda - a_{pp})x_p \Rightarrow\:\vline\sum_{j = 1, j \neq p}^n a_{pj}x_j\vline = |(\lambda - a_{pp})x_p|$$
Per le proprietà del valore assoluto
$$\Rightarrow |\lambda - a_{pp}|\cdot|x_p| \leq \sum_{j = 1, j \neq p}^n |a_{pj}|\cdot |x_j| $$
$x_p \in \mathbb{C}$ in generale, quindi posso dividere per $|x_p| \in \mathbb{R}$, sicuramente $> 0$ e sicuramente positivo quindi la disequazione non cambia verso
$$\Rightarrow |\lambda - a_{pp}| \leq \sum_{j = 1, j \neq p}^n |a_{pj}|\cdot \frac{|x_j|}{|x_p|} $$
Siccome $x_p$ è la componente di modulo massimo, allora tutte le $\frac{|x_j|}{|x_p|} \leq 1$ posso concludere che
$$\Rightarrow |\lambda - a_{pp}| \leq \sum_{j = 1, j \neq p}^n |a_{pj}|$$
Per la definizione di $K_p$, questa relazione $\Rightarrow \lambda \in K_p$. Dato che non conosco $p$ in generale, $$\Rightarrow \lambda \in \bigcup_{i = 1}^n K_i$$\\\\\\
Abbiamo considerato i cerchi \textbf{per riga}. Ma posso anche considerare i cerchi per colonna, perché \textbf{gli autovalori di $A$ sono gli autovalori di $A^T$} in generale.\\
Questo perché $\textsl{det}(A - \lambda I) = \textsl{det}((A - \lambda I)^T) = \textsl{det}(A^T - \lambda I)$

\section{Teorema 5.1.1: Esistenza ed unicità della fattorizzazione LU} Sia $A \in \mathbb{R}^{n \times n}$ e siano $A_k = A(1:k, 1:k)$ con $k = 1\ldots n$ le \textbf{sottomatrici principali di testa}.\\
Se $A_1, \ldots, A_{n-1}$ sono invertibili $\Rightarrow\exists !\: LU$ di $A$
\subsection{Dimostrazione} Per induzione sulla dimensione della matrice\\
$n = 1$, $A = [a] \Rightarrow a = 1\cdot a$
$$* = \left[ 
\begin{array}{c c | c}
	A_{n-1} & & z\\
	& &\\
	\hline
	v^T & & \alpha
\end{array}
\right]
= \left[ 
\begin{array}{c c | c}
	L_{n-1} & & 0\\
	& &\\
	\hline
	x^T & & 1
\end{array}
\right]
\left[ 
\begin{array}{c c | c}
	U_{n-1} & & y\\
	& &\\
	\hline
	0^T & & \beta
\end{array}
\right]
\Leftrightarrow
\left\{
\begin{array}{l}
A_{n-1} = L_{n-1}U_{n-1}\\
z = L_{n-1}y\\
v^T = x^T U_{n-1}\\
\alpha = x^T y + \beta
\end{array}
\right.
\Leftrightarrow
\left\{
\begin{array}{l}
y = L_{n-1}^{-1}z\\
x^T = v^T U_{n-1}^{-1}\\
\beta = \alpha - x^Ty
\end{array}
\right.
$$
Con $A_{n-1}$ matrice di ordine $n - 1$\\
Le sue sottomatrici principali di testa di ordine $1\ldots n-2$ sono quelle di A e quindi, per ipotesi, invertibili\\
$\Rightarrow$ per ipotesi $\exists !$ fattorizzazione LU di $A_{n-1}$\\
Per ipotesi, $A_{n-1}$ invertibile, quindi
$$0 \neq \textsl{det}(A_{n-1}) = \textsl{det}(L_{n-1}U_{n-1}) =$$
Per Binet
$$ = \textsl{det}(L_{n-1})\textsl{det}(U_{n-1}) = \textsl{det}(U_{n-1})$$
\section{Teorema 6.1.2: Norma Matriciale nei Metodi Iterativi} Il metodo $x^{(k+1)} = Px^{(k)} + q$ è \textbf{convergente} se $\exists$ una norma matriciale indotta dalla norma vettoriale $||\cdot||$ tale che $||P|| < 1$
\subsection{Dimostrazione} $e^{(k + 1)} = Pe^{(k)} \Leftrightarrow ||e^{(k + 1)}|| = ||Pe^{(k)}|| \leq ||P||\cdot||e^{(k)}|| \leq ||P||^2\cdot||e^{(k-1)}|| \leq \ldots \leq ||P||^{k+1}\cdot||e^{(0)}||$\\
$0 \leq ||e^{(k+1)}|| \leq ||P||^{k+1}\cdot||e^{(0)}||$, che per $k\to\infty$
\begin{list}{}{}
	\item $0\to 0$
	\item $||P||^{k+1}\cdot||e^{(0)}||\to 0$ perché $||P|| < 1$ per ipotesi $\Rightarrow||P||^{k+1}\to 0$ per $k\to\infty$
	\item $\Rightarrow||e^{(k+1)}||\to 0$ per il teorema del confronto.
\end{list}
Quindi se trovo una norma come sopra, posso dire che il metodo è convergente: il teorema fornisce una \textbf{condizione sufficiente}.\\
Adesso cerchiamo una condizione necessaria.
\section{Teorema 6.1.3: Raggio Spettrale nei Metodi Iterativi} Se il metodo $x^{(k+1)} = Px^{(k)} + q$ è convergente, allora $\rho(P) < 1$\\
$\rho(P)$ è il \textbf{raggio spettrale} di $P$, il \textbf{modulo dell'autovalore di modulo massimo di $P$}.\\
Quindi il metodo \textbf{non converge} se $\rho(P) \geq 1$, mentre \textbf{può convergere} se $\rho(P) < 1$.
\subsection{Dimostrazione} Se è convergente, allora la successione converge $\forall\: x^{(0)}$ e quindi $\forall\:e^{(0)}$ ho $e^{(k)}\to 0$\\
Prendo $e^{(0)} = v\:|\: Pv = \lambda v$ con $|\lambda| = \rho(P)$ quindi un autovettore relativo all'autovalore di modulo massimo.
$$e^{(k+1)} = Pe^{(k)} = \ldots = P^{k+1}e^{(0)} = P^{k+1}v$$
$$P^{k+1}v = \lambda^{k+1}v = \lambda^{k+1}e^{(0)}$$
$$\Rightarrow||e^{(k+1)}|| = |\lambda|^{k+1}\cdot||e^{(0)}|| \neq 0$$
$$||e^{(k+1)}||\to 0 \Leftrightarrow |\lambda| < 1$$
Si può dimostrare che è una condizione anche sufficiente, quindi \textbf{necessaria e sufficiente}.

\section{Teorema 6.3.1 (1): Convergenza di metodi iterativi} Se $A \in \mathbb{R}^{n \times n}$ è \textbf{predominante diagonale} $\Rightarrow$ A è \textbf{invertibile}
\subsection{Dimostrazione} Segue dal teorema di Gerschgorin, che dice $\lambda$ autovalore di $A \Rightarrow \lambda \in \bigcup_{i=1}^n K_i$.\\
Mostriamo che $0 \not\in \mathlarger{\bigcup_{i=1}^n} K_i \Rightarrow$ 0 non è autovalore $\Rightarrow A$ è invertibile.\\
$K_i = \left\{ z \in \mathbb{C}\: |\: |z - a_{ii}| \leq \mathlarger{\sum_{j=1, j\neq i}^n} |a_{ij}|\right\}$, $0 \in? K_i$\\
$|0 - a_{ii}| = |a_{ii}| > \mathlarger{\sum_{j=1, j\neq i}^n} |a_{ij}|$ per predominanza diagonale, $\Rightarrow 0 \not\in K_i$\\
Vale $\forall\: i \Rightarrow 0 \not\in \mathlarger{\bigcup_{i=1}^n} K_i$ quindi A è invertibile.\\
La dimostrazione per colonne è analoga, utilizzando il teorema di Gerschgorin per colonne.
\subsection{Corollario} $A \in \mathbb{R}^{n \times n}$ predominante diagonale $\Rightarrow \exists!$ la fattorizzazione $LU$ di $A$
\subsubsection{Dimostrazione} Se $A$ è predominante diagonale $\Rightarrow A_k$ è predominante diagonale per $k = 1 .. n-1$\\
$\Rightarrow A_k$ è invertibile $\Rightarrow$ per il teorema di esistenza ed unicità, $A$ ammette unica la fattorizzazione $LU$
\paragraph{} La \textbf{predominanza diagonale} è una \textbf{condizione sufficiente} per: \textbf{invertibilità} ed \textbf{esistenza ed unicità di $LU$}\\
In generale, il metodo di eliminazione Gaussiana \textbf{non preserva} la sparsità: si riempiono le matrici.
\section{Teorema 6.3.1 (2): Convergenza di Metodi Iterativi} Sia $A \in \mathbb{R}^{n \times n}$ \textbf{predominante diagonale} $\Rightarrow A$ è invertibile $\wedge$ Jacobi e Gauss-Seidel sono applicabili e convergenti.
\subsection{Dimostrazione} $A$ predominante diagonale $\Rightarrow A$ invertibile \textbf{OK}\\
$A$ predominante diagonale $\Leftrightarrow |a_{ii}| > \mathlarger{\sum_{j=1, j\neq i}^n} |a_{ij}|$ per $i=1\ldots n$\\
$|a_{ii}| > \mathlarger{\sum_{j=1, j\neq i}^n} |a_{ij}| \geq 0$ per $i=1\ldots n \Rightarrow |a_{ii}| > 0 \Rightarrow a_{ii} \neq 0$ per $i=1\ldots n$ quindi i metodi sono applicabili \textbf{OK}\\
La convergenza dipende dalla matrice d'iterazione $P$, più precisamente dagli autovalori di $P$. Vogliamo dimostrare che gli autovalori di $P$ hanno tutti modulo $< 1$, perché quindi $\rho(P) < 1 \Rightarrow$ convergente.\\
Se cerchiamo per quali valori il determinante è 0 allora il verso del calcolo non cambia perché cambierebbe solo il segno, perché $\textsl{det}(A) = -\textsl{det}(-A)$
$$\textsl{det}(\lambda I - P) = \textsl{det}(\lambda I - M^{-1}N) = \textsl{det}(\lambda M^{-1}M - M^{-1}N) = \textsl{det}( M^{-1}(\lambda M - N)) = \textsl{det}(M^{-1})\cdot\textsl{det}(\lambda M - N)$$
E dato che $M$ è invertibile, $\textsl{det}(M^{-1})\neq 0$, quindi scopriamo che
$$\textsl{det}(\lambda I - P) = 0 \Leftrightarrow \textsl{det}(\lambda M - N) = 0$$
Quindi $\lambda$ autovalore di $P \Leftrightarrow \textsl{det}(\lambda M - N) = 0$ cioè se $\lambda M - N$ è singolare\\
Per quali $\lambda \in \mathbb{C}$ $\lambda M - N$ è una matrice singolare? Se $|\lambda|\geq 1$, mostriamo che $\lambda M - N$ è predominante diagonale e quindi invertibile.\\
Consideriamo il metodo di Gauss-Seidel $M = L + D$, $N = - U$ (per Jacobi c'è una dimostrazione analoga)
$$\lambda M - N = \lambda(L+D)+U$$
Quindi l'elemento sulla diagonale principale è $\lambda a_{ii}$, quindi la condizione di predominanza diagonale (ricordiamo che se non è specificato si sottintende predominanza diagonale per righe) è la seguente.
$$|\lambda a_{ii}| > \sum_{j=1}^{i-1} |\lambda a_{ij}| + \sum_{j=i+1}^{n} |\lambda a_{ij}|$$
Sappiamo che la matrice è predominante diagonale, cioè che
$$|a_{ii}| > \sum_{j=1}^{i-1} |a_{ij}| + \sum_{j=i+1}^{n} |a_{ij}|$$
In $\mathbb{C}$ non c'è un ordinamento, quindi non ci sono le disequazioni, e $\lambda \in \mathbb{C}$ ma $|\lambda| \in \mathbb{R}$ quindi posso moltiplicare ambo i membri per $|\lambda| > 0$
$$|\lambda||a_{ii}| > |\lambda|\sum_{j=1}^{i-1} |a_{ij}| + |\lambda|\sum_{j=i+1}^{n} |a_{ij}|$$
$$|\lambda a_{ii}| > \sum_{j=1}^{i-1} |\lambda a_{ij}| + |\lambda|\sum_{j=i+1}^{n} |a_{ij}|$$
Nella seconda sommatoria lo abbiamo lasciato fuori perché, siccome $|\lambda| \geq 1$ allora vale la seguente disequazione
$$\sum_{j=1}^{i-1} |\lambda a_{ij}| + |\lambda|\sum_{j=i+1}^{n} |a_{ij}| \geq \sum_{j=1}^{i-1} |\lambda a_{ij}| + \sum_{j=i+1}^{n} |a_{ij}|$$
Dimostrando che 
$$|\lambda a_{ii}| > \sum_{j=1}^{i-1} |\lambda a_{ij}| + \sum_{j=i+1}^{n} |a_{ij}|$$

\section{Teorema 10.1.1: Metodi di Bisezione} Se $[a_1, b_1] = [a, b]$ è un \textbf{intervallo di separazione} per $f \in C^0([a, b])$ con $f(a)f(b) < 0$ allora le successioni $a_k, b_k, c_k \longrightarrow \alpha$
\subsection{Dimostrazione} Assumendo che l'intervallo iniziale sia un intervallo di separazione, è un ipotesi semplificata. Quindi nell'intervallo $[a, b]$ non solo c'è una radice ma è unica, e la chiamiamo $\alpha$. Per costruzione $\alpha \in [a_k, b_k]$. La distanza tra $\alpha$ e $a_k$ la possiamo sempre maggiorare con l'ampiezza dell'intervallo.
$$0 \leq |\alpha - a_k| \leq |b_k - a_k| = b_k - a_k = \frac{b_{k-1} - a_{k-1}}{2} = \ldots = \frac{b_1 - a_1}{2^{k-1}} \Rightarrow \lim_{k\to +\infty} |\alpha - a_k| = 0$$
Analogamente per \begin{list}{}{}
	\item $b_k\rightarrow$ $0\leq |\alpha - b_k| \leq \mathlarger{\frac{b_1 - a_1}{2^{k-1}}}$
	\item $c_k\rightarrow$ $0\leq |\alpha - c_k| \leq \mathlarger{\frac{b_1 - a_1}{2^{k}}}$ (notare il $2^k$, leggero vantaggio, perché la distanza tra $c_k$ e $\alpha$ al primo passo si può maggiorare con la semiampiezza dell'intervallo).
\end{list}
Abbiamo mostrato che gli $a_k, b_k, c_k$ convergono alla soluzione.\\
Supponiamo di voler determinare un'approssimazione della radice $\alpha$, denotandola con $\overline{\alpha}$, tale per cui $|\overline{\alpha} - \alpha| < \epsilon$\\ allora se applico il metodo di bisezione ottengo una successione $c_k$ tale per cui $|c_k - \alpha| \leq \mathlarger{\frac{b_1 - a_1}{2^k}}$\\
Chiedo che $\mathlarger{\frac{b_1 - a_1}{2^k}} \leq \epsilon$, così posso prendere $\overline{\alpha} = \epsilon$. Quindi chiedo un $k$ sufficientemente grande da permettermi $$\frac{b_1 - a_1}{2^k} \leq \epsilon \Leftrightarrow 2^k \geq \frac{b_1 - a_1}{\epsilon} \Leftrightarrow k \geq \log_2\left(\frac{1}{\epsilon}\right) + \log_2\left(b_1 - a_1\right)$$
da cui si evince che il numero di passi dipende da $\log_2\left(\mathlarger{\frac{1}{\epsilon}}\right)$.\\
Quindi se volessi una precisione di $10^{-14}$, dovrò fare un numero di passi dell'ordine di $\log_2(10^{14}) = 14\log_2(10)$. Fornisce un criterio \textbf{a priori} sul numero di passi per ottenere una cerca precisione.

\section{Teorema 10.2.2: Teorema del Punto Fisso} Data $g : [a, b] \rightarrow\mathbb{R}$, con $g \in C^1([a, b])$, $\alpha \in (a, b)\:|\:g(\alpha) = \alpha$\\
Se $\exists\:\rho>0\:|\:|g'(x)|<1\:\forall\:x\in I_\alpha = [\alpha - \rho, \alpha + \rho]\subseteq[a, b]$  allora il metodo $\left\{\begin{array}{l}
x_0 \in I_\alpha\\
x_{k+1} = g(x_k)
\end{array}\right.$ genera successioni tali che:
\begin{enumerate}
	\item $x_k \in I_\alpha\:\:\forall\:k\geq 0$
	\item $\mathlarger{\lim_{k\to+\infty}}x_k = \alpha$
\end{enumerate}
\subsection{Dimostrazione} $g$ è una funzione di classe $C^1 \Rightarrow g'$ continua $\Rightarrow |g'|$ continua quindi, per Wierstrass (una funzione continua in un intervallo chiuso e limitato ammette massimo e minimo) $\Rightarrow \mathlarger{\textsl{max}_{x\in I_\alpha}}|g'(x)| = \lambda < 1$\\
Dimostro per induzione che la successione generata dal metodo $\left\{\begin{array}{l}
x_0 \in I_\alpha\\
x_{k+1} = g(x_k)
\end{array}\right.$ soddisfa la relazione\\$|x_k - \alpha| \leq \lambda^k\rho\:\:\forall\:k\geq 0$
\begin{list}{}{}
	\item Passo base $k = 0$\\
	$|x_0 - \alpha| \leq \lambda^0\rho = \rho$ \textbf{ok} per l'ipotesi $x_0 \in I_\alpha$
	\item Passo induttivo $k \Rightarrow k+1$\\
	$|x_{k+1} - \alpha| = |g(x_k) - g(\alpha)| =$ per \textbf{Lagrange} $|g'(\xi_k)(x_k - \alpha)|$ con $\xi_k$ punto tra $x_k$ e $\alpha$\\
	Lagrange dice che $f(a) - f(b) = f'(\xi)(a - b)$ con $\xi$ tra $a$ e $b$\\
	Per ipotesi induttiva $|\xi_k - \alpha| \leq |x_k - \alpha| \leq \lambda^k\rho < \rho$ perché $\lambda < 1$, quindi $\xi_k \in I_\alpha$\\
	Quindi $|x_{k+1} - \alpha| = |g'(\xi_k)|\cdot|x_k - \alpha| \leq \lambda\cdot\lambda^k\rho = \lambda^{k+1}\rho$
\end{list}
Traiamo che tutti i punti della successione appartengono a $I_\alpha$, ma concludiamo anche che $$0 \longleftarrow 0 \leq |x_k - \alpha| \leq \lambda^k\rho \longrightarrow 0$$
Quindi $|x_k - \alpha| \rightarrow 0 \Leftrightarrow x_k \rightarrow \alpha$
\section{Teorema 10.2.3: Convergenza dei Metodi Iterativi Funzionali} Data $g : [a, b] \rightarrow\mathbb{R}$, con $g \in C^1([a, b])$, $\alpha \in (a, b)\:|\:g(\alpha) = \alpha$, se riesco a trovare un intervallo \textbf{centrato in $\alpha$} tale per cui $|g'(x)| < 1$ allora per ogni punto iniziale di quell'intervallo la successione converge ad $\alpha$. Cioè\\
se $|g'(x)| < 1 \Rightarrow \exists\:\rho\geq 0\:\:|$ posto $I_\alpha = [\alpha - \rho, \alpha + \rho]$ il metodo $\left\{\begin{array}{l}
x_0 \in I_\alpha\\
x_{k+1} = g(x_k)
\end{array}\right.$ genera successioni che soddisfano le precedenti proprietà 1 e 2
\subsection{Dimostrazione} $h(x) = |g'(x)| - 1$ continua in $[a, b]$\\
Allora $h(\alpha) = |g'(\alpha)| - 1 < 0 \Rightarrow$ per il teorema della permanenza del segno $\exists$ un intorno di $\alpha$ dove $h(x) < 0 \Leftrightarrow |g'(x)| < 1$ in cui vale la dimostrazione precedente.\\\\
Il corollario ci dice che \textit{esiste un intorno}, ma non ci da informazioni su quando ampio sia o quanto ci si possa spostare da $\alpha$. Ci dà un risultato sulla \textbf{convergenza locale}.\\
Un metodo è localmente convergente se possiamo trovare un intorno tale per cui per ogni punto iniziale preso dall'intorno tutti i punti stanno lì dentro e la successione converge.
\section{Teorema 10.3.1: Convergenza Locale} Sia $f\in C^2([a, b])$ e $f(\alpha) = 0$, $f'(\alpha) \neq 0$, $\alpha \in (a, b)$\\
Allora il \textbf{metodo delle tangenti} è \textbf{localmente convergente in $\alpha$ è la convergenza è almeno quadratica}.\\Cioè $\exists\:I_\alpha = [\alpha - \rho, \alpha + \rho]\:\:|\:\:\forall\:x_0\in I_\alpha$ valgono
\begin{enumerate}
	\item $x_k \in I_\alpha$
	\item $\mathlarger{\lim_{k\to+\infty} x_k = \alpha}$\\
	Fino a qui per la convergenza locale
	\item Se $\forall\:k\:\:x_k\neq\alpha \Rightarrow \mathlarger{\lim_{k\to+\infty} \frac{|x_{k+1} - \alpha|}{|x_k - \alpha|^2}} = c \in \mathbb{R}$
\end{enumerate}
\subsection{Dim} $f'(\alpha) \neq 0 \Rightarrow \exists\:\overline{I_{\alpha}}\:\:|\:\: \forall\:x\in\overline{I_\alpha}\:\:f'(x)\neq 0$\\
Su $\overline{I_\alpha}$ posso definire $g : \overline{I_\alpha} \rightarrow \mathbb{R}$ con $g(x) = x -\mathlarger{\frac{f(x)}{f'(x)}}$\\
$g'(x) = 1 - -\mathlarger{\frac{f'(x)^2 - f(x)f''(x)}{f'(x)^2}} = \mathlarger{\frac{f(x)f''(x)}{f'(x)^2}}$ continua perché composizione di funzioni continue, quindi $g \in C^1$.\\
$g'(\alpha) = \mathlarger{\frac{f(\alpha)f''(\alpha)}{f'(\alpha)^2}} = 0 \Rightarrow |g'(\alpha)| < 1$\\ $\Rightarrow$ per il corollario del teorema del punto fisso $\exists\:I_\alpha\subseteq\overline{I_\alpha}\:\:|$ si ha convergenza locale.\\\\
Prendiamo un punto iniziale $x_0 \in I_\alpha$ e supponiamo che la successione generata soddisfi la proprietà $x_k \neq \alpha\:\:\forall\: k$\\
Ricordiamo l'espressione dell'errore inerente, con la \textbf{formula di Taylor arrestata al secondo ordine} di centro $x_k$ e raggio $\alpha - x_k$
$$0 = f(\alpha) = f(x_k) + f'(x_k)(\alpha - x_k) + \frac{f''(\xi_k)(\alpha - x_k)^2}{2}\:\:\:\:\:\text{con }\alpha < \xi_k < x_k$$
Posso portare $f(x_k)$ al primo membro, dividere per $f'(x_k)$ e portare $x_k$ al primo membro, ottenendo
$$x_k - \frac{f(x_k)}{f'(x_k)} = \alpha + \frac{f''(\xi_k)}{f'(x_k)}\cdot\frac{(\alpha - x_k)^2}{2}$$
Mi accorgo che $\mathlarger{x_k - \frac{f(x_k)}{f'(x_k)} = x_{k+1}}$
$$x_{k+1} = \alpha + \frac{f''(\xi_k)}{f'(x_k)}\cdot\frac{(\alpha - x_k)^2}{2}$$
Posso dividere per $(\alpha - x_k)^2$ e spostare $\alpha$
$$\frac{x_{k+1} - \alpha}{(\alpha - x_k)^2} = \frac{1}{2}\cdot\frac{f''(\xi_k)}{f'(x_k)}$$
Ora posso fare il limite. Sapendo che $x_k \rightarrow \alpha$
$$\lim_{k\to+\infty}\frac{|x_{k+1} - \alpha|}{|\alpha - x_k|^2} = \frac{1}{2}\cdot\:\vline\frac{f''(\alpha)}{f'(\alpha)}\vline \in \mathbb{R}$$
\end{document}