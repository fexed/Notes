\documentclass[10pt]{article}
\usepackage[utf8]{inputenc}
\usepackage[italian]{babel}
\usepackage{multicol}
\usepackage[bookmarks]{hyperref}
\usepackage[a4paper, total={18cm, 25cm}]{geometry}
\usepackage{listings}
\usepackage{graphicx}
\usepackage{makecell}
\graphicspath{ {./img/} }
\usepackage{color}

\begin{document}
\title{Elementi di Calcolo e Complessità}
\author{Federico Matteoni}
\date{ }
\renewcommand*\contentsname{Indice}
\maketitle

\section{Introduzione}
Prof. Pierpaolo Degano \texttt{pierpaolo.degano@unipi.it}\\
Con Giulio Masetti \texttt{giulio.masetti@isti.snr.it}\\
Esame: compitini/scritto + orale

\section{Astrazione}
Termini astratti per descrivere la possibilità di eseguire un programma ed avere un risultato. Un po' come l'equazione per dire qunato ci mette il gesso a cadere.\\
Ma quell'equazione ad esempio non tiene conto dell'attrito: è un \textbf{modello} che non tiene conto di dettagli al momento irrilevanti.\\
\textbf{Non c'è soluzione}: non c'è un programma che possa risolvere il problema in tutte le possibili incarnazioni. Problema.
\paragraph{Problema della Decisione} Risolto se si conosce una \textbf{procedura} che permette di decidere con un numero \textbf{finito} di operazioni di decedere se una proposizione logica è vera o falsa.

\subsubsection{Algoritmo}
Un algoritmo è un insieme \textbf{finito} di istruzioni.
\paragraph{Istruzioni} Elementi da un insieme di \textbf{cardinalità finita} ed ognuna ha \textbf{effetto limitato} (localmente e "\textit{poco}") sui dati. Un'istruzione deve richiedere tempo finito per essere elaborata.
\paragraph{Dati} I dati sono \textbf{discreti}.
\paragraph{Computazione} Successione di istruzioni finite in cui ogni passo dipende solo dai precedenti. Verificando una porzione finita dei dati (\textbf{deterministico}). Non c'è limite alla memoria necessaria al calcolo (è finita ma illimitata). Neanche il tempo è limitato (necessario al calcolo).


\end{document}