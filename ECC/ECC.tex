\documentclass[10pt]{article}
\usepackage[utf8]{inputenc}
\usepackage[italian]{babel}
\usepackage{multicol}
\usepackage[bookmarks]{hyperref}
\usepackage[a4paper, total={18cm, 25cm}]{geometry}
\usepackage{listings}
\usepackage{graphicx}
\usepackage{makecell}
\graphicspath{ {./img/} }
\usepackage{color}

\begin{document}
\title{Elementi di Calcolo e Complessità}
\author{Federico Matteoni}
\date{ }
\renewcommand*\contentsname{Indice}
\maketitle

\section{Introduzione}
Prof. Pierpaolo Degano \texttt{pierpaolo.degano@unipi.it}\\
Con Giulio Masetti \texttt{giulio.masetti@isti.snr.it}\\
Esame: compitini/scritto + orale

\section{Astrazione}
Termini astratti per descrivere la possibilità di eseguire un programma ed avere un risultato. Un po' come l'equazione per dire qunato ci mette il gesso a cadere.\\
Ma quell'equazione ad esempio non tiene conto dell'attrito: è un \textbf{modello} che non tiene conto di dettagli al momento irrilevanti.\\
\textbf{Non c'è soluzione}: non c'è un programma che possa risolvere il problema in tutte le possibili incarnazioni. Problema.
\paragraph{Problema della Decisione} Risolto se si conosce una \textbf{procedura} che permette di decidere con un numero \textbf{finito} di operazioni di decedere se una proposizione logica è vera o falsa.

\subsubsection{Algoritmo}
Un algoritmo è un insieme \textbf{finito} di istruzioni.
\paragraph{Istruzioni} Elementi da un insieme di \textbf{cardinalità finita} ed ognuna ha \textbf{effetto limitato} (localmente e "\textit{poco}") sui dati. Un'istruzione deve richiedere tempo finito per essere elaborata.
\paragraph{Dati} I dati sono \textbf{discreti}.
\paragraph{Computazione} Successione di istruzioni finite in cui ogni passo dipende solo dai precedenti. Verificando una porzione finita dei dati (\textbf{deterministico}). Non c'è limite alla memoria necessaria al calcolo (è finita ma illimitata). Neanche il tempo è limitato (necessario al calcolo). Tanto tempo e tanta memoria quante ce ne servono.

\section{Macchina di Turing}
1936, confuta la speranza "\textit{non ignorabimus}" e la speranza di risolvere qualsiasi cosa con un programma.\\
Turing la presenta supponendo di aver un impiegato precisissimo ma stupido, con una pila di fogli di carta ed una penna, ed un foglio di carta con le istruzioni che esegue con estrema diligenza. Non capisce quello che fa, e si chiama "\textbf{computer}".
Una macchian di turing è una struttura matematica di quattro elementi
\begin{center}
M = \{Q, $\sum$, $\delta$, q$_0$\}
\end{center}
\begin{list}{}{}
	\item Q = insieme finito di stati in cui si può trovare la macchina. Non include elemento \textit{h} che non è uno stato, ma è la fine della computazione
	\item $\sum$ = insieme finito di simboli (A\ldots $\sum_1$\ldots).\\
	Ci sono elementi che devono per forza esistere:
	\begin{list}{}{}
		\item Carattere bianco, carattere dove non c'è scritto niente (come lo spazio o \#)
		\item Carattere di inizio della memoria, chiamato \textbf{respingente} ($\nabla$). Una sorta di inizio file
	\end{list}
	\item $\delta$ = funzione di transizione da Q x $\sum$ $\rightarrow$ Q' (o \textit{h}) x $\sum$' x \{L, R, -\} (muoversi sx, dx, fermo)\\
	Mantiene determinismo perché funzione, ad un elemento associa un solo elemento (la transizione è univoca). Transizioni finite perché prodotto cartesiano di insiemi finiti.\\
Se $\sigma$(q, $\nabla$) = (q', $\nabla$, R), se sono a inzio file possono solo andare a dx.
	\item q$_0 \in $ Q, stato iniziale 
\end{list}
Mappatura a coda di rondine, bigezione tra (m, n) $\rightarrow$ k, cioè $N^2 \rightarrow N$.

\end{document}