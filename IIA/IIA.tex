\documentclass[10pt]{book}
\usepackage[utf8]{inputenc}
\usepackage[italian]{babel}
\usepackage{multicol}
\usepackage[bookmarks]{hyperref}
\usepackage[a4paper, total={18cm, 25cm}]{geometry}
\usepackage{listings}
\usepackage{graphicx}
\usepackage{makecell}
\graphicspath{ {./img/} }
\usepackage{color}

\begin{document}
\renewcommand*\contentsname{Indice}
\title{Introduzione all'Intelligenza Artificiale}
\author{Federico Matteoni}
\date{ }
\maketitle
\tableofcontents
\pagebreak
\section*{Introduzione}
Alessio Micheli, Maria Simi\\
\texttt{elearning.di.unipi.it/course/view.php?id=174}\\
Intelligenza Artificiale si occupa della \textbf{comprensione} e della \textbf{riproduzione} del comportamento \textit{intelligente}.\\
Psicologia cognitiva: obiettivo comprensione intelligenza umana, costruendo modelli computazionali e verifica sperimentale.\\
Approccio costruttivo: costruire entità dotate di intelligenze e \textbf{razionalità}. Questo tramite codifica del pensiero razionale per risolvere problemi che richiedono intelligenza non necessariamente facendolo come lo fa l'uomo.\\
Definizioni di IA: pensiero-azione, umanamente-razionalmente.\\
Costruire macchine intelligenti sia che operino come l'uomo che diversamente.\\
formalizzaz conoscenze e meccanizzazione ragionemtno in tutti i settori dell'uomo\\
comprensione tramite modelli comp della psicologia e comportamente di uomini, animali ecc\\
rendere il lavoro con il calcolatore altrettanto facile e utile che del lavoro con persone capaci, abili e disponibili.\\\\
Poniamo definizione di IA: arte di creare machcine che svolgono funzioni che richiedono intelligenza quando svolte da esseri umani. Non definisce "Intelligenza", cosa significa "intelligente"?\\
...
\paragraph{Agenti intelligenti} Sono situati in un ambiente (percezioni e agiscono su ambiente), hanno abilità sociale (comunicazione. collaborazione, difesa da altri agenti), obiettibi, credenze, intenzioni. Hanno un corpo e forse provano "emozioni".
\end{document}
