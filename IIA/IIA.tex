\documentclass[10pt]{book}
\usepackage[utf8]{inputenc}
\usepackage[italian]{babel}
\usepackage{multicol}
\usepackage[bookmarks]{hyperref}
\usepackage[a4paper, total={18cm, 25cm}]{geometry}
\usepackage{listings}
\usepackage{graphicx}
\usepackage{makecell}
\graphicspath{ {./img/} }
\usepackage{color}

\begin{document}
\renewcommand*\contentsname{Indice}
\title{Introduzione all'Intelligenza Artificiale}
\author{Federico Matteoni}
\date{A.A. 2019/20}
\maketitle
\tableofcontents
\pagebreak
\section*{Introduzione}
Alessio Micheli, Maria Simi\\
\texttt{elearning.di.unipi.it/course/view.php?id=174}\\
Intelligenza Artificiale si occupa della \textbf{comprensione} e della \textbf{riproduzione} del comportamento \textit{intelligente}.\\
Psicologia cognitiva: obiettivo comprensione intelligenza umana, costruendo modelli computazionali e verifica sperimentale.\\
Approccio costruttivo: costruire entità dotate di intelligenze e \textbf{razionalità}. Questo tramite codifica del pensiero razionale per risolvere problemi che richiedono intelligenza non necessariamente facendolo come lo fa l'uomo.\\
Definizioni di IA: pensiero-azione, umanamente-razionalmente.\\
Costruire macchine intelligenti sia che operino come l'uomo che diversamente.\\
formalizzaz conoscenze e meccanizzazione ragionemtno in tutti i settori dell'uomo\\
comprensione tramite modelli comp della psicologia e comportamente di uomini, animali ecc\\
rendere il lavoro con il calcolatore altrettanto facile e utile che del lavoro con persone capaci, abili e disponibili.\\\\
Poniamo definizione di IA: arte di creare macchine che svolgono funzioni che richiedono intelligenza quando svolte da esseri umani. Non definisce "Intelligenza", cosa significa "intelligente"?\\
...
\paragraph{Agenti intelligenti} Sono situati in un ambiente (percezioni e agiscono su ambiente), hanno abilità sociale (comunicazione. collaborazione, difesa da altri agenti), obiettibi, credenze, intenzioni. Hanno un corpo e forse provano "emozioni".

\chapter{Agenti Intelligenti}
\section{Intelligenza}
L'intelligenza ha capacità diverse: buon senso, interazione, esperienza, comunicazione, ragionamento\ldots\\
Non è per risolvere problemi \textbf{specifici}.
\section{Agenti Intelligenti}
\paragraph{Visione del corso} Visione ad agenti offre quadro di riferimento: comoda. Risoluzione problemi vista come ricerca in uno spazio di stati.\\
Agente ha un corpo, sensori, attutatori. Gli agenti sono situati, bilità sociale... non parliamo di un modulo software, ma è qualcosa di più.
Sequenza percettiva: tutta la sequenza di percezioni ricevute. Scelta dell'azione è determinata unicamente dalla sequenza percettiva. Funzione agente definisce l'azione da intraprendere per ogni sequenza percettiva e \textbf{descrive completamente l'agente}.\\
Compito dell'IA è progettare il programma agente.
\paragraph{Agenti razionali} Agente che interagisce con l'ambiente in maniera efficace: "\textit{fa la cosa giusta}". Razionale raggiunge obiettivo nella maniera più efficiente. Misura di prestazione \textit{come vogliamo che il mondo evolva}, a seconda del problema considerato l'ambiente. Esterna importante perché è importante darsi un obiettivo \textbf{prima}.\\
Razionalità relativa e dipendente da: misura prestazione, conoscenze ambiente\ldots\\
Agente razionale: massimizza valore atteso della misura delle prestazioni per ogni sequenza di percezioni considerando le sue percezioni passate e la sua conoscenza pregressa.
\paragraph{Agenti autonomi} nella misura in cui il suo comportamento dipende dalla sua esperienza
\paragraph{PEAS} Prestazioni, Environment, Attuatori, Sensori.
\paragraph{Esempio guidatore di taxi} \ldots
\paragraph{Proprietà Ambiente-Problema} Completamenteo/parzialmente osservabile, agente/multi agente, deterministico/stocastico, episodico/sequenziale, statico/dinamico, discreto/continuo, noto/ignoto
\paragraph{Simulatore di ambienti}
\paragraph{Proprietà degli ambienti}
\paragraph{Strutture di agenti caratteristici}
\paragraph{Architettura} Architettura, un corpo e il \textbf{programma}. Ag: P $\longrightarrow$ ...
\paragraph{Agente reattivo semplice} Programma contiene le condizioni azioni. in base a stato interno e regola fornisce la regola da prendere e con la sua azione agisco.
\paragraph{Agenti con obiettivo}
\paragraph{Agenti che apprendono}
\end{document}
