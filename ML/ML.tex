\documentclass[10pt]{report}
\usepackage[utf8]{inputenc}
\usepackage[italian]{babel}
\usepackage{multicol}
\usepackage[bookmarks]{hyperref}
\usepackage[a4paper, total={18cm, 25cm}]{geometry}
\usepackage{graphicx}
\usepackage{xcolor}
\usepackage{textcomp}
\graphicspath{ {./img/} }
\usepackage{listings}
\usepackage{makecell}
\lstdefinestyle{customasm}{
  belowcaptionskip=1\baselineskip,
  frame=line,
  xleftmargin=\parindent,
  language=[x86masm]Assembler,
  basicstyle=\ttfamily,
  commentstyle=\itshape\color{purple!40!black},
}
\lstset{escapechar=@,style=customasm}
\lstnewenvironment{C}
  {\lstset{language=C++,frame=none}}
  {}
\begin{document}
\title{Machine Learning}
\author{Federico Matteoni}
\date{A.A. 2021/22}
\renewcommand*\contentsname{Index}

\maketitle
\begin{multicols}{2}
\tableofcontents
\end{multicols}
\pagebreak
\section{Introduction}
What is ML? Area of research combining aims of creating computers that could learn and powerful and adaptive statistical tools with rigorous foundation in computational science. Luxury or necessity? Growing availability and need for analysis of empirical data and difficult to provide intelligence and adaptivity by programming it. Change of paradigm.\\
Examples: spam classification, written text recognition\ldots No or poor prior knowledge and rules for solving the problem, but easier to have a source of training experience.\\
ML is considered the latest general-purpose technology, capable of drastically affect pre-existing economic and social structures. And already has. The ultimate aim is to bring benefits to the people by solving big and small problems, accelerating human progress and empowering humans to add intelligence in any other science field.
\end{document}