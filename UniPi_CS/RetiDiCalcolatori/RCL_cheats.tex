\documentclass[12pt]{article}
\usepackage[utf8]{inputenc}
\usepackage[italian]{babel}
\usepackage{multicol}
\usepackage[a4paper, total={18cm, 25cm}]{geometry}
\usepackage{listings}
\usepackage{graphicx}
\graphicspath{ {./img/} }
\usepackage{color}

\begin{document}
\title{Quick Reference per Reti di Calcolatori e Laboratorio}
\author{Federico Matteoni}
\date{V 1.0 -- Primo Compitino}
\renewcommand*\contentsname{Indice}
\definecolor{pblue}{rgb}{0.13,0.13,1}
\definecolor{pgreen}{rgb}{0,0.5,0}
\definecolor{pred}{rgb}{0.9,0,0}
\definecolor{pgrey}{rgb}{0.46,0.45,0.48}
\lstset{
  language=Java,
  showspaces=false,
  showtabs=false,
  breaklines=true,
  showstringspaces=false,
  breakatwhitespace=true,
  commentstyle=\color{pgreen},
  keywordstyle=\color{pblue},
  stringstyle=\color{pred},
  basicstyle=\small\ttfamily
}

\maketitle

\section{TCP}
\subsubsection{Controllo congestione}
\paragraph{cWnd} parte a \textbf{1 MSS}
\paragraph{RENO} \begin{list}{-}{}
\item \texttt{cWnd $<$ soglia} -- crescita \textbf{esponenziale} (\textit{slow start}): + 1 MSS ad ogni ACK
\item \texttt{cWnd $>$ soglia} -- crescita \textbf{lineare} (\textit{AI}): incremento di \texttt{1MSS * (MSS/cWnd)}
\item \textbf{Perdita: 3 ACK duplicati} soglia = cWnd/2, cWnd = soglia + 3 MSS (\textit{fast recovery})
\item \textbf{Timeout}: soglia = cWnd/2, cWnd = 1 MSS (\textit{slow start})
\end{list}
\paragraph{TAHOE} \begin{list}{-}{}
\item \textbf{Timeout/3 ACK duplicati} soglia = cWnd/2, cWnd = 1 MSS
\end{list}
\subsubsection{Throughput}
\texttt{W} = valore massimo di byte della finestra\\
\texttt{throughput = $\frac{0.75 * W}{RTT}$}
\subsubsection{Apertura connessione}
\begin{verbatim}
   SYN    ACK    SEQ num    ACK num    DATA
S:  T      F       1111        -        --
R:  T      T       2222       1112      --
S:  F      T       1112       2223     X byte    //Se piggybacking
\end{verbatim}
\subsubsection{Tempo necessario chiusura connessione}

\section{SMTP}
\subsubsection{Scambio di messaggi}
\paragraph{Apertura connessione}
\begin{verbatim}
R: 220 service ready
S: HELO server.com
R: 250 OK
\end{verbatim}
\paragraph{Invio messaggio}
\subparagraph{Busta}
\begin{verbatim}
S: MAIL FROM: user@server.com
R: 250 OK
S: RCPT TO: receiver@other.com
R: 250 OK
\end{verbatim}
\subparagraph{Intestazioni e corpo}
\begin{verbatim}
S: DATA
R: 354 start mail input
S: From: name surname
S: To: othername othersurname
S: Date: dd/mm/yyyy
S: Subject: text
S: <empty line>
S: <message
S: body>
...
S: .                          //un punto, il <CRLF>.<CRLF>
R: 250 OK
\end{verbatim}
\subparagraph{Chiusura connessione}
\begin{verbatim}
S: QUIT
R: 221 service closed
\end{verbatim}
\section{DNS}
\subsubsection{Risoluzione dei nomi}
n livelli $\Rightarrow$ n server DNS \textit{oltre} quello locale\\
\textbf{Iterativo}: il local name server contatta uno alla volta i name server esterni\\
\textbf{Ricorsivo}: ogni name server contatta il successivo e riceve la risposta, che rigira a chi ha fatto la richiesta.
\end{document}