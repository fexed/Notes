\documentclass[10pt]{article}
\usepackage[utf8]{inputenc}
\usepackage[italian]{babel}
\usepackage{multicol}
\usepackage[a4paper, total={18cm, 25cm}]{geometry}
\usepackage{listings}
\usepackage{graphicx}
\usepackage{makecell}
\graphicspath{ {./img/} }
\usepackage{color}

\begin{document}
\title{Laboratorio di Reti}
\author{Federico Matteoni}
\date{ }
\renewcommand*\contentsname{Indice}

\maketitle
\section{Thread}
\paragraph{Processo} Istanza di un programma in esecuzione
\paragraph{Thread} \textbf{Flusso di esecuzione} all'interno di un processo $\Rightarrow$ Ogni processo ha almeno un thread.\\I \textbf{thread condividono le risorse di un processo}.\\
Possono essere eseguiti sia su single-core (es. interleaving, time-sharing\ldots) che su multicore (più flussi di esecuzione in parallelo)
\paragraph{Multitasking} \begin{list}{}{Si può riferire a}
	\item Processi, controllato esclusivamente dal S.O.
	\item Thread, controllato in parte dal programmatore
\end{list}
\paragraph{Contesto di un processo} Insieme delle informazioni necessarie per ristabilire esattamente lo stato in cui si trova il S.O. nel momento in cui si interrompe l'esecuzione di un processo per passare ad un altro: registri del processore, memoria del processo\ldots
\paragraph{Perché?} Per gestire più funzionalità contemporaneamente, come gestire input, visualizzare a schermo, monitorare la rete ed eseguire calcoli.\\
Esempi noti: browser web, videogame multiplayer. Si creano \textbf{più componenti interagenti} in modo da:\begin{list}{}{}
	\item Usare meglio le risorse
	\item Migliorare le performance per applicazioni che richiedono grossi calcoli: si dividono i task per eseguirli in parallelo.
	\item Anche problemi: difficile debugging e manutenzione, sincronizzazione, deadlocks\ldots
\end{list}
\paragraph{In Java} Il main thread, invocato dalla JVM all'esecuzione del programma, può attivare altri thread. La JVM attiva automaticamente altri thread come il garbage collector.\\
Un thread è un oggetto. Per creare un thread si definisce un task che implementi l'interfaccia \texttt{Runnable} e si crea un thread passandogli l'istanza del task creato. Altrimenti si può estendere la classe \texttt{java.lang.Thread}.
\paragraph{Runnable} Appartiene a \texttt{java.language}, contiene solo la firma del metodo \texttt{void run()}. Un oggetto che la implementa è un frammento di codice che può essere eseguito in un thread.
\paragraph{Stati}
\begin{list}{}{}
	\item \textbf{Created}/\textbf{New}: subito dopo l'istruzione \texttt{new}, variabili allocate ed inizializzate. Thread in attesa di passare in esecuzione
	\item \textbf{Runnable}/\textbf{Running}: thread in esecuzione o in attesa per ottenere la CPU (Java non separa i due stati).
	\item \textbf{Not Runnable} (\textbf{Blocked}/\textbf{Waiting}): thread non può essere messo in esecuzione, può accadere quando attende un'operazione I/O o ha invocato metodi come \texttt{sleep()} oppure \texttt{wait()}.
	\item \textbf{Dead}: termine naturale o dopo l'invocazione di \texttt{stop()} da parte di altri thread (deprecato).
\end{list}
\subsection{Threadpool}
\paragraph{Perché?} In caso di task leggeri molto frequenti risulta impraticabile attivare ulteriori thread. Diventa quindi utile definire un limite massimo di thread che possono essere attivati contemporaneamente, così da sfruttare meglio i processori, evitare troppi thread in competizione e diminuire i costi di attivazione/terminazione dei thread.
\paragraph{Threadpool} Struttura dati la cui dimensione massima può essere prefissata, contenente riferimenti ad un insieme di thread. I thread possono essere riutilizzati: la sottomissione di un task al threadpool è \textbf{disaccoppiata} dall'esecuzione del thread. L'esecuzione può essere ritardata se non vi sono risorse disponibili.\\
La \textbf{politica di gestione dei thread} stabilisce quando i thread vengono attivati (al momento della creazione del pool, on demand, all'arrivo di un nuovo task\ldots) e quando è opportuno terminare l'esecuzione di un thread.\\
Il threadpool, quindi, al momento della sottomissione di un task può:
\begin{list}{}{}
	\item Usare un thread attivato in precedenza e al momento inattivo
	\item Creare un nuovo thread
	\item Memorizzare il task in una coda, in attesa
	\item Respingere la richiesta
\end{list}
\textbf{Runnable} può solo eseguire un'attività in maniera asincrona, senza parametri e senza valori di ritorno.
\paragraph{Callable} Interfaccia per definire un task che può restituire un risultato e sollevare eccezioni.
\paragraph{Future} Interfaccia per rappresentare il risultato di una computazione asincrona. Definisce metodi per controllare se la computazione è terminata, attendere la terminazione oppure cancellarla. Viene implementata nella classe \texttt{FutureTask}.\\
Con i threadpool si sottomette un oggetto \texttt{Callable} e si ottiene un oggetto \texttt{Future}, a cui si applicano metodi per verificare se la computazione è terminata.
\section{Monitor}
\paragraph{Lock Implicite} Ogni oggetto ha associate una lock implicita ed una coda. La lock si acquisisce mediante metodi o blocchi di codice \texttt{synchronized}. Quando questo viene invocato:\begin{list}{}{}
	\item Se nessun metodo \texttt{synchronized} della classe è in esecuzione, l'oggetto viene bloccato (la lock viene acquisita ed il metodo viene eseguito.
	\item Se l'oggetto è già bloccato, il thread viene sospeso nella coda associata finché la lock non viene rilasciata.
\end{list}
Notare che la lock è \textbf{associata all'istanza della classe}, non alla classe. Metodi su istanze (oggetti) diverse della stessa classe possono essere eseguiti concorrentemente.\\
I costruttori non possono essere dichiarati \texttt{synchronized} (errore di compilazione), perché solo il thread che crea l'oggetto deve poterci accedere mentre l'oggetto viene creato.\\
Non ha senso specificare \texttt{synchronized} nelle interfacce poiché è riferito all'implementazione.\\
Inoltre il \texttt{synchronized} non viene ereditato.
\paragraph{Monitor} Meccanismo linguistico ad alto livello per la sincronizzazione. classe di oggetti utilizzabili concorrentemente in modo safe. La risorsa è un oggetto passivo, le sue operazioni vengono invocate da entità attive (thread). La sincronizzazione sullo stato della risorsa è garantita esplicitamente: mutua esclusione sulla struttura garantita dalla lock implicita (un solo thread per volta è all'interno del monitor), meccanismi per sospensione/risveglio sullo stato dell'oggetto condiviso simili a variabili di condizione (\texttt{wait}/\texttt{notify})\\
Il monitor è quindi un \textbf{oggetto} con un insieme di metodi \texttt{synchronized} che incapsula lo stato di una risorsa condivisa. Ha due code gestite implicitamente: entry set (thread in attesa di acquisire la lock) e wait set (thread che hanno eseguito una wait e attendono una notify)
\paragraph{Wait} Sospende il thread in attesa che si verifichi una condizione (opzionalmente per un tempo massimo). Rilascia la lock (a differenza di \texttt{sleep} e \texttt{yield}).
\paragraph{Notify} Sveglia ad un thread in attesa il verificarsi di una certa condizione (\texttt{notifyAll} sveglia tutti i thread in attesa)
\paragraph{Deadlock} Due o più thread bloccati per sempre in attesa uno dell'altro
\paragraph{Starvation} Thread ha difficoltà ad accedere ad una risorsa condivisa e quindi difficoltà a procedere. In generale task "greedy" che invocano spesso metodi lunghi obbligando gli altri ad aspettare
\paragraph{Livelock} Programma che genera una sequenza ciclica di operazioni inutili ai fini dell'effettivo avanzamento della computazione.
\section{Concurrent collections}
\paragraph{Collezioni di oggetti} Insieme di classi che consentono di lavorare con gruppi di oggetti. L'essere o meno thread-safe varia da classe a classe. In generale, tre tipi di collezioni: \textbf{senza supporto} per multithreading, \textbf{synchronized} collections e \textbf{concurrent} collections (introdotte in \texttt{java.util.concurrent}).
\paragraph{Vector} Contenitore elastico (dimensione variabile) e non generico. Thread-safe conservative locking.
\paragraph{ArrayList} Elastico come \texttt{Vector}. Prima di JDK5 poteva contenere solo elementi di tipo \texttt{Object}, adesso è parametrico (generic) rispetto al tipo di oggetti contenuti.\\Elementi possono essere acceduti in modo diretto tramite l'indice. \textbf{Non thread-safe} di default: nessuna sincronizzazione per maggiore efficienza.
\paragraph{Unsynchronized Collections} Come \texttt{ArrayList}, un loro uso incontrollato in un programma multithread può portare a risultati scorretti.
\subsection{Synchronized Collections}
\texttt{Collections} contiene metodi statici per l'elaborazione delle collezioni, \textbf{factory methods} per creare versioni sincronizzate di \texttt{list}/\texttt{set}/\texttt{map}.\begin{list}{}{}
	\item Input: una collezione
	\item Output: la stessa collezione con le operazioni sincronizzate.
\end{list}
La collection risultante è protetta da \textbf{lock sull'intera collezione} $\rightarrow$ degradazione di performance
\paragraph{Nota bene} Nessun thread deve accedere all'oggetto originale, requisito ottenibile con istruzioni del tipo\\\texttt{List<String> synchList = Collections.synchronizedList(new ArrayList<String>());}
\paragraph{Svantaggi} Garantiscono la thread-safety a scapito della scalabilità del programma.
\subsection{Concurrent Collections}
\paragraph{Idea} Accettare un compromesso sulla semantica delle operazioni, così da rendere possibile il mantenimento di un livello di concorrenza tale da garantire una buona scalabilità.\\
Invece di sincronizzare l'intera struttura, si sincronizza solo la parte interessata. Esempio: una hash table ha diversi buckets, quindi sincronizzo solo il bucket a cui accedo.
\paragraph{Concurrent Collections} Implementate in \texttt{java.util.concurrent}, superano l'approccio "\textit{sincronizza l'intera struttura dati}" garantendo quindi un supporto più sofisticato per la sincronizzazione permettendo l'overlapping di operazioni sugli elementi della struttura.
\paragraph{Vantaggio} Maggior livello di concorrenza e quindi migliori performance
\paragraph{Prezzo} Modifica della semantica di alcune operazioni.
\paragraph{Lock Striping} Invece di una sola lock per tutta la struttura, mantengo lock per sezioni. Ad esempio, una hash table suddivisa in sezioni rende possibili write simultanee se modificano sezioni diverse.\\
Posso usare 16 lock per controllare l'accesso alla struttura, un numero arbitrario di lettori ed un numero fisso massimo di scrittori (16) che lavorano simultaneamente. Così lettori e scrittori possono convivere.
\subparagraph{Vantaggi} Maggiore parallelismo e scalabilità
\subparagraph{Svantaggi} Non si può eseguire la lock sull'intera struttura. Si approssima la semantica di alcune operazioni, ad esempio \texttt{size()} e \texttt{isEmpty()}, che restituiscono un valore approssimato.\\
Diventa impossibile sincronizzare funzioni composte da operazioni elementari a livello utente (nelle \texttt{synchronized} si usavano blochi \texttt{synchronized} che lockavano l'intera collezione).\\
Per risolvere questo problema si mettono a disposizione operazioni eseguite atomicamente, come \texttt{putIfAbsent(K key, V val)}, \texttt{removeIfEqual(K key, V val)}, \texttt{replaceIfEqual(K key, V oldVal, V newVal)}
\section{Java NIO}
\paragraph{Obiettivi} Incrementare la performance dell'I/O, fornire un insieme eterogeneo di funzionalità per l'I/O e aumentare l'espressività delle applicazioni.
\paragraph{Non semplice} Per migliorare le performance è necessario definire le primitive a livelli di astrazione più bassi. Inoltre i risultati dipendono dalla piattaforma di esecuzione. Si rende necessario uno sforzo di progettazione maggiore.
\paragraph{Costrutti base}
\subparagraph{Canali e buffer} Invece dell'I/O standard Java che opera su stream di byte o caratteri. I dati sono trasferiti da canali a buffer o viceversa, ed i buffer vengono gestiti esplicitamente dal programmatore.\\
Un \textbf{channel} è simile ad uno stream. I dati possono essere letti dal channel in un buffer e viceversa scritti dal buffer in un channel.
\subparagraph{Buffer} Contenitori di dati di dimensione fissa. Contengono dati appena letti o da scrivere.\\
Sono oggetti della classe \texttt{java.nio.Buffer} \textbf{non thread-safe}.\\
\textbf{Direct Buffers}: buffer allocati fuori dalla JVM, nella memoria gestita dal S.O.
\subparagraph{Selector} Oggetto in grado di monitorare un insieme di canali. Itercetta eventi provenienti dai canali monitorati: dati arrivati, connessione aperta\ldots\\
Sono simili agli stream, ma \textbf{bidirezionali}, con \textbf{scattered read} (distribuisce i dati letti da un canale in uno o più buffer), \textbf{gathering write} (scrive su un canale i dati recuperati da più buffer) e supporta \textbf{trasferimenti diretti tra canali}.
\subparagraph{Channel} Rappresentano connessioni con entità capaci di eseguire operazioni I/O. \texttt{Channel} è un'interfaccia radice di una gerarchia di interfacce:
\begin{list}{}{}
	\item \texttt{FileChannel}: legge/scrive dati su file
	\item \texttt{DatagramChannel}: legge/scrive dati su rete via \texttt{UDP}
	\item \texttt{SocketChannel}: legge/scrive dati su rete via \texttt{TCP}
	\item \texttt{ServerSocketChannel}: attende richieste di connessioni \texttt{TCP} e crea un \texttt{SocketChannel} per ogni connessione creata
\end{list}
Sono \textbf{bidirezionali}. Tutti i dati sono gestiti tramite oggetti di tipo \texttt{Buffer}, quindi senza leggere/scrivere direttamente sul canale. Inoltre possono essere \textbf{bloccanti o meno}, non bloccati utili soprattutto per le comunicazioni in cui i dati arrivano in modo incrementale come i collegamenti di rete.
\section{Network}
\paragraph{Applicazione di rete} In un'applicazione di rete due o più processi (\textbf{non thread}) in \textbf{esecuzione su host diversi} comunicano e cooperano per realizzare una funzionalità globale.
\subparagraph{Cooperazione} Scambio di informazioni utile per perseguire l'obiettivo globale (implica la comunicazione)
\subparagraph{Comunicazione} Utilizza protocolli (insieme di regole da seguire per comunicare correttamente).\\\textbf{Connection-Oriented} oppure \textbf{Connectionless}.
\paragraph{Socket} Uno standard per connettere dispositivi distribuiti, diversi ed eterogenei. È un modo per standardizzare la comunicazione: una "presa" a cui un processo può collegarsi per spedire dati.\\
Viene creato su richiesta dell'applicazione e controllato dal S.O.. Sono associati ad una porta locale, ed un processo applicativo può spedire/ricevere messaggi a/da un altro processo applicativo mediante il proprio socket.
\paragraph{Comunicazione connection-oriented} Come una chiamata telefonica.\\
Viene creata una \textbf{connessione logica end-to-end} tra mittente e destinatario. Si comunica in tre fasi: apertura della connessione, invio dei dati e chiusura.\\
Indirizzo del destinatario specificato all'apertura della connessione.\\
Ordinamento dei messaggi garantito dal protocollo di trasporto\\
Trasmissione affidabile.
\paragraph{Comunicazione connectionless} Come una lettera.\\
Non viene stabilito un canale di comunicazione logico dedicato. Ogni messaggio viene instradato indipendentemente dagli altri (\textbf{one-way message}).\\
Indirizzo del destinatario specificato per ogni pacchetto.\\
Nessuna garanzia sull'ordinamento dei messaggi.\\
Poiché si inviano pochi dati, non è necessario che l'invio sia affidabile.
\subsection{Multicast}
Un indirizzo IP multicast si basa sul concetto di gruppo: tutti i membri di un gruppo multicast ricevono i messaggi spediti su quel gruppo, ma non occorre essere membri del gruppo per spedire i messaggi.\\
Primitive per: unirsi/lasciare il gruppo, spedire/riceve i messaggi. Il supporto deve fornire anche uno schema di indirizzamento, per identificare univocamente il gruppo, oltre ad un meccanismo che registri gli appartenenti al gruppo e che ottimizzi l'uso della rete nel caso di pacchetti multicast.
\paragraph{Indirizzi} Si riserva un certo insieme di indirizzi IP per il multicast: in IPv6 tutti quelli che iniziano per \texttt{FF}, in IPv4 quelli di classe D (\texttt{224.0.0.0} -- \texttt{239.255.255.255})
\paragraph{Addressing} Come scegliere un indirizzo multicast? Deve essere noto collettivamente a tutti i partecipanti del gruppo, e su internet è una procedura complessa con tanti casi da tenere conto. Possono essere
\begin{list}{}{}
	\item \textbf{Statici} se assegnati da un'autorità. Usati da particolari protocolli e applicazioni: indirizzo rimane assegnato a quel gruppo anche se in un dato istante non ha partecipanti.
	\item \textbf{Dinamici} se si usano protocolli particolari che consentono di evitare doppioni: esistono solo fintanto che c'è almeno un partecipante e richiedono un protocollo per l'assegnazione.\\
	Protocolli: \texttt{MADCAP} per una sottorete, \texttt{MASC} per reti generiche.
\end{list}
\paragraph{Scoping} Come limitare la diffusione del pacchetto? \texttt{TTL} oppure \textbf{administrative scoping}: a seconda dell'indirizzo multicast scelto, la diffusione è limitata ad una parte della rete i cui confini sono definiti da un amministratore di rete.
\subparagraph{TTL} Poco usato, il valore è tra 0 -- 255 e indica il numero massimo di router attraversabili
\section{Serializzazione}
\paragraph{Oggetti} Caratterizzati da stato (vive con l'istanza) e comportamento (specificato dalla classe).
\paragraph{Serialization} Scrittura e lettura di oggetti. Si basa sulla possibilità di scrivere lo stato di un oggetto in una forma sequenziale, sufficiente per ricostrutire l'oggetto quando viene riletto.\\
Il flattening/salvataggio dell'oggetto riguarda lo stato e può avvenire in diversi modi:
\begin{list}{}{}
	\item Accedendo ai singoli campi, salvando i valori per esempio su un file di testo in un formato opportuno
	\item Trasformando automaticamente l'oggetto in uno stream di byte mediante il \textbf{meccanismo della object serialization} di Java
\end{list}
La serializzazione consente di persistere degli oggetti al di fuori del ciclo di vita della JVM. Si crea una rappresentazione dell'oggetto indipendente dalla JVM che ha generato l'oggetto stesso. Un'altra JVM, se ha accesso alla classe ed all'oggetto serializzato, può ricostruirlo.
\paragraph{Persistenza} Fornisce un meccanismo di persistenza dei programmi, consentendo l'archiviazione di un oggetto su un file. Ad esempio, per memorizzare lo stato di una sessione.
\paragraph{Interoperabilità} Meccanismo di interoperabilità mediante oggetti condivisi tra diverse applicazioni.
\paragraph{Approcci}
\subparagraph{Java Object Serialization} Utilizzabile solo se chi serializza e chi deserializza è scritto in Java.
\subparagraph{Testo} Come \texttt{XML} e \texttt{JSON} (JavaScript Object Notation), formati dalla sintassi semplice e facilmente riproducibile. Interpretabile da qualsiasi linguaggio.
\section{JSON}
Formato leggero per interscambio di dati, indipendente dalla piattaforma poiché è semplice testo. Non dipende dal linguaggio di programmazione ed è self-describing.
\paragraph{Strutture} Coppie \texttt{chiave:valore} e liste ordinate di valori. Una risorsa JSON ha una struttura ad albero.
\section{RMI}
\paragraph{Remote Method Invocation} Un'applicazione RMI è composta da due programmi: server e client. L'obiettivo è di \textbf{permettere al programmatore di sviluppare applicazioni distribuite} usando la \textbf{stessa sintassi} e semantica \textbf{dei programmi non distribuiti}.\\
Viene raggiunto solo in parte, ha una buona trasparenza ma non totale.
\subparagraph{Server} \textbf{Crea} un oggetto remoto e \textbf{pubblica} un riferimento all'oggetto. Dopodiché, \textbf{attende} che i client invochino metodi sull'oggetto remoto.
\subparagraph{Client} \textbf{Ottiene} un riferimento all'oggetto remoto e \textbf{invoca} i suoi metodi.
\paragraph{Applicazione distribuita} Un'applicazione JAVA distribuita è composta da computazioni eseguite su JVM differenti, possibilmente in esecuzione su host differenti comunicanti tra loro. Un'applicazione multithread non è distribuita, perché è eseguita sulla solita JVM.
\subparagraph{Socket} Richiedono la progettazione di protocolli di comunicazione e la verifica delle loro funzionalità. Sono meccanismi flessibili ma di basso livello. La \textbf{serializzazione} consente di ridurre la complessità dei protocolli inviando dati strutturati.
\paragraph{RPC} Un'alternativa è una tecnologica di più alto livello originariamente chiamata \textbf{Remote Procedure Call}, la cui interfaccia di comunicazione è rappresentata dall'invocazione di una procedura remota invece che dall'utilizzo diretto di un socket.\\
Sfrutta il paradigma \textbf{domanda}/\textbf{risposta}:
\begin{enumerate}
	\item Client invoca procedura del server remoto
	\item Server esegue la procedura con i parametri passati dal client e restituisce al client il risultato
\end{enumerate}
La connessione remota è trasparente rispetto a client e server, prevede meccanismi di affidabilità a livello sottostante.
\paragraph{Funzionamento} Il programmatore non si preoccupa di sviluppare i protocolli per lo scambio di dati, la verifica, la codifica e decodifica. Le \textbf{operazioni} sono \textbf{interamente gestite dal supporto} tramite \textbf{stub}/proxy presenti sul client.
\subparagraph{Marshalling} Preparazione e impacchettamento dell'informazione per la trasmissione. Ad esempio, traduce formati non portabili in formati portabili o canonici. Viene spesso \textbf{implementato con stub generati automaticamente}: il programmatore si limita a sviluppare l'interfaccia per la RPC e lo stub generator prende la definizione dell'interfaccia e crea gli stub per il server e il client.\\
Gli stub si occupano di fare il marshalling/unmarshalling dei parametri, oltre che della comunicazione e dell'invocazione delle procedure. Questo è possibile perché queste azioni sono facilmente definibile dati i parametri e l'identità della procedura.
\paragraph{Limiti della RPC} La programmazione è essenzialmente procedurale, inoltra la locazione del server non è trasparente (il client deve conoscere IP e porta del server). Quindi vengono proposte diverse tecnologie per superarli, tra cui l'RMI.
\paragraph{RMI} Permette di usare veri e propri \textbf{oggetti remoti} come se fossero oggetti come tutti gli altri, senza preoccuparsi della realizzazione di protocolli, connessione ecc\ldots\\
Il loro utilizzo è largamente trasparente: una volta che viene localizzato si possono usare i metodi dell'oggetto come se fosse un oggetto locale. La codifica, decodifica, verifica e trasmissione dei dati sono effettuati dal supporto RMI in maniera trasparente.
\subparagraph{Problemi da gestire} Il client deve in qualche modo trovare il riferimento all'oggetto remoto, e l'oggetto remoto deve esplicitamente rendere i suoi servizi invocabili e rendersi reperibile.
\paragraph{Funzionamento} RMI permette di costruire riferimento remoti.\\
Due \textbf{proxy}: \begin{list}{}{}
	\item Proxy locale: \textbf{stub} nel client, su cui vengono fatte le invocazioni destinate all'oggetto remoto
	\item Entità remota: \textbf{skeleton} nel server, che riceve le invocazioni fatte sullo stub e le realizza effettuando le corrispondenti chiamate sul server.
\end{list}
\textbf{Pattern proxy}: questi componenti nascondono al livello applicativo la natura distribuita dell'applicazione. Non è possibile riferire direttamente l'oggetto remoto, si una una \textbf{variabile d'interfaccia} per contenere il riferimento al proxy, che permette di controllare e preparare il passaggio da client a server.
\paragraph{Remote Reference Layer} RRL, responsabile della gestione dei riferimenti agli oggetti remoti, dei parametri e delle astrazioni stream-oriented connection.
\paragraph{Registry} Servizio di naming che agisce da "pagine gialle":
\begin{list}{}{}
	\item Registra i nomi e i riferimenti agli oggetti i cui metodi possono essere invocati da remoto
	\item Gli oggetti devono essere registrati (\textbf{bind}) con un nome pubblico
	\item I client possono richiedere (\textbf{lookup}) gli oggetti chiedendone un riferimento a partire dal nome pubblico
\end{list}
\paragraph{Operazioni} Le principali operazioni necessarie rispetto all'uso di un oggetto locale sono:
\begin{list}{}{}
	\item Server deve \textbf{esportare} gli oggetti remoti e registrarli tramite la \textbf{bind}
	\item Client deve \textbf{individuare} un riferimento all'oggetto remoto, cercandoli tramite la \textbf{lookup} nel registry
	\item Il client invoca il servizio mediante le chiamate di metodi che sono gli stessi delle invocazioni dei metodi locali.
	\item \textbf{Livello logico} identico all'invocazione di metodi locali
	\item \textbf{Livello di supporto}: gestita dall'RMI che provvede a trasformare i parametri della chiamata remota in dati da spedire in rete. Il network support provvede all'invio vero e proprio.
\end{list}
\paragraph{Considerazioni pratiche}
\begin{list}{}{}
	\item Separazione tra definizione del comportamento (\textbf{interfaccia}) e sua implementazione (\textbf{classe})
	\item Componenti remoti riferiti tramite variabili d'interfaccia.
	\item Le interfacce devono estendere \texttt{java.rmi.Remote} ed ogni metodo deve propagare \texttt{java.rmi.RemoteException}.
	\item Le classi devono implementare l'interfaccia definita ed estendere \texttt{java.rmi.UnicastRemoteObject}.
\end{list}
\paragraph{Interfaccia} L'interfaccia è remota solo se estende \texttt{java.rmi.Remote}.
\texttt{Remote} non definisce metodi ma identifica gli oggetti che possono essere usati in remoto.
\section{REST}
\paragraph{Web Service} Programma che utilizza la connessione web e richiede ad un server esterno di fornire dati o eseguire un algoritmo. Tipologie possibili: RESTful, RPC, ibrido.
\paragraph{REST} Il \textbf{RE}presentational \textbf{S}tate \textbf{T}ransfer è un \textbf{stile architetturale} per lo sviluppo di web services. Viene usato per costruire sistemi debolmente accoppiati: il web è una \textit{istanza} di questo stile architetturale.
\paragraph{Principi base} ROA (Resource Oriented Architecture) è un insieme di linee guida per implmenetare un web service RESTful. I principi base REST sono:
\begin{list}{}{}
	\item \textbf{Resource Identification}\\
	Una \textbf{risorsa} è qualsiasi cosa sia sufficientemente importante da poter essere considerata come un'\textbf{entità}. Bisogna \textbf{dare un nome a tutto ciò di cui si vuole parlare}, utilizzando gli \textbf{URI}.
	\item \textbf{Uniform Interface}\\
	Stesso piccolo insieme di operazioni applicato a \textbf{tutto}. Un piccolo insieme di \textbf{verbi universali} (non inventati in base all'applicazione) applicato ad un largo insieme di \textbf{sostantivi}. Se servono nuovi verbi a molte applicazioni si estende l'interfaccia. \begin{list}{}{}
		\item \texttt{POST}, crea una sottorisorsa
		\item \texttt{GET}, restituisce una rappresentazione della risorsa
		\item \texttt{PUT}, inizializza o aggiorna lo stato di una risorsa
		\item \texttt{DELETE}, elimina una risorsa
		\item Altre: \texttt{HEAD} per leggere i metadati, \texttt{OPTIONS} per le operazioni disponibili, \texttt{PATCH} per modificare una parte.
	\end{list}
	\item \textbf{Self-Descriptive Messages}\\
	Ogni messaggio contiene le informazioni necessarie per la propria gestione. Le risorse sono entità \textbf{astratte} (non utilizzabili di per sé). L'accesso alla risorsa avviene mediante la \textbf{rappresentazione}, cioè allo stato attuale. Il formato è negoziabile, ma deve supportare i link.
	\item \textbf{Hypermedia as the Engine of Application State} (\texttt{HATEOAS})\\
	Sono trasferite rappresentazioni delle risorse contenenti link, con cui il client può procedere al passo successivo dell'interazione. Le risorse e lo stato sono \textbf{navigati} tramite i link $\rightarrow$ \textbf{le applicazioni RESTful navigano invece di chiamare}.
	\item \textbf{Stateless Interactions}\\
	Ogni richiesta al server deve contenere le informazioni necessarie per capire completamente la richiesta, \textbf{indipendentemente dalle richieste precedenti}. La richiesta avviene in completo isolamento.\\
	Ciò non significa applicazioni stateless: l'idea è di evitare transazioni di lunga durata nelle applicazioni. Lo \textbf{stato della risorsa è gestito sul server}, è lo stesso per ogni client quindi se uno lo cambia gli altri devono vedere il cambio. Lo \textbf{stato del client è gestito sul client}, specifico per ognuno e potrebbe avere effetto sulle richieste mandate al server ma non sulle risorse stesse.\\
	Ad esempio è il client a comunicare che mi trovo in una data directory o che ho visualizzato la figura n. 3.
\end{list}
\subsection{Interfaccia Uniforme}
\paragraph{POST} Consente di creare risorse \textbf{suboordinate} ad una esistente, in pratica una append. Spesso il server risponde con \texttt{201 Created}, e nell'header \texttt{Location} è indicato l'URI della risorsa creata utilizzabile per future \texttt{GET}, \texttt{PUT} e \texttt{DELETE}.\\
Utilizzato quando il server decide l'URI.\\
Operazione di lettura e scrittura, può modificare lo stato della risorsa e provocare effetti collaterali sul server.
\paragraph{GET} Operazione di sola lettura, quindi \textbf{idempotente} (può essere ripetuta senza generare effetti sullo stato della risorsa, ma ciò non significa che restituisca ogni volta la stessa rappresentazione).
\paragraph{PUT} Usato per aggiungere una risorsa quando il client decide l'URI.
\paragraph{Idempotente} Un metodo HTTP idempotante genera sul server lo stesso effetto quando applicato 1 o \textit{n} volte. Cioè la seconda richiesta e tutte le successive lasciano lo stato della risorsa nello stesso stato in cui lo ha lasciato la prima richiesta.\\
Le richieste idempotenti possono essere processate molte volte senza effetti collaterali.\\
\texttt{GET}, \texttt{PUT}, \texttt{DELETE}.
\paragraph{Safe} Un metodo HTTP safe non genera effetti collaterali sul server, e possono essere ripetuti \textit{n} volte senza effetti collaterali \textbf{evidenti} (quindi a meno di contatori di accesso, log ecc\ldots)\\
\texttt{GET}, \texttt{HEAD}
\subsection{Content Negotiation}
Utile per quando si aggiornano i requisiti e quindi si rende necessario, con una stessa chiamata, supportare versioni legacy dei client. Con la content negotiation il client manda al server i formati a lui comprensibili (\textbf{MIME types}) e il server sceglie il formato più adatto tra quelli proposti.\\
Il client può anche specificare il grado di preferenza di ogni rappresentazione con i \textbf{fattori di qualità}.\\
Può essere usata su più livelli: formato delle risorse, lingua, charset, codifica\ldots
\paragraph{Rappresentazioni} La rappresentazione scelta dipende da diversi fattori: la natura delle risorse, la capacità del server e del mezzo di comunicazione, oltre che da requisiti e vincoli dello scenario applicativo.
\subsection{Gestione dello Stato}
Essenziale per supportare le stateless interactions. I \textbf{cookies} sono un meccanismo usato di frequente per la gestione dello stato. In molti casi sono usati per mantenere lo stato della sessione, più convenienti rispetto all'incorporare lo stato nella rappresentazione. Hanno due effetti collaterali lato client: sono memorizzati in maniera persistente e indipendente dalla rappresentazione, e sono uno "stato condiviso" con il contesto del browser.\\
I cookies identificativi richiedono un costoso tracciamento lato server: sono potenzialmente globali (non associati a risorse).
\section{URI}
\paragraph{URI} Essenziali per implementare la resource identification. Gli URI sono identificatori \textbf{universali} per le cose, generalmente leggibili dagli umani. Struttura:\begin{center}
\texttt{<scheme>://<authority><path>?<query>\#<fragment>}
\end{center}
Esempio: \texttt{http://www.google.it/search?q=rest\&start=10\#1}\\
I fragment non sono inviati al server.
\paragraph{Linee guida} Preferire i nomi ai verbi, mantenere URI brevi e non cambiarli (usare i redirect).\\
Gli URI REST sono identificatori che devono essere scoperti seguendo i link e non costruiti dal client.
\end{document}