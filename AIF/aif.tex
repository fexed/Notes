\documentclass[10pt]{report}
\usepackage[utf8]{inputenc}
\usepackage[italian]{babel}
\usepackage{multicol}
\usepackage[bookmarks]{hyperref}
\usepackage[a4paper, total={18cm, 25cm}]{geometry}
\usepackage{graphicx}
\usepackage{xcolor}
\usepackage{textcomp}
\graphicspath{ {./img/} }
\usepackage{listings}
\usepackage{makecell}
\lstdefinestyle{customasm}{
  belowcaptionskip=1\baselineskip,
  frame=line,
  xleftmargin=\parindent,
  language=[x86masm]Assembler,
  basicstyle=\ttfamily,
  commentstyle=\itshape\color{purple!40!black},
}
\lstset{escapechar=@,style=customasm}
\lstnewenvironment{C}
  {\lstset{language=C++,frame=none}}
  {}
\begin{document}
\title{Artificial Intelligence Fundamentals}
\author{Federico Matteoni}
\date{A.A. 2021/22}
\renewcommand*\contentsname{Index}

\maketitle
\begin{multicols}{2}
\tableofcontents
\end{multicols}
\pagebreak
\section{Introduction}
Prof.s: Maria Simi, Vincenzo Lomonaco\\
AI is taking over the world. Formalizing common sense is a lot more difficult. We can formalize knowledge in very specific and small domains. But is deep learning the final solution to AI? "It will transform many industries, but it's not magic. Almost all of AI's recent progress is based on one type of AI, in which some input is used to quickly generate simple response." (\textit{Andrew Ng})\\
\textit{This} AI can do supervised learning, but requires huge amount of data (tens of thousands of pictures to build a photo tagger, for example). The rule of thumb of Ng is: if a person can do a mental task with less than one second of thought, we can automate it using AI either now or in the near future.\\
The challenges are:
\begin{list}{}{}
	\item Software is not a problem, the community is open and the software can be replicated the software can be replicated
	\item Data is exceedingly difficult to get access to. Data is the defensible barrier for many businesses
	\item Talent, because downloading and applying open-source software to your data won't work. AI needs to be customized to context and data, that's why there's a war for the scarce AI talent that can do this work.
	\item Computational resources are also very important.
\end{list}
\paragraph{Deep Learning} Is only one approach inside the much wider field of ML and ML is only one approach in the wider field of AI. Book: \textit{Thinking Fast and Slow}, Kahneman. Two systems: system 1 does perceptual tasks, simple computations, system 2 instead does complex computation, recalling from memory\ldots this is a distinction in our brains.
\paragraph{Machine Learning} Is AI all about machine learning? Possible arguments against ML are:
\begin{list}{}{}
	\item Explanation and accountability: ML systems are not (yet?) able to justify in human terms their results. For some applications this is essential: knowledge must be meaningful to humans to be able to generate explanations? Some regulations requires the right to an explanation in decision-making, and seek to prevent discrimination based on race, opinions, sex\ldots (see GDPR)
	\item ML systems learn what's in the data, \textbf{without understanding what's true or false, real or imaginary, fair or unfair}. It is possible to develop unfair, bad models. People are generally more critical about information.
\end{list}
Building AI systems is a goal far from being solved, still quite challenging. Complex AI systems requires the combination of several techniques and approaches, not only ML.
\paragraph{AI Fundamentals} Is mostly about reasoning and \textit{slow thinking}. Different approaches, "good old-fashioned artificial intelligence" or "symbolic AI": teaching about the foundations of the discipline, now 60 years old.
\paragraph{Symbolic AI} High-level human readable representations of problems, the general paradigm of searching for a solution, knowledge representation and reasoning, planning. Dominant paradigm from the mid 1950s until late 1980s.\\
Central to the building of AI systems is the physical symbol systems hypothesis (PSSH), formulated by Newell and Simon (\textit{Computer Science as Empirical Inquiry: Symbols and Search})\\
The approach is based on the assumption that many aspects of intelligence can be achieved by the mainpulation of symbols (the PSSH): \textit{a physical symbol system has the necessary and sufficiente means for general intelligent action}.\\
Human thinking is a king of symbol manipulation system (so a symbol system is \textbf{necessary} for intelligence), and machine can be intelligent (a symbol system is \textbf{sufficient} for intelligence). This cannot be prove, we can only collect empirical evidence: observation and experiments on human behavior in tasks requiring intelligence, and solving tasks of increasing complexity.
\paragraph{Strong and Weak AI} The Chinese room argument, by John Searle, introduced the following distinction: strong ai relies on the \textit{strong} assumption that human intelligence can be reproduced in all its aspects (general AI), including adaptivity, learning, consciousness\ldots, while weak AI is the simulation of human-like behavior, without effetctive thinking or understanding, no claim that it works like the human mind. Dominant approach today, fragmented AI.\\
One strong argument against strong AI is the lack of needs by the sistems: biological need, safety, relationships, self esteem, self-actualization (Maslow's hierarchy of needs).\\
\textit{What stands in the way of all-powerful AI is not a lack of smarts: it's that computers can't have needs, cravings or desires.}
\end{document}