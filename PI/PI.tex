\documentclass[10pt]{article}
\usepackage[utf8]{inputenc}
\usepackage[T1]{fontenc}
\usepackage{lmodern}
\usepackage[italian]{babel}
\usepackage{multicol}
\usepackage[a4paper, total={18cm, 25cm}]{geometry}
\usepackage{array}
\usepackage{graphicx}
\graphicspath{ {./img/} }
\begin{document}
\title{Programmazione d'Interfacce}
\author{Federico Matteoni}
\date{ }
\renewcommand*\contentsname{Indice}

\maketitle
\tableofcontents
\pagebreak
\section{Introduzione}
Appunti del corso di \textbf{Programmazione d'Interfacce} presi a lezione da \textbf{Federico Matteoni}.\\More like \textit{design d'interfacce}.\\\\
Prof.: \textbf{Daniele Mazzei}, mazzei@di.unipi.it\\
\begin{list}{-}{Riferimenti web:}
\item \emph{?}
\end{list}
Esame: compitini/scritto + orale discorsivo dove si discute un software noto.\\Possibile proporre un software personale da presentare all'esame orale, spiegando come si è applicati i rudimenti del corso sul software presentato.\\\\
\begin{list}{-}{Materiale didattico:}
\item \textbf{Google Classroom}, slide presentate a lezione e altro materiale didattico\\Codice \textbf{c14kiy} con le credenziali d'ateneo.\\La suite Google è attivabile a \emph{start.unipi.it/gsuite}\\\textit{Non è autorizzata la divulgazione}
\item La Caffettiera del Masochista, Donald A. Norman\\Eng: The Design of Everyday Things
\item Designing the User Interface, Ben Shneiderman
\item \emph{www.usability.gov}
\item \emph{interaction-design.org}
\end{list}
Ricevimento: Mercoledì 14.30-16, Stanza 366\\

\section{Il Corso}
\textbf{Interface Development in 2020} diventa \textbf{Interface Design in 2020}
\paragraph{Diviso in due parti}
\begin{list}{-}{}
\item \textbf{UX e UI} con introduzione, UI vs UX, HCI, paradigmi, gamification...
\item \textbf{Strumenti per lo sviluppo dell'interfaccia utente} presentati da vari ospiti: Unity, Zerynth, Ubidots, Angular, Amazon Lex, ...
\end{list}

\textbf{Interfaccia} è qualsiasi metodo utilizzato da una persona per \textbf{interagire} con un dispositivo.

\section{Design}
\paragraph{Cos'è il design} Il design è la \textbf{pianificazione o la specifica per la costruzione di un oggetto o sistema} o per l'implementazione di un'attività o processo. Diventa l'esatto opposto della decomposizione del problema in sottopassaggi, cioè del pensiero computazionale. Il design parte dalla base del problema e \textbf{identifica soluzioni per la causa del problema}. Si può avere anche il design di una strategia di implementazione.\\
\quotedblbase \textbf{Bisognerebbe progettare le applicazioni come se fossero persone che ci piacerebbe frequentare}\textquotedblright. Ad esempio Netflix, o il frigorifero.

\subsection{XX Designer}
Discernere tra Graphic Design, User Experience Design (UX Design) e User Interface Design (UI Design).
\paragraph{UX}: come l'utente si sente per interagire e cosa vuole fare. Aspetto più psicologico, guida la UI design in base a statistica fatta su gruppi di utenti. Manda "\textit{l'output}" a chi fa UI e al marketing.\\
\paragraph{UI}: come l'utente interagisce col prodotto (shortcut, sottomenu...)\\
\subsubsection{UX Designer}
Si deve porre il problema di quali approcci usare per risolvere problemi evidenziati da analisi di mercato.\\
\textbf{Chi paga non è detto che sia chi usa il servizio}. Ad esempio Netflix viene pagato da una persona, ma lo stesso account viene usato anche da altre persone (anzi, in particolare \textbf{il 90\% del tempo} chi usa l'account non è chi paga).\\
Uno dei metodi usati per fare UX è quello della \textbf{definizione delle \textbf{personas}} (cioè un archetipo di utente). Una persona può assumere diverse personas.
\paragraph{User Experience} Con User Experience si parla del prodotto e di come si comporta nel mondo reale, che è fatto di \textit{personas}.
\textbf{Non si può progettare \textit{una user experience}, si può progettare \textit{per la user experience}}. La user experience è ciò che fa l'utente, e lo sviluppatore non ha controllo su ciò. L'utente si approccia al software come gli pare.
\subsubsection{UI Designer}
Dalla UX si crea lo \textbf{sketch} dell'interfaccia. Non viene prodotto subito il wireframe ma bisogna partire da altro, ad esempio dai \textbf{casi di studio}. Esistono più casi di studio per ogni personas (casalinga voghera che fa bonifico, casalinga voghera che cambia password, ecc.). Ogni caso di studio è \textbf{specifico per personas}, poiché personas diverse hanno capacità diverse (non conoscere alcuni concetti, non saper fare determinate operazioni...).\\
L'\textbf{UI design è un procedimento diverso dal front-end developing}, quindi possono essere persone separate. Il designer progetta le guideline che istruiscono il developer.\\
Si può dire che la UI design è sottoarea di UX design.\\

\subsection{Front-End Developer}
Esegue il design della UI convertendolo in funzionalità del prodotto.\\

\section{Interfacce Utente}
\paragraph{L'interfaccia utente (UI)} L'UI di un sistema è \textbf{lo spazio dove avviene l'interazione uomo-macchina}: lo schermo, le casse, il mouse e quant'altro.\\
L'obiettivo dell'interfaccia è far si che \textbf{l'utente possa controllare la macchina}, e \textbf{non il contrario}. L'interfaccia può però influenzare il comportamento dell'utente, ad esempio se voglio guidare l'utente in un particolare modo l'interfaccia deve dare un feedback tale da guidare l'utente.\\
L'altro obiettivo dell'interfaccia è \textbf{rendere fruibile \textit{in maniera piacevole} le funzionalità che una macchina eroga} verso l'utente. Il termine \textbf{user-friendly} non può essere omesso: tra un'app facile e piacevole da usare e una solo facile da usare, l'utente medio preferirà sempre la prima.\\
L'interfaccia è strutturata a layer. lo HID (Human Interface Device) è la periferica con cui l'umano interagisce col sistema. Questo server per usare più HID per interagire con diverse applicazioni.\\
HMI (Human Machine Interface) è più astratta rispetto a HCI (Human Computer Interface), quindi in HMI è più teorica la cosa.
\paragraph{Diversi tipi di interfacce} Abbiamo 5 sensi, quindi diverse \textbf{categorie d'interfaccia}: le più comuni sono \textbf{grafiche} e \textbf{tattili} (\textbf{GUI}, Graphical User Interface). Se si aggiunge anche il suono diventano \textbf{MUI} (Multimedia User Interface). Il concetto di GUI è stato coniato in un tempo in cui l'audio era raro. Adesso \textbf{praticamente tutte le interfacce sono MUI}.\\
Esempio di MUI riprogettata in GUI: Facebook. I video partivano in automatico con l'audio attivo, mentre ora sono mutati. Poi sono stati aggiunti i sottotitoli automatici: questo è un esempio di tecnica ideata per le utentze disabili e riusata per poter far fruire il prodotto a quelle personas che in quel momento non possono usufruire dell'audio. \textit{Meglio un sottotitolo sbagliato che niente}.
\paragraph{Categorizzare interfacce} Le CUI (Composite User Interfaces) sono le UI che interagiscono con due o più sensi.
\begin{list}{}{Esistono tre diverse macrocategorie di CUI:}
\item \textbf{Standard}, che utilizzano dispositivi standard come tastiere, mouse e monitor
\item \textbf{Virtuale}, che \textbf{bloccano il mondo reale e creano un mondo virtuale} e tipicamente utilizzano dei caschi VR
\item \textbf{Aumentata}, che \textbf{non blocca il mondo reale e eroga contenuti non completamente digitali}, ma che prendono dalla realtà esterna che circonda l'utente
\end{list}
Le CUI possono anche essere \textbf{classficiate per il numero di sensi} con cui esse interagiscono. Per esempio, lo \textit{Smell-O-Vision} è una CUI standard 3S (3 sensi) con un display, suono e odori. Se si aggiungesse la vibrazione della poltrona, diventerebbe 4S poiché si aggiunge il tatto.\\
Si parla di \textbf{Qualia Interfaces} quando si stimolano tutti i sensi.
\paragraph{Mancata evoluzione} Le UI \textbf{sono le stesse di 10 anni fa}. Bisogna mettere in discussione i paradigmi attuali. L'industria ha convertito l'ambiente fisico della scrivania in ambiente digitale, prendendo ispirazione dall'abitudine dell'utente per rendere più semplice il passaggio. Ora l'utente è abituato, la realtà da cui si prende spunto non esiste più. Bisogna cambiare.

\section{Good and Bad Design}
\textbf{Il buon design non esiste}, poiché si fa design \textit{per} la user experience \textbf{di una determinata personas}.
\begin{list}{}{Le due caratteristiche più importanti su cui misurare il buon design sono:}
\item \textbf{Discoverability}: è la \textbf{capacità innata di un sistema di veicolare i possibili usi e dire come si usa}. Non è detto che una volta che si è capito cosa si può fare si riesca a farlo.\\
Per avere buona discoverability si usa tipicamente la visibilità: un rubinetto con i pomelli bene in vista incrementa la discoverability. Nel software, \textbf{questo lavoro lo fanno i pulsanti}.
\item \textbf{Understanding}: è la \textbf{capacità di comprendere i possibili usi}. Ad esempio: il fornello, è in cucina quindi so che si usa per scaldare ecc., il problema maggiore però è il mapping pomello $\rightarrow$ fornello. Si può risolvere con l'icona del fornello corrispondente, ma non risolve effettivamente il problema. Una soluzione efficacie è disporre fornelli e pomelli in modo che sia evidente la correlazione fra essi.\\
\textbf{Non sottovalutare il costo mentale dell'utente}.
\end{list}
\begin{center}
\includegraphics[scale=0.5]{fornelli.png}
\end{center}

\subsubsection{Design of Useful Things}
\paragraph{Il paradosso di TripAdvisor} \quotedblbase \emph{Quando la gente mangia bene, non recensisce. Quando mangia male, recensisce}\textquotedblright.
\paragraph{Sensazioni} Quando le cose vanno bene, si dimenticano subito. Questo perché, in qualche modo, l'uomo pensa che \textbf{le cose vadano bene per definizione}. Quando qualcosa va storto, invece, \textbf{l'amigdala crea un ricordo con un peso molto maggiore}.\\
Il design deve quindi preoccuparsi di come funzionano le cose, come vengono controllate e della natura delle interazioni. Quando la progettazione è fatta bene, crea prodotti piacevoli e brillanti. Quando è fatta male, i prodotti sono inutilizzabili e ciò porta a notevole frustrazione e irritazione.\\
\textbf{Marcatore somatico}: ricordo le esperienze in base alle sensazioni che provavo durante esse. \textbf{Più forte è la sensazione più si cementifica il ricordo}. Ad esempio se faccio un incidente ad una curva, la ricorderò bene per molto tempo. La strada che faccio per andare in vacanza non la ricordo più già al ritorno.\\
\textbf{Con il software si applica lo stesso discorso}. Se non riesco ad usare un programma inizio a provare frustrazione. Gli umani non informatici tengono a ritenere le macchine come superintelligenti, quindi associano alla frustrazione l'incapacità personale: \textbf{se credo di non essere in grado di usare il software non ci riprovo}.\\
Confrontando IA e intelligenza umana, l'IA risulta strettamente limitata a computazione e risoluzione di problemi logici. Al contrario, \textbf{la mente umana non funziona ad algoritmi ma procede per deduzione}. Per lo più generando ipotesi senza fondamento e autoconvincendosene.\\
Le macchine seguono regole semplici: gli algoritmi. Essi non hanno la flessibilità (\textbf{common sense}) tale da assecondare l'utente. Per esempio, se chiedo telecomando per l'aula D2 ma non esiste o non c'è il proiettore, la signora mi corregge in D1 e dà il telecomando corretto. La macchina dice semplicemente che non esiste l'aula D2 o il proiettore in aula D2.\\
\textbf{Le macchine non hanno buonsenso}. La maggiorparte delle regole sotto il software sono note solo agli sviluppatori. Potrebbe andare bene, basta renderle discoverable.\\\\
Bisogna invertire il paradigma attuale: se qualcosa va storto \textbf{è colpa dello sviluppatore} e non dell'utente. \textbf{Il dovere della macchina è essere comprensibile} da parte dell'utente.\\\\\textbf{Bisogna accettare che il comportamento umano è com'è e non come vogliamo che sia.}
\section{Human Centered Design}
\textit{Alla fine di ogni passaggio c'è l'utente}.\\Si tratta di una norma \texttt{ISO 9241-210:2010(E)}.
\paragraph{Un approccio} Lo HCD è un \textbf{approccio di design} specificamente orientato allo sviluppo di sistemi interattivi con l'\textbf{obiettivo di fare sistemi utili, altamente usabili e che si focalizzino sull'utente}. Il metodo è orientato \textbf{all'efficienza ed all'efficacia}, per aumentare la soddisfazione dell'utente ed evitare il più possibile gli effetti negativi.\\
\paragraph{Prima l'utente, poi le features} Lo HCD mette \textbf{i bisogni, comportamenti e capacità umane prima di tutto, e progetta in funzione di esse}.\\
Significa che se devo risolvere un problema, non mi interessa risolverlo completamente ma raggiungere il miglior risultato che posso far ottenere all'utente che usa il mio software. Se il \textit{70\% degli utenti raggiungono il proprio scopo} col nostro software, allora esso ha un'\textit{efficacia del 70\%}. Posso puntare ad un'efficienza maggiore magari risolvendo una parte minore del problema.\\
\textit{Less is more.} Meglio una feature in meno che una in più. Ogni volta che aggiungi una feature devi dimostrare perché e a cosa serve, perché tale feature va: spiegata, testata, mantenuta oggi e domani (\textbf{backward compatibility}).\\\\
Il problema principale delle UI è un \textbf{problema di comunicazione} in particolare dalla macchina verso la persona. Una buona interfaccia sa comunicare con l'utente.\\
Progettare interfacce che funzionano egregiamente fintanto che le cose vanno bene è relativamente facile, ma \textbf{la comunicazione è ancora più importante quando le cose non vanno bene}: entrano in gioco le \textbf{strategie di mitigazione dell'errore}. Si focalizza l'interazione soprattutto nel \textbf{comunicare ciò che è andato storto}, in quel momento devo aiutare l'utente frustrato a risolvere il problema perché se lo aiuto a risolvere il problema da solo proverà una sensazione positiva di successo per aver capito cosa non funzionava. Ciò \textbf{crea empatia col sistema}.\\Quindi bisogna \textbf{evitare la frustrazione}, e \textbf{aiutare a risolvere} quando insorge un problema.
\paragraph{Capire l'utente} Lo HCD è una filosofia di design che parte dalla \textbf{comprensione delle persone e dei bisogni} che si intende soddisfare. Spesso gli utenti non si rendono conto dei loro effettivi bisogni e nemmeno delle difficoltà che incontrano.\\
Per capire l'utente la tecnica più utilizzata è l'osservazione. Non è detto sia sempre possibile. Versioni alpha e beta non servono solo debuggare il software, ma servono anche a capire ciò che fanno gli utenti. Diventa utile avere statistiche sull'utilizzo effettivo del sistema: quanti click su un determinato pulsante, quante volte una determinata procedura finisce e così via.\\
\textbf{Le specifiche dello HCD}, quindi, \textbf{nascono dalle persone} e per questo \textbf{non si possono scrivere}. Quindi risulta essere un paradigma che si sposa bene con la computer science perché va avanti per iterazioni: si esegue una specifica ad alto livello, ne implemento una parte, la testo sull'utente reale e tramite il feedback modifico la parte implementata e ri-testo. Quando ritengo buono ciò che ho prodotto lo congelo, e passo ad implementare un'altra parte dell'interfaccia.\\
\begin{multicols}{2}
\begin{tabular}{ c | m{10em} }
\multicolumn{2}{c}{ \textbf{Il ruolo dello HCD nel design} }\\
\hline
Experience design & Area di focus\\
\hline
Industrial design & Area di focus\\
\hline
Interaction design & Area di focus\\
\hline
Human Centered Design & Il processo che assicura che la progettazione incontra i bisogni e le capacità degli utenti che useranno il sistema
\end{tabular}
\columnbreak

Possiamo progettare per esperienza utente, il design industriale e progettare per l'interazione. lo HCD non è area di focus del processo di design ma è metodo.\\Utilizzo l' HCD per progettare tutto il resto.
\end{multicols}
\section{Design Thinking vs HCD}
Insieme al termine Human Centered Desing, spesso si può vedere il termine \textbf{Design Thinking}. I termini vengono da due scuole di pensiero molto forti ma con visioni diverse.
\paragraph{Cos'è il Design Thinking} Il \textbf{Design Thinking} segue il filone Stanford, dove è nato: è un \textbf{processo di design} con cui \textbf{progettare nuovi prodotti} che verranno \textbf{effettivamente adottati dalle persone}. Come processo è più vicino alla disruptive innovation che all'antropocentricità.\\
\textbf{Metodo}, strumento per sviluppare prodotti innovativi. Per sviluppare qualsiasi modello di business orientato all'essere profittevole.\\
Si suddivide in 5 fasi iterative.
\subparagraph{Empathize} \textbf{Studiare} il proprio pubblico. Progettare il prodotto in modo che stabilisca un collegamento empatico con l'utente.
\subparagraph{Define} Delineare meglio le \textbf{domande chiave}, cioè quali sono i bisogni a cui assolvere.
\subparagraph{Ideate} \textbf{Brainstorming}, creare soluzioni.
\subparagraph{Prototype} \textbf{Costruire} una o più idee.
\subparagraph{Test} \textbf{Testare} le idee e \textbf{ricevere feeback}.
\paragraph{HCD e DT} Lo HCD è un mindset che viene sovrapposto al design thinking, il quale è orientato a garantire che le idee siano rilevanti e beneficiali, sul lungo termine, per le persone obiettivo.\\
Lo HCD quindi viene sovrapposto all design thinking: identificato il modello di business, uso lo HCD per sincerarmi che la famiglia di soluzioni identificate venga "pulita", attraverso un processo che \textbf{garantisce l'usabilità da parte di soggetti umani}.\\
\begin{center}
\begin{tabular}{ m{8cm} | m{8cm} }
\textbf{Design Thinking} & \textbf{Human Centered Design}\\
\hline
Processo iterativo a 5 fasi che porta all'effettivo sviluppo di prodotti/soluzioni che verranno adottate dall'utente finale desiderato & Mentalità e strumento da applicare insieme al Design Thinking che crea un impatto a lungo termine positivo, per gli utenti della soluzione\\
\hline
\end{tabular}
\end{center}
Quando l'ispirazione (divergente: produrre idee) cala, si passa all'ideazione (convergente: unire le idee simili, scartare idee ridondanti...).
\section{Principi Fondamentali dell'Interazione}
\paragraph{Life is made of experiences} Bravi designer producono \textbf{esperienze} piacevoli. L'esperienza è molto importante, perché determina quanto bene gli utenti si ricorderanno l'interazione.
\paragraph{Cognizione ed Emozione} Quando la tecnologia si comporta in maniera inaspettata, proviamo confusione, frustrazione e rabbia: \textbf{emozioni negative}. Quando invece comprendiamo il comportamento della tecnologia, abbiamo una sensazione di controllo, bravura e persino orgoglio: \textbf{emozioni positive}. \textbf{Cognizione ed emozione sono profondamente legate}. Se non metto l'utente in un \textbf{mood positivo} farà più fatica ad apprendere l'interfaccia. Più mi arrabbio meno sono predisposto a comprendere e riutilizzare il prodotto.
\subsection{Sei Fondamenti} La \textbf{Discoverability}, cioè il grado di facilità con cui un utente \textbf{scopre come funzione l'interfaccia}, è il risultato della corretta applicazione di sei principi psicologici.
\subsubsection{Affordance}
Il termine \textbf{affordance} si riferisce alla \textbf{relazione tra un oggetto fisico e una persona}: precisamente la relazione tra \textbf{le proprietà di un oggetto e le capacità dell'utente che determinano i possibili utilizzi dell'oggetto}.\\
\textbf{Questa proprietà determina il modo con cui l'oggetto può essere usato.}\\
\emph{"Cosa posso fare sull'interfaccia"}.
\paragraph{Esempi} \textbf{Un pulsante di una UI}, da premere con uno HID che sia il dito o il mouse, \textbf{è un oggetto fisico}. Il pulsante \textit{afforda} (\textbf{consente}) l'essere premuto.\\
Una sedia \textit{afforda} il sostenere, quindi \textit{afforda} di sedercisi.\\
Un potenziometro \textit{afforda} l'essere ruotato.
\paragraph{Tipi di affordance} Ci sono affordance \textbf{innate nel cervello}, \textbf{forme} che il sistema visivo e il cervello interpretano automaticamente.\\
L'affordance è una \textbf{proprietà scaturita da una relazione con un particolare soggetto} (quindi è peculiarità della relazione). Ad esempio, una poltrona \textit{afforda} il sostenere per quasi tutti, ma lo spostamento non è detto sia \textit{affordato} per tutti (per esempio una persona debole non può spostare la poltrona).\\
\textbf{Anti-affordance}: prevenzione dell'interazione. Ad esempio degli spunzoni per evitare che piccioni si posino su un cornicione, \textbf{prevengono l'\textit{affordance} che il cornicione ha verso i piccioni di sedersi}.\\
Affordance e anti-affordance \textbf{devono essere discoverable e percievable}. Questo fatto non è scontato: il vetro \textit{afforda} l'essere attraversato dalla luce e non \textit{afforda} l'essere attraversato dalla materia, ma si può non vedere e \textbf{percepire una falsa \textit{affordance}} di passarci attraverso... e ci batto.\\
Un altro esempio: anche a schermo spento, lo smartphone ha comunque l'\textit{affordance} di essere premuto.\\\\
Assolutamente sbagliato dire che "metto un \textit{affordance}". Posso dire che "metto un significante", ma solo se ho un'\textit{affordance}. I tre pallini per il tasto menu sono un \textbf{significante}.
\subsubsection{Signifiers}
I designer hanno problemi pratici: devono sapere come progettare le cose per renderle understandable. Un significante è un \textbf{modo per indicare dove applicare un determinato \textit{affordance} per ottenere un risultato}
\paragraph{Esempi} Un box quadrato in una GUI (un pulsante) è un \textbf{significante}: se applichi l'\textit{affordance} "tocco" qua ottieni un determinato risultato.\\
L'\textit{affordance} del touch, lo slide, il pinch... esiste su tutto lo schermo. L'\textit{affordance} dice \textbf{cosa} posso fare, il significante dice \textbf{dove} fare l'azione.\\
A volte \textbf{i significanti sono indispensabili} perché la maggiorparte delle \textit{affordance} sono invisibili. I significanti servono per fare capire le \textit{affordance} che non si vedono. Per esempio, le porte scorrevoli: se non vedo i cardini, quando vedo la maniglia decido di spingere la porta ma essa non si muove perché è scorrevole. La spinta è un'\textbf{\textit{affordance} percepita} che non esiste.\\\\
Nel design i \textbf{significanti sono molto più importanti delle affordance}, perché \textbf{comunicano come usare il design}. Questo perché viviamo in un mondo in cui le affordance sono state già presentate in genere. Creare nuove affordance è molto molto difficile.
\paragraph{Convenzioni} Come associare l'affordance e il significante ad azioni reali? Nella maggiorparte dei casi tramite \textbf{convenzioni}. La comprensione di un'affordance percepita è dovuta alle convenzioni culturali.
\paragraph{Tipi di signifiers} I significanti possono essere \textbf{voluti} o \textbf{accidentali}.
\subparagraph{Voluto} Ad esempio un'etichetta, una stringa, un'icona.
\subparagraph{Accidentale} Ad esempio delle persone in fila alla stazione.
\subsubsection{Mapping}
Il \textbf{mapping} è di grande importanza nel progettare le interfacce e stabilire i significanti. La \textbf{disposizione} dei significanti, a parità di significanti, può dire di più sull'interfaccia e le funzionalità.\\
Il \textbf{mapping è la relazione tra elementi di due insiemi}. Il modo migliore per fare mapping è \textbf{quello naturale}, perché è un'attività in cui il nostro cervello è molto bravo, ed il mapping di forme geometriche è la prima cosa che si impara da bambini.
\subsubsection{Feedback}
Un altro elemento fondamentale per il design delle interfacce è il \textbf{feedback} inteso come \textbf{risposta dell'interfaccia verso l'utente}.
\paragraph{Immediato} Il feedback \textbf{deve essere immediato}. Il sistema sensoriale è parte integrante del sistema cognitivo, e l'uomo usa i propri sensi per guidare i propri ragionamenti. Se progetto un'interfaccia che non abilita i miei sensi a capire cosa sto facendo, inizio a fare più fatica a usare il prodotto o non ci riesco proprio. Un esempio: una pagina web che \textbf{non mostra se sta caricando la procedura richiesta}.\\
Uno dei \textbf{problemi principali} del feedback quindi è \textbf{il tempo}. Se faccio un'azione, \textbf{devo avere un feedback entro un certo lasso di tempo}. Se questo tempo è superato, il mio cervello non è più in grado di associare il feedback all'azione compiuta e ho così due pessimi risultati: \textbf{non ho dato feedback} e \textbf{ho mandato in confusione l'utente}.\\
\textbf{Il feedback deve avvenire entro massimo 100 ms dall'azione, altrimenti non sarà efficace}. Meglio un \textbf{buon} feedback che un \textbf{bel} feedback.
\paragraph{Informativo} Inoltre, il feedback \textbf{deve essere informativo}. Questo non significa che deve portare con sé tanta informazione, ma che deve \textbf{assolvere al proprio obiettivo}. Un esempio: se premo un pulsante non ho bisogno di fare grandi cose come feedback, posso \textbf{semplicemente} farlo diventare grigio. Non servono messaggi del tipo "\textit{ok pulsante premuto}" ecc., sono superflui.\\
Colorare il pulsante di rosso o di verde \textbf{non è più informativo}: creo confusione a causa del mapping naturale tra il colore e il significato (rosso $\rightarrow$ errore, verde $\rightarrow$ successo) e l'utente non capirà se ha ottenuto un errore o se la richiesta è stata ricevuta correttamente.\\
Il feedback deve quindi essere \textbf{informativo nell'accezione dell'azione a cui è associato}. \emph{Meglio nessun feedback rispetto ad un feedback errato}.
\paragraph{Semplicità} \textbf{Non bisogna essere troppo pedanti}. Se il feedback è eccessivo, l'interfaccia utente diventa pesante.\\
Altro problema che può insorgere è un feedback non allineato con il contesto dell'utilizzo del dispositivo. Per esempio, non posso usare lo stesso beep delle cinture per segnalare la riserva. Il beep delle cinture è fastidioso perché \textit{deve esserlo}, ma la riserva, quando viene segnalata, non è in un contesto urgente. Se il beep è fastidioso, o spaventa, posso mettere in pericolo la vita dell'autista se viene spaventato mentre guida. \textbf{Non limitarsi alla tecnologia disponibile} "\textit{ho solo quel buzzer, non posso fare altrimenti}". Nell'esempio non sono obbligato a far partire un beep quando si entra in riserva, posso \textbf{semplicemente} fare lampeggiare la spia.
\paragraph{Esempi} Il \textbf{feedback} della luce del pulsante dell'ascensore quando viene premuto.\\Un messaggio "\textit{Pagamento eseguito}" a termine di una procedura pagamento.
\subsubsection{Conceptual Model}
Un \textbf{modello concettuale} è \textbf{una descrizione estremamente semplificata delle funzionalità del sistema}. L'esempio classico sono i file e le cartelle. Come racconto a qualcuno com'è organizzata memoria di archiviazione di un computer? Uso un \textbf{modello concettuale noto}, cioè \textit{fogli di carta con contenuti, vengono raccolti in raccoglitori e quest'ultimi raccolti in schedari}.\\
Agli utenti \textbf{non interessano come funzionano} le cose, ma che \textbf{funzionino}. Perché l'hanno comprato.\\\\
Il modello concettuale è \textbf{come il designer vuole che l'utente percepisca la piattaforma}. Sarebbe l'\textit{ambizione} di progettare (la comprensione) della UX.
\paragraph{Per l'utente} I modelli concettuali servono per \textbf{andare incontro all'utente}, per convertire i vari aspetti di complessità tecnica in aspetti \textbf{comprensibili da chiunque}. I \textbf{modelli concettuali già in commercio sono difficili da mettere in discussione}. Questo perché, in caso si esca con un prodotto concorrente ad uno già affermato ma che funziona in modo diverso (diverso modello concettuale), l'utente diventa costretto a confrontare i due prodotti. \textbf{Mai far valutare "\textit{uno contro uno}" agli umani}, perché non esistono le sfumature ma la valutazione si risolverà in \textbf{vivo o morto}. Inventare un nuovo sistema di streaming musica/film è pressoché \textbf{impossibile}. La gente ha Spotify e Netflix, non importa cosa fanno o come.

Modello concettuale "\textit{film non comprati su DVD}" $\rightarrow$ \textbf{Netflix}

Modello concettuale "\textit{musica non comprata su CD}" $\rightarrow$ \textbf{Spotify}

\paragraph{Modello Mentale} Una volta pensato e progettato il modello concettuale si \textbf{implementa l'interfaccia in modo che il modello concettuale venga veicolato all'utente} tramite i significanti.\\
Quando persona si interfaccia con un sistema, sviluppa un \textbf{modello mentale}. Se il modello mentale e quello concettuale sono \textbf{allineati}, la persona \textbf{è in grado di usare il sistema}. \textbf{Più è grande la differenza} tra il modello mentale e  quello concettuale, \textbf{più la persona farà fatica} ad usare il sistema. Inoltre, l'utente può sviluppare \textbf{modelli mentali diversi per diverse funzionalità} dello stesso sistema.\\
Il \textbf{modello concettuale viene trasferito all'utente per spiegare come funziona l'interfaccia}, non com'è fatta. Il pomello che comanda un determinato fornello è una questione di \textit{mapping}, ma \textbf{che il fornello spruzzi fuoco o acqua è modello mentale dell'utente}.
\paragraph{Telefono senza fili} Solitamente, e idealmente, la gente apprende i modelli concettuali direttamente dal device andando per tentativi. La \textbf{problematica è quando lo apprende per passaparola}. Seguendo la filosofia del \textit{telefono senza fili}, quando avviene il passaparola da persona a persona cambia l'iterpretazione. Per ogni disallineamento tra modello concettuale e mentale, l'interpretazione cambia di conseguenza. Per questo vi è \textbf{necessità che il modello concettuale sia pressoché univoco con quello mentale}. Più piccolo è il delta tra i due, meno rischio di creare comportamenti assurdi col passaparole.\\
\textbf{Less is more}, se la feature è difficile da veicolare non metterla.\\\\
\textbf{Quando gli utenti di una determinata feature la usano con successo l'85\% delle volte, e il 70\% degli utenti del sistema usano quella feature, allora posso inserirne un'altra}.
\subsubsection{System Image}
Le persone creano \textbf{modelli mentali di sé stessi, degli altri, dell'ambiente che hanno intorno e delle cose con cui interagiscono}. Questi modelli mentali sono creati attraverso l'\textbf{esperienza}, l'\textbf{allenamento} e l'\textbf{istruzione}.\\
Nell'immagine di sistema troviamo tutti questi componenti. Essa è il \textbf{modello concettuale che l'utente si crea dell'intero sistema grazie all'esperienza}. Si può quasi descrivere come l'insieme dei modelli mentali, racchiude il modello concettuale e quello mentale e descrive come stanno nel sistema complesso.
\paragraph{Scoperta} L'utente non può chiedere allo sviluppatore come funziona il sistema, ma \textbf{deve scoprire le funzionalità da solo}. La \textbf{teoria dell'immagine di sistema} dice che l'utente sviluppa un proprio \textbf{user model} grazie all'oggetto e agli elementi di contorno (il manuale d'istruzioni, la pubblicità, il passaparola ecc.). Quindi \textbf{il modello concettuale è solo parte dell'immagine di sistema}. Devo \textbf{aiutare l'utente a sviluppare un modello mentale vicino a quello concettuale}, dotandolo di altre informazioni, come ad esempio il blog del software.\\
Il modello concettuale è \textbf{ciò che voglio che l'utente pensi}, il modello mentale \textbf{ciò che l'utente pensa}. L'\textbf{immagine di sistema è insieme delle informazioni} (UI, materiali di contorno) \textbf{che consente all'utente di formarsi il modello mentale più consono}. Bisogna essere bravi a farlo in modo che tutti i modelli generati siano compatibili con il modello concettuale.
\section{Cambiare le convenzioni}
Viene naturale pensare che l'innovazione debba essere un segno di discontinuità con il passato (\textbf{disruptive innovation}), ma far digerire questo agli utenti spesso è un \textbf{problema}. Ogni volta che si \textbf{viola una convenzione}, sia essa culturale, legale, tecnologica, o anche frutto di pessime abitudini, si \textbf{chiede all'utente di fare un nuovo passaggio di apprendimento}: questa richiesta genera \textbf{attrito} con l'utente perché il cambiamento mette stress, indipendentemente dai meriti del nuovo sistema. Se devo cambiare abitudini ci penso due volte, \textit{perché cambiare fa fatica}.\\
Quindi il processo del cambiamento delle convenzioni \textbf{dovrebbe essere graduale}.
\paragraph{Consistenza} Bisogna cercare di trovare il \textbf{modo per rimanere consistenti}. Diventa necessario regolare la quantità di innovazione introdotta ad ogni passo in modo che sia \textbf{abbastanza da portare l'utente avanti} ma \textbf{non troppo da essere percepita come diversa}. In altri termini, il \textbf{modello mentale} che l'utente si è costruito \textbf{deve rimanere tale, solo leggermente allargato}. La volta dopo lo si allargherà ancora e così via.
\paragraph{No Sistemi Misti!} La cosa peggiore che si può fare è quella di \textbf{innovare lasciando il vecchio sistema presente per un po'}, in modo da "\textit{facilitare il passaggio}". Al contrario, ciò genera solo confusione per l'utente.\\
I sistemi misti confondono perché l'utente è portato a creare un nuovo modello mentale che è un mix dei due. L'utente non si chiederà il perchè del nuovo modello, quindi se gli diamo la possibilità di scegliere andrà sul sicuro, cioè il vecchio sistema. Quando il vecchio sistema verrà tolto, l'utente entrerà in crisi.
\subsection{Rethinking OS} Mercury OS (link slide)\\
Tanta innovazione abilitata da richiesta per consentire accesso a categorie meno fortunate si è trasformata nell'ottimizzazione dell'intero paradigma. es sottotioli prima per sordi ora tutti i video sono sottotitolati. Altro es riscrivere un'interfaccia per deficit attenzione, diventa meno faticosa anche per gli altri.\\Reinventare SO facendo si che comunicazione tra applicazioni esista e sia fluida (nella realtà l'interoperabilità fra le app si crea grazie all'utente)\\
PEnsa SO a chi usa pc come strumento di lavoro, computer come sistema di ingresso uscita, interfaccia verso il proprio lavoro. SO in cui l'elemento cardine non siano finestre ma il tempo. Flussi composti da moduli, elemento atomico base. In uno spazio diversi flussi, flussi diversi condividono moduli. il concetto di app rientra nel concetto di spazio. spazi/flussi generabili con moduli taggati.\\
nomi oggetti (cose tangibili, funzionali), verbi affordance (cosa posso fare con nome), modifiers (quasi)significanti (azioni specifiche che consentono di attuare quei verbi e ottenere risultati)

\section{Constraints, Discoverability and Feedback}
\subsection{Constraints} Ricordando che \textit{non si può progettare LA UX} ma \textit{si progetta PER la UX}, \textbf{non si può vincolare} più di tanto l'utente a fare specifiche azioni.
\paragraph{Vincolare} I \textbf{constraint} in un certo senso vanno esattamente contro la precedente frase. Si possono usare i constraint perché quando l'utente vede la nostra interfaccia per la prima volta si fa un modello mentale \textbf{mischiando la propria conoscenza pregressa}. Si limita quindi l'utente nella libertà d'azione, impedendo eventuali affordance e signifier percepiti grazie alla sua conoscenza.\\
Il \textbf{vincolo} più famoso è la \textbf{pila stilo}: si può inserire in un solo verso, e se si prova ad inserirla al contrario si fa fatica a spingere la molla. Inoltre, pur presentando esempi chiari di affordance, significanti (+ e -, disegno ecc.), si è reso necessario un sistema di vincoli che evita l'inversione batterie.\\ Perché creare dei \textbf{vincoli fisici}? Perché non è detto che un determinato vincolo culturale sia onnipresente. Il \textbf{vincolo fisico è l'ultima spiaggia}. Per progettare una UX a volte è possibile usare anche solo vincoli.
\paragraph{Categorie di vincoli}
\begin{list}{}{I vincoli che tipicamente si trovano in nelle interfacce si dividono in 4 macrocategorie:}
\item \textbf{Fisici}: ad esempio un pezzo Lego che entra solo in un determinato verso, il tappo di una biro entra solo in un verso. Sono vincoli concreti del mondo fisico
\item \textbf{Culturali}: ad esempio la guida a destra, indossare la t-shirt sul torace
\item \textbf{Semantici}: vincoli relativi ai significati, ad esempio il limite di velocità (cioè la semantica del cartello stradale con il numero)
\item \textbf{Logici}
\end{list}
\pagebreak
L'assenza di vincoli e mapping genera frustrazione.\\
\begin{multicols}{2}
\includegraphics[scale=0.75]{interuttori.png}\\
\columnbreak

Vincoli e mapping alle volte si confondono fra loro. Nell'esempio, posizionare gli interuttori in corrispondenza delle luci sulla piantina della stanza è un mapping così forte che è quasi un vincolo logico: non ti puoi permettere di sbagliare interuttore.
\end{multicols}
\subsection{Forcing functions}
\textbf{Forzare le funzioni} è una forma di \textbf{vincolo fisico}.\\\\
\textbf{Interlock}: azione \textbf{dopo serie di passi}\\
\textbf{Lock-In}: azione \textbf{prima di concludere}\\
\textbf{Lock-Out}: azione \textbf{prima di iniziare}
\subsubsection{Interlock}
L'\textbf{interlock} consiste nell'\textbf{obbligare l'utente ad eseguire una successione di azioni} per raggiungere un certo stato/obiettivo (ad esempio premere pedale e due pulsanti distanti per azionare una pressa idraulica) oppure anche per guidare il learning (ad esempio un tutorial).
\paragraph{Esempi} "\textit{Verifica la tua mail}" prima di poter accedere con un account appena creato.
\subsubsection{Lock-In}
Con un \textbf{lock-in} invece \textbf{metto in pausa l'attuale situazione fino a che l'utente non ha fatto una determinata cosa}. 
\paragraph{Esempi} Non puoi uscire dall'editing di un documento se consciamente non mi hai detto di salvarlo o buttarlo.\\\\
A volte una lock-in \textbf{funziona così bene che diventa una shortcut}: chiudo e uso la finestrella di lock-in per salvare, invece di andare col mouse a premere tasto in alto a sinistra.
\subsubsection{Lock-Out}
Una \textbf{lockout} è l'opposto della lock-in, cioè \textbf{chiude fuori l'utente finché non compie una determinata azione}
\paragraph{Esempi} Finestra \textit{v.m. 18}, assolve alla richiesta legale di vietare l'accesso ai minorenni. Io utente posso \textbf{dichiarare il falso} ed entrare comunque, ma così il reato l'ho fatto io e non il sito.
\subsection{Activity-Centered Controls}
In molti casi è comodo avere \textbf{controlli associati alle attività} piuttosto che alle funzioni.\\Particolare modalità di interazione utente che usa tutti i principi precedentemente elencati.
\paragraph{Un esempio} Invece di avere un pulsante "\textit{abbassa telo}", un altro pulsante "\textit{alimenta proiettore}", un altro "\textit{accendi proiettore}" e così via, posso fare un pulsante "\textit{\textbf{presentazioni con slide}}" che esegue tutte quelle operazioni.\\Altro esempio sono i preset per le impostazioni di schermi e audio: si possono ottenere i medesimi risultati manualmente. Il preset lo rende semplicemente più veloce.
\paragraph{Teoria vs Pratica} Nella teoria le activity-centered controls sono eccellenti, ma nella pratica sono difficili da costruire bene. Se fatte male, creano \textbf{difficoltà}. Per questo \textbf{le activity-centered controls devono essere disegnate sugli utenti} e non sullo sviluppatore.\\
L'elettricista programmerà il pulsante "\textit{presentazioni con slide}" \textbf{con la sua idea} di presentazione: probabilmente questa idea sarà super completa ma \textbf{talmente specifica che non andrà bene a nessun utente}. Invece \textbf{bisogna fare user activity-centered control}: tutta la parte sullo HCD si applica anche qua, cioù chiedersi chi è l'utente, le sue caratteristiche e così via. Ciò deve portare a \textbf{creare quella scena di presentazione che abilita tutti gli utenti}. Eventualmente per qualcuno andrà bene così com'è, mentre per un altro potrebbe essere incompleta e premerà altri pulsanti.\\\\
Un altro errore comune è creare un'apparente activity-centered control ma in realtà creare device-centered control.
\section{How People do Things}
\paragraph{Come fanno le persone a fare cose?} \textbf{Finchè le cose vanno bene allora sembrano essere semplici}, l'utente crede di aver chiaro come ha ottenuto un risultato e allo sviluppatore sembra chiaro come l'utente esegue le attività.\\
Invece, \textbf{quando le cose vanno male allora l'interpretazione è opposta}, cioè l'utente non capisce perché siano andate male e diventa frustrato: questo accade perché \textbf{il modo in cui le persone fanno le cose è complesso}. I fenomini mentali in atto quando si fanno le cose sono complessi.\\
\textbf{Quando le cose vanno bene non ce ne rendiamo conto}, ma \textbf{quando ci sono problemi ce ne rendiamo conto \textit{in una serie di fasi}}, che sono le fasi eseguite durante l'ottenimento di un obiettivo.
\paragraph{Come ottenere un obiettivo} Si dà per scontato che le persone eseguano delle azioni, ma \textbf{prima di tutto persone scelgono le azioni da compiere}. Si dà per scontato che l'utente scelga l'azione scelta dal progettista.
\subsection{I Golfi}
\begin{multicols}{2}
\includegraphics[scale=0.6]{golfi.png}
\columnbreak


Immagino l'\textbf{utente} e il \textbf{mondo fisico/device} contrapposti. Considero due "flussi": quello dell'\textbf{esecuzione} di una serie di attività e quello della \textbf{valutazione} della risposta ottenuta dal sistema.\\
I due flussi vengono descritti come due \textbf{golfi}, seguendo l'analogia che \textbf{tanto più profondo è un golfo tanto più lunga è la strada da percorrere per giungere dall'altra parte}.
\begin{list}{-}{}
\item Nel \textbf{golfo dell'esecuzione} gli utenti cercano di capire cosa fare e, una volta capito, eseguono le azioni
\item Nel \textbf{golfo della valutazione} l'utente interpreta e valuta il feedback ricevuto dal sistema
\end{list}
Il golfo della valutazione è ritenuto dai designer il più semplice da superare. Nel golfo della valutazione c'è una quantità di effort mentale che l'utente usa per comprendere l'interfaccia e capire se l'azione che aveva deciso di fare è stata eseguita come da lui pianificato. \textbf{Tanto più il modello mentale differisce} da quello concettuale, \textbf{tanto più il golfo valutazione sarà grande}.
\end{multicols}
\paragraph{Mitigare} Gli elementi del design che contribuiscono alla mitigazione del \textbf{golfo dell'esecuzione} sono: \textbf{significanti}, \textbf{constraint}, \textbf{mapping} e \textbf{modello concettuale}.\\
Gli elementi che contribuiscono a alla mitigazione del \textbf{golfo della valutazione}: \textbf{feedback} e \textbf{modello concettuale}.
\paragraph{Fare ed Interpretare} Le \textbf{due parti di un'azione} sono: \textbf{fare} l'azione e \textbf{valutare} i risultati. In entrambe le parti bisogna \textbf{garantire l'understanding}, cioè la \textbf{discoverability} e la \textbf{visibility}: ad esempio se non so quando ottengo punti ma vedo aumentarli apparentemente a caso non posso capire perché li ottengo se non è facilmente discoverable.
\subsection{Sette Stati dell'Azione}
\begin{multicols}{2}
Sono i sette stati attraverso i quali passano i due golfi:\\
Per prima cosa \textbf{(1)} si \textbf{specifica il proprio obiettivo}. Dopodiché si passa ai 3 stati dell'esecuzione:
\begin{list}{}{}
\item \textbf{2}: \textbf{pianifico} ciò che devo fare
\item \textbf{3}: \textbf{specifico} la mia pianificazione in task
\item \textbf{4}: \textbf{eseguo} i task pensati: click del pulsante, riempimento dei campi...
\end{list}
Anche la valutazione ha 3 stati:
\begin{list}{}{}
\item \textbf{5}: \textbf{perfepisco} cosa è accaduto (il feedback)
\item \textbf{6}: \textbf{interpreto} ciò che ho percepito. Non è detto che io abbia lo strumento corretto per interpretare o che il feedback sia sufficiente da essere interpretato
\item \textbf{7}: \textbf{comparo} il risultato dell'interpretazione con il mio obiettivo, che non devono differire
\end{list}
\columnbreak


\includegraphics[scale=0.45]{settestatiazione.png}
\end{multicols}
\begin{center}
\texttt{Goal $\rightarrow$  Plan $\rightarrow$ Specify $\rightarrow$ Perform $\rightarrow$ Percieve $\rightarrow$ Interpret $\rightarrow$ Compare}\\
\end{center}
Non è detto che tutti i comportamenti debbano passare da tutte le fasi sopra descritte. Per esempio, l'abitudine porta ad abilitare altre parti cervello e non rende più necessaria la specify o consente di saltare la pianificazione, perché al goal associo già la sequenza di azioni più veloce che conosco già.\\
Un'altra situazione è il caso in cui il feedback non si possa comparare al goal, che quindi abilita nuovo processo. Per esempio quando vado a concludere un acquisto ma non sono registrato, il feedback è "inserisci i dati per la registrazione" che non è compatibile con il mio obiettivo "acquista prodotto" $\rightarrow$ Inizia un nuovo processo)\\
Il trucco è sviluppare delle skill per utilizzare questo pardigma di analisi interazioni per capire se le nostre interfaccia performano bene o no.
\pagebreak
\subsubsection{Tre livelli di Processing}
\includegraphics[scale=0.5]{treprocessing.png}
visceral, behavioural e reflective\\
viscerale: clicco qui, seguo mouse e vedo su schermo, basilare\\
comportamentale: specifica e interpretazione, decide in che modo si vanno a fare le determinate task\\
riflessiva: emozioni, pianifico e comparo con la parte del cervello più emotiva

\section{Sette Principi Fondamentali della Progettazione}
\includegraphics[scale=0.5]{setteprincipi.png}
sette domande\\
Cosa voglio ottenere? dalle personas individuate genero come goal "voglio selezionare scarpa per modello", vedo che non ho messo scelta per modello nell'interfaccia\\
Quali sono le sequenze d'azione alternative che posso compiere per raggiungere comeunque obiettivo?\\
Che azione posso fare ora?\\
Come posso falra\\
cos'è successo\\
cosa significa\\\
va bene? ho raggiunto l'obiettivo?
\section{Feedforward e Feedback}
feedforward umano pensa di andare verso il dispositivo, è l'insieme di affordances, signficanti e mapping\\
feedback sistema verso umano\\\\
tutto questo si costituisce nei sette principi di design, non uno ad uno con processoi nterazione. è una checklist da fare prima del commit
\begin{list}{}{}
\item discoverable deve mettere utente nella psoizione di capire cosa posso fare e lo stato del dispositivo
\item feedback un sistema che dà feedbk sbagliati, non esplicativi, in ritardo, che annoiano ecc. è un sistema che ha problemi
\item modello concettuale, il modello concettuale che ho pensato è percepibile? regge? lo capisco solo io o tutti?
\item affordances il dispositivo/interfaccia abilità la proprietà dell'interazione \textbf{con le categorie di utenti con cui sto lavorando}. affordances esiste laddove è percebile, se non so se schermo è touch o no potenzialmente per me l'affordances non esiste
\item significanti, sono importanti perché abilitano discoverability funzionalità
\item mapping ho messo sign che vadano in contro a capacità mentali utente?
\item constraint limito libertà utente per il suo bene da usare con parsimonia
\end{list}
con questi sette principi si conclude parte dedicata a strumenti, mnetodi e elementi per design human computer interaction.\\
non sempre utenti eseguono azioni deliberatamente pianificate, ma eseguono azioni di tipo opportunistico. sono quelle in cui il comportamento scaturito dalle circostanze prevale sulle pianificazione e quindi su modo di essere e quindi su tutto il discorso precedente. può accadere che ho fatto tutto bene ma utente non si comporta in un modo previsto. Perché utenti agiscono in maniera incontrollata specialmente se guidati da opportunità.
L'opportunismo \textbf{rompe} lo schema d'interazione goal $->$ world
\end{document} 