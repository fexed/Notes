\documentclass[10pt]{article}
\usepackage[utf8]{inputenc}
\usepackage[italian]{babel}
\usepackage{multicol}
\usepackage[a4paper, total={18cm, 25cm}]{geometry}
\usepackage{amsfonts}
\begin{document}
\title{Analisi Matematica}
\author{Federico Matteoni}
\date{ }
\renewcommand*\contentsname{Indice}

\maketitle
\tableofcontents
\pagebreak
\section{Introduzione}
Appunti del corso di \textbf{Analisi Matematica} presi a lezione da \textbf{Federico Matteoni}.\\\\
Prof.: Leone Slavich \texttt{leone.slavich@unipi.it}\\
\begin{list}{-}{Riferimenti web:}
\item \texttt{people.dm.unipi.it/slavich/analisi-19-20.html}
\end{list}
Esame:
\begin{list}{-}{Libri e materiale didattico:}
\item Analisi Matematica ABC -- Acerchi, Buttazzo
\end{list}
Ricevimento: Mar 15-18 \textit{previa mail}
\section{Insiemi Numerici}
\begin{list}{}{}
\item $\mathbb{N}$ \textbf{numeri naturali}, cioè gli interi positivi compreso lo 0\\
$\mathbb{N}$ = \{0, 1, 2, 3, \ldots\}
\item $\mathbb{Z}$ \textbf{numeri interi} sia positivi che negativi\\
$\mathbb{Z}$ = \{\ldots, -3, -2, -1, 0, 1, 2, 3, \ldots\}
\item $\mathbb{Q}$ \textbf{numeri razionali}, le frazioni\\
$\mathbb{Q}$ = \{$\frac{p}{q}$ : p, q $\in \mathbb{Z}$, q $\neq$ 0\}\\
L'\textbf{espressione} di un numero razione \textbf{come rapporto} di interi \textbf{non è unica}\\
Ad es: $\frac{1}{2}$ = $\frac{4}{8}$, $\frac{3}{4}$ = $\frac{6}{8}$, $\frac{0}{1}$ = $\frac{0}{n} \forall n \neq 0$
\item $\mathbb{R}$ \textbf{numeri reali}\\
$\mathbb{R}$ contiene $\mathbb{Q}$ \textbf{e molto altro}, ad esempio $\sqrt{2}, \sqrt{3}, \sqrt[3]{5}, \pi, e$
\end{list}
\begin{center}
$\mathbb{N} \subset \mathbb{Z} \subset \mathbb{Q} \subset \mathbb{R}$
\end{center}
\subsection{Teorema}  $\sqrt{2} \not\in \mathbb{Q}$
$\sqrt{2}$ = \textit{l'unico elemento reale positivo x tale che} $x^2$ = 2\\
Dimostriamo che $\sqrt{2} \not\in \mathbb{Q}$ \textbf{per assurdo}, cioè \textbf{negando la tesi e cercando di giungere ad una contraddizione}.\\\\
Supponiamo \textit{per assurdo} che $\sqrt{2} \in \mathbb{Q}$,\\
pertanto $\exists p, q \in \mathbb{Z} : \sqrt{2} = \frac{p}{q}$ ($^1$) con $q \neq 0$ e supponiamo anche $p, q \in \mathbb{N}$ non nulli)\\
Posso supporre che p e q non siano entrambi pari.\\
Da ($^1$) 2 = $\frac{p^2}{q^2} \Rightarrow p^2 = 2q^2 \Rightarrow p^2$ è un numero pari.\\
$\Rightarrow$ p è pari $\Rightarrow \exists m \in \mathbb{N}$ : p = 2m $\Rightarrow p^2 = 4m^2$\\
Poiché $p^2 = 2q^2$ allora $4m^2 = 2q^2 \Rightarrow 2m^2 = q^2$\\
$\Rightarrow q^2$ è un numero pari $\Rightarrow q$ è pari.\\
$\Rightarrow$ \textbf{assurdo} perché avevamo supposto che p e q non fossero entrambi pari. $\square$
\subsection{Intervallo di R}
\paragraph{Def} $I \subset \mathbb{R}$ si dice \textbf{intervallo} se $\forall x, y \in I, x < y$ dato $z \in \mathbb{R}, x < z < y \Rightarrow z \in I$
\end{document}