\documentclass[10pt]{article}
\usepackage[utf8]{inputenc}
\usepackage[italian]{babel}
\usepackage{multicol}
\usepackage[a4paper, total={18cm, 25cm}]{geometry}
\usepackage{amsfonts}
\begin{document}
\title{Cheatsheet di Analisi Matematica}
\author{Federico Matteoni}
\date{ }
\renewcommand*\contentsname{Indice}

\maketitle
\tableofcontents
\pagebreak
\section{Intervalli}
\textbf{Intervallo} I = [x$_1$, x$_2$] $\subset$ $\mathbb{R}$ = (($\forall$ x$_1$, x$_2$ $\in$ I : x$_1$ $\leq$ x$_2$) $\wedge$ ($\forall$ x $\in \mathbb{R}$ : x$_1$ $\leq$ x $\leq$ x$_2$)) $\Rightarrow$ x $\in$ I
\begin{list}{}{}
\item \texttt{[n, m]}: chiuso, con estremi compresi
\item \texttt{]n, m[} o \texttt{(n, m)}: aperto, con estremi non compresi
\item \texttt{[n, m[} o \texttt{[n, m)}: semiaperto o semichiuso
\subsection{Estremi}
\begin{list}{-}{}
\item \textbf{Estremo superiore} L è estremo superiore se\\
L maggiorante di X $\forall$ x $\in$ X $\Rightarrow$ L $\geq$ x\\
$\wedge$\\
L è il minore dei maggioranti $\forall \epsilon >$ 0 $\exists \overline{x} \in$ X : L - $\epsilon < \overline{x}$
\item \textbf{Estremo inferiore} l è estremo inferiore se\\
l maggiorante di X $\forall$ x $\in$ X $\Rightarrow$ l $\leq$ x\\
$\wedge$\\
l è il minore dei maggioranti $\forall \epsilon >$ 0 $\exists \overline{x} \in$ X : L + $\epsilon > \overline{x}$
\end{list}
\end{list}
\end{document}