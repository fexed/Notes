\documentclass[10pt]{article}
\usepackage[utf8]{inputenc}
\usepackage[italian]{babel}
\usepackage{multicol}
\usepackage[a4paper, total={18cm, 25cm}]{geometry}
\usepackage{amsfonts}
\begin{document}
\title{Cheatsheet di Analisi Matematica}
\author{Federico Matteoni}
\date{ }
\renewcommand*\contentsname{Indice}

\maketitle
\textbf{Limiti Notevoli} --- Sostituire $x$ con $f(x)$ per ottenere lo stesso risultato
\begin{multicols}{4}
$$\lim_{x\to0} \frac{\sin(x)}{x} = 1$$\\
$$\lim_{x\to0} \frac{1 - \cos(x)}{x} = \frac{1}{2}$$\\
$$\lim_{x\to0} \frac{\tan(x)}{x} = 0$$\\
$$\lim_{x\to0} \frac{\ln(1 + x)}{x} = 1$$\\
$$\lim_{x\to0} \frac{\alpha^x - 1}{x} = \ln(\alpha)$$\\
$$\lim_{x\to\infty} \frac{(1 + x)^\alpha - 1}{x} = \alpha$$\\
$$\lim_{x\to\infty} (1 + \frac{\alpha}{x})^{nx} = e^{n\alpha}$$\\
$$\lim_{x\to\infty} (1 + \frac{1}{x})^{x} = e$$
\end{multicols}
\textbf{Derivate e integrali} di funzioni elementari
\begin{center}
	\begin{tabular}{c | c c}
		\textbf{Funzione} & \textbf{Derivata} & \textbf{Integrale} \\
		$x$ & $\frac{d}{dx} x = 1$ & $\int x\,dx = \frac{x^2}{2} + c$ \\
		\\
		$x^n$ & $\frac{d}{dx} x^n = n\,x^{n-1}$ & $\int x^n\,dx = \frac{x^{n+1}}{n+1} + c$ \\
		\\
		$\sin(x)$ & $\frac{d}{dx} \sin(x) = \cos(x)$ & $\int \sin(x)\,dx = -\cos(x) + c$ \\
		\\
		$\cos(x)$ & $\frac{d}{dx} \cos(x) = -\sin(x)$ & $\int \cos(x)\,dx = \sin(x) + c$ \\
		\\
		$\frac{1}{x}$ & $\frac{d}{dx} \frac{1}{x} = -\frac{1}{x^2}$ & $\int \frac{1}{x}\,dx = \ln(x) + c$ \\
		\\
		$\ln(x)$ & $\frac{d}{dx} \ln(x) = \frac{1}{x}$ & $\int \ln(x)\,dx = x(\ln(x) - 1) + c$ \textit{(per parti)} \\
	\end{tabular}
	\begin{multicols}{4}
		$\frac{d}{dx} \tan(x) = \frac{1}{\cos^2(x)}$ \\
		$\frac{d}{dx} \arctan(x) = \frac{1}{1 + x^2}$ \\
		$\frac{d}{dx} \arcsin(x) = \frac{1}{\sqrt[]{1 - x^2}}$\\
		$\frac{d}{dx} \arccos(x) = -\frac{1}{\sqrt[]{1 - x^2}}$\\
	\end{multicols}
\end{center}
\begin{multicols}{2}
\textbf{Derivate} -- Regole di derivazione
\begin{center}
	\begin{list}{}{}
		\item $\frac{d}{dx} f(g(x)) = f'(g(x)) \cdot g'(x)$
		\item $\frac{d}{dx} f(x) \cdot g(x) = f'(x)g(x) + f(x)g'(x)$
		\item $\frac{d}{dx} \frac{f(x)}{g(x)} = \frac{f'(x)g(x) - f(x)g'(x)}{g^2(x)}$
	\end{list}
\end{center}
\columnbreak
\textbf{Integrali} -- Regole di integrazione
\begin{center}
	\begin{list}{}{}
		\item $\int f(g(x))\cdot g'(x)\,dx = F(g(x)) + c$
		\item \textit{Integrale per parti}\\$\int f(x)\cdot g(x)\,dx = F(x)g(x) - \int F(x)\cdot g'(x)\,dx$
	\end{list}
\end{center}
\end{multicols}
\textbf{Differenziali} -- Soluzioni generali
\begin{multicols}{2}
	$y' = a(x)\cdot y + b(x)$\\
	$\Rightarrow y = e^{A(x)}\cdot \int e^{-A(x)}\cdot b(x)\,dx + c$\\
	con $A(x) = \int a(x)\,dx$\\\\

	$y' = a(x)\cdot f(y)$ \textit{Variabili separabili}\\
	$\Rightarrow \int \frac{1}{f(y(x))}\,dx = \int a(x)\,dx$\\
	\columnbreak
	
	$y'' + ay' + by = 0$ \textit{Di secondo grado}
	$\lambda^2 + a\lambda + b = 0 \rightarrow \lambda_1,\lambda_2$\\\\
	$\lambda_1 \neq \lambda_2 \in R$\,$\Rightarrow y(x) = c_1e^{\lambda_1x} + c_2e^{\lambda_2x}$\\\\
	$\lambda_1 = \lambda_2 \in R$\,$\Rightarrow y(x) = c_1e^{\lambda_1x} + xc_2e^{\lambda_1x}$\\\\
	$\lambda_1 = \alpha + i\beta,\, \lambda_2 = \alpha - i\beta$\\$\Rightarrow y(x) = e^{\alpha x}(c_1\cos(\beta x) + c_2\sin(\beta x))$
\end{multicols}
\begin{multicols}{4}
$\sin^2(x) + \cos^2(x) = 1$\\
$\sin(2x) = 2\sin(x)\cos(x)$\\
$\cos(2x) = \cos^2(x) - \sin^2(x)$\\
$\tan(2x) = \frac{2\tan(x)}{1 - \tan^2(x)} $
\end{multicols}
\pagebreak
\maketitle
\textbf{Studio di funzione}
\begin{enumerate}
	\item \textbf{Insieme di definizione}
	\begin{multicols} {3}
		$\frac{1}{f(x)} \Rightarrow f(x) \neq 0$\\
		$\log_a(f(x)) \Rightarrow f(x) > 0$\\
		$\sqrt[n]{f(x)} \Rightarrow f(x) \geq 0$ per $n$ pari
	\end{multicols}
	$$\arcsin(f(x))\, oppure\, \arccos(f(x)) \Rightarrow -1 \leq f(x) \leq 1$$
	\item \textbf{Asintoti}
	\begin{multicols}{3}
		\textbf{Orizzontali}\\
		$$\lim_{x\to\pm\infty} f(x) = l \neq \infty$$
		\columnbreak
		
		\textbf{Verticali}\\
		$$\lim_{x\to x_0} f(x) = \infty$$\\
		$x_0$ punto di discontinuità
		\columnbreak
		
		\textbf{Obliqui}\\
		$y(x) = mx + q$ asintoto obliquo se\\
		$$\lim_{x\to\pm\infty} \frac{f(x)}{x} = m \neq \infty, 0$$\\
		$$\lim_{x\to\pm\infty} f(x) - mx = q$$
	\end{multicols}
	\item \textbf{Derivate e segno}
	\begin{multicols}{2}
		$f'(x) = \lim_{x\to x_0} \frac{f(x) - f(x_0)}{x - x_0}$\\\\
		$f'(x_0) \geq 0 \Rightarrow f(x_0)$ \textbf{crescente}\\
		$f'(x_0) \leq 0 \Rightarrow f(x_0)$ \textbf{decrescente}\\
		$x_0$ \textbf{angoloso} se $f'_+(x_0) \neq f'_-(x_0)$ finiti\\
		$x_0$ \textbf{cuspide} se $f'_+(x_0) = \pm\infty \wedge f'_-(x_0) = \mp\infty$\\
		$x_0$ \textbf{minimo} se $f(x)$ decrescente $\rightarrow x_0 \rightarrow$ crescente\\
		$x_0$ \textbf{massimo} se $f(x)$ crescente $\rightarrow x_0 \rightarrow$ decrescente\\
		\columnbreak
		
		$f''(x)$\\\\
		$f''(x_0) > 0 \Rightarrow f(x_0)$ \textbf{convessa}\\
		$f''(x_0) < 0 \Rightarrow f(x_0)$ \textbf{concava}\\
		$x_0$ \textbf{flesso} se $f''(x_0) = 0$
	\end{multicols}
\end{enumerate}

\end{document}